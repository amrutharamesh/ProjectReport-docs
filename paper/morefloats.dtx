% \iffalse meta-comment
%
% File: morefloats.dtx
% Version: 2015/07/22 v1.0h
%
% Copyright (C) 2010 - 2015 by
%    H.-Martin M"unch <Martin dot Muench at Uni-Bonn dot de>
% Portions of code copyrighted by other people as marked.
%
% LaTeX 2015 provides the extrafloats command.
% Don Hosek, Quixote, 1990/07/27 (Thanks!)
% invented the main code for handling more floats
% before extrafloats was available.
% Maintenance has been taken over in September 2010
% by H.-Martin M\"{u}nch.
% David Carlisle pointed the maintainer to the new
% extrafloats command (Thanks!).
%
% This work may be distributed and/or modified under the
% conditions of the LaTeX Project Public License, either
% version 1.3c of this license or (at your option) any later
% version. This version of this license is in
%    http://www.latex-project.org/lppl/lppl-1-3c.txt
% and the latest version of this license is in
%    http://www.latex-project.org/lppl.txt
% and version 1.3c or later is part of all distributions of
% LaTeX version 2005/12/01 or later.
%
% This work has the LPPL maintenance status "maintained".
%
% The Current Maintainer of this work is H.-Martin Muench.
%
% This work consists of the main source file morefloats.dtx,
% the README, and the derived files
%    morefloats.sty, morefloats.pdf,
%    morefloats.ins, morefloats.drv,
%    morefloats-example.tex, morefloats-example.pdf.
%
% 'morefloats' is available on CTAN:
% https://www.ctan.org/pkg/morefloats
%
% Also a TDS.ZIP file is provided that contains all the files
% already sorted in a TDS tree:
% http://mirror.ctan.org/install/macros/latex/contrib/morefloats.tds.zip
%
%<*ignore>
\begingroup
  \catcode123=1 %
  \catcode125=2 %
  \def\x{LaTeX2e}%
\expandafter\endgroup
\ifcase 0\ifx\install y1\fi\expandafter
         \ifx\csname processbatchFile\endcsname\relax\else1\fi
         \ifx\fmtname\x\else 1\fi\relax
\else\csname fi\endcsname
%</ignore>
%<*install>
\input docstrip.tex
\Msg{*******************************************************************************}
\Msg{* Installation                                                                *}
\Msg{* Package: morefloats 2015/07/22 v1.0h Raise limit of unprocessed floats (HMM)*}
\Msg{*******************************************************************************}

\keepsilent
\askforoverwritefalse

\let\MetaPrefix\relax
\preamble

This is a generated file.

Project: morefloats
Version: 2015/07/22 v1.0h

Copyright (C) 2010 - 2015 by
    H.-Martin M"unch <Martin dot Muench at Uni-Bonn dot de>
Portions of code copyrighted by other people as marked.

The usual disclaimer applies:
If it doesn't work right that's your problem.
(Nevertheless, send an e-mail to the maintainer
 when you find an error in this package.)

This work may be distributed and/or modified under the
conditions of the LaTeX Project Public License, either
version 1.3c of this license or (at your option) any later
version. This version of this license is in
   http://www.latex-project.org/lppl/lppl-1-3c.txt
and the latest version of this license is in
   http://www.latex-project.org/lppl.txt
and version 1.3c or later is part of all distributions of
LaTeX version 2005/12/01 or later.

This work has the LPPL maintenance status "maintained".

The Current Maintainer of this work is H.-Martin Muench.

LaTeX 2015 provides the extrafloats command.
Don Hosek, Quixote, 1990/07/27 (Thanks!)
invented the main code for handling more floats
before extrafloats was available.
Maintenance has been taken over in September 2010
by H.-Martin Muench.
David Carlisle pointed the maintainer to the new
extrafloats command (Thanks!).

This work consists of the main source file morefloats.dtx,
the README, and the derived files
   morefloats.sty, morefloats.pdf,
   morefloats.ins, morefloats.drv,
   morefloats-example.tex, morefloats-example.pdf.

In memoriam
 Claudia Simone Barth + 1996/01/30
 Tommy Muench + 2014/01/02
 Hans-Klaus Muench + 2014/08/24

\endpreamble
\let\MetaPrefix\DoubleperCent

\generate{%
  \file{morefloats.ins}{\from{morefloats.dtx}{install}}%
  \file{morefloats.drv}{\from{morefloats.dtx}{driver}}%
  \usedir{tex/latex/morefloats}%
  \file{morefloats.sty}{\from{morefloats.dtx}{package}}%
  \usedir{doc/latex/morefloats}%
  \file{morefloats-example.tex}{\from{morefloats.dtx}{example}}%
}

\catcode32=13\relax% active space
\let =\space%
\Msg{************************************************************************}
\Msg{*}
\Msg{* To finish the installation you have to move the following}
\Msg{* file into a directory searched by TeX:}
\Msg{*}
\Msg{*  morefloats.sty}
\Msg{*}
\Msg{* To produce the documentation run the file `morefloats.drv'}
\Msg{* through (pdf)LaTeX, e.g.}
\Msg{*  pdflatex morefloats.drv}
\Msg{*  makeindex -s gind.ist morefloats.idx}
\Msg{*  pdflatex morefloats.drv}
\Msg{*  makeindex -s gind.ist morefloats.idx}
\Msg{*  pdflatex morefloats.drv}
\Msg{*}
\Msg{* At least three runs are necessary e.g. to get the}
\Msg{*  references right!}
\Msg{*}
\Msg{* Happy TeXing!}
\Msg{*}
\Msg{************************************************************************}

\endbatchfile
%</install>
%<*ignore>
\fi
%</ignore>
%
% \section{The documentation driver file}
%
% The next bit of code contains the documentation driver file for
% \TeX , i.\,e., the file that will produce the documentation you
% are currently reading. It will be extracted from this file by the
% \texttt{docstrip} programme. That is, run \LaTeX{} on \texttt{docstrip}
% and specify the \texttt{driver} option when \texttt{docstrip}
% asks for options.
%
%    \begin{macrocode}
%<*driver>
\NeedsTeXFormat{LaTeX2e}[2015/01/01]
\ProvidesFile{morefloats.drv}%
  [2015/07/22 v1.0h Raise limit of unprocessed floats (HMM)]
\documentclass{ltxdoc}[2015/03/26]%   v2.0w
\usepackage[T1]{fontenc}[2005/09/27]% v1.99g
\usepackage{pdflscape}[2008/08/11]%   v0.10
\usepackage{holtxdoc}[2012/03/21]%    v0.24
%% morefloats should work with earlier versions of LaTeX2e and
%% may work with earlier versions of the class and those packages,
%% but this was not tested.
%% Please consider updating your LaTeX, class, and packages
%% to the most recent version (if they are not already the most
%% recent version).
\hypersetup{%
 pdfsubject={LaTeX2e package for increasing the limit of unprocessed floats (HMM)},%
 pdfkeywords={LaTeX, morefloats, floats, H.-Martin Muench},%
 pdfencoding=auto,%
 pdflang={en},%
 breaklinks=true,%
 linktoc=all,%
 pdfstartview=FitH,%
 pdfpagelayout=OneColumn,%
 bookmarksnumbered=true,%
 bookmarksopen=true,%
 bookmarksopenlevel=2,%
 pdfmenubar=true,%
 pdftoolbar=true,%
 pdfwindowui=true,%
 pdfnewwindow=true%
}
\CodelineIndex
\hyphenation{docu-ment}
\gdef\unit#1{\mathord{\thinspace\mathrm{#1}}}%
\begin{document}
  \DocInput{morefloats.dtx}%
\end{document}
%</driver>
%    \end{macrocode}
%
% \fi
%
% \CheckSum{3565}
%
% \CharacterTable
%  {Upper-case    \A\B\C\D\E\F\G\H\I\J\K\L\M\N\O\P\Q\R\S\T\U\V\W\X\Y\Z
%   Lower-case    \a\b\c\d\e\f\g\h\i\j\k\l\m\n\o\p\q\r\s\t\u\v\w\x\y\z
%   Digits        \0\1\2\3\4\5\6\7\8\9
%   Exclamation   \!     Double quote  \"     Hash (number) \#
%   Dollar        \$     Percent       \%     Ampersand     \&
%   Acute accent  \'     Left paren    \(     Right paren   \)
%   Asterisk      \*     Plus          \+     Comma         \,
%   Minus         \-     Point         \.     Solidus       \/
%   Colon         \:     Semicolon     \;     Less than     \<
%   Equals        \=     Greater than  \>     Question mark \?
%   Commercial at \@     Left bracket  \[     Backslash     \\
%   Right bracket \]     Circumflex    \^     Underscore    \_
%   Grave accent  \`     Left brace    \{     Vertical bar  \|
%   Right brace   \}     Tilde         \~}
%
% \GetFileInfo{morefloats.drv}
%
% \begingroup
%   \def\x{\#,\$,\^,\_,\~,\ ,\&,\{,\},\%}%
%   \makeatletter
%   \@onelevel@sanitize\x
% \expandafter\endgroup
% \expandafter\DoNotIndex\expandafter{\x}
% \expandafter\DoNotIndex\expandafter{\string\ }
% \begingroup
%   \makeatletter
%     \lccode`9=32\relax
%     \lowercase{%^^A
%       \edef\x{\noexpand\DoNotIndex{\@backslashchar9}}%^^A
%     }%^^A
%   \expandafter\endgroup\x
% \DoNotIndex{\\,\,}
% \DoNotIndex{\def,\edef,\gdef, \xdef}
% \DoNotIndex{\ifnum, \ifx}
% \DoNotIndex{\begin, \end, \LaTeX, \LateXe}
% \DoNotIndex{\bigskip, \caption, \centering, \hline, \MessageBreak}
% \DoNotIndex{\documentclass, \markboth, \mathrm, \mathord}
% \DoNotIndex{\NeedsTeXFormat, \usepackage, \ProvidesPackage, \RequirePackage}
% \DoNotIndex{\newline, \newpage, \pagebreak}
% \DoNotIndex{\section, \subsection, \space, \thinspace}
% \DoNotIndex{\textsf, \texttt}
% \DoNotIndex{\the, \@tempcnta,\@tempcntb}
% \DoNotIndex{\@elt,\@freelist, \newinsert}
% \DoNotIndex{\bx@A,  \bx@B,  \bx@C,  \bx@D,  \bx@E,  \bx@F,  \bx@G,  \bx@H,  \bx@I,  \bx@J,  \bx@K,  \bx@L,  \bx@M,  \bx@N,  \bx@O,  \bx@P,  \bx@Q,  \bx@R,  \bx@S,  \bx@T,  \bx@U,  \bx@V,  \bx@W,  \bx@X,  \bx@Y,  \bx@Z}
% \DoNotIndex{\bx@AA, \bx@AB, \bx@AC, \bx@AD, \bx@AE, \bx@AF, \bx@AG, \bx@AH, \bx@AI, \bx@AJ, \bx@AK, \bx@AL, \bx@AM, \bx@AN, \bx@AO, \bx@AP, \bx@AQ, \bx@AR, \bx@AS, \bx@AT, \bx@AU, \bx@AV, \bx@AW, \bx@AX, \bx@AY, \bx@AZ}
% \DoNotIndex{\bx@BA, \bx@BB, \bx@BC, \bx@BD, \bx@BE, \bx@BF, \bx@BG, \bx@BH, \bx@BI, \bx@BJ, \bx@BK, \bx@BL, \bx@BM, \bx@BN, \bx@BO, \bx@BP, \bx@BQ, \bx@BR, \bx@BS, \bx@BT, \bx@BU, \bx@BV, \bx@BW, \bx@BX, \bx@BY, \bx@BZ}
% \DoNotIndex{\bx@CA, \bx@CB, \bx@CC, \bx@CD, \bx@CE, \bx@CF, \bx@CG, \bx@CH, \bx@CI, \bx@CJ, \bx@CK, \bx@CL, \bx@CM, \bx@CN, \bx@CO, \bx@CP, \bx@CQ, \bx@CR, \bx@CS, \bx@CT, \bx@CU, \bx@CV, \bx@CW, \bx@CX, \bx@CY, \bx@CZ}
% \DoNotIndex{\bx@DA, \bx@DB, \bx@DC, \bx@DD, \bx@DE, \bx@DF, \bx@DG, \bx@DH, \bx@DI, \bx@DJ, \bx@DK, \bx@DL, \bx@DM, \bx@DN, \bx@DO, \bx@DP, \bx@DQ, \bx@DR, \bx@DS, \bx@DT, \bx@DU, \bx@DV, \bx@DW, \bx@DX, \bx@DY, \bx@DZ}
% \DoNotIndex{\bx@EA, \bx@EB, \bx@EC, \bx@ED, \bx@EE, \bx@EF, \bx@EG, \bx@EH, \bx@EI, \bx@EJ, \bx@EK, \bx@EL, \bx@EM, \bx@EN, \bx@EO, \bx@EP, \bx@EQ, \bx@ER, \bx@ES, \bx@ET, \bx@EU, \bx@EV, \bx@EW, \bx@EX, \bx@EY, \bx@EZ}
% \DoNotIndex{\bx@FA, \bx@FB, \bx@FC, \bx@FD, \bx@FE, \bx@FF, \bx@FG, \bx@FH, \bx@FI, \bx@FJ, \bx@FK, \bx@FL, \bx@FM, \bx@FN, \bx@FO, \bx@FP, \bx@FQ, \bx@FR, \bx@FS, \bx@FT, \bx@FU, \bx@FV, \bx@FW, \bx@FX, \bx@FY, \bx@FZ}
% \DoNotIndex{\bx@GA, \bx@GB, \bx@GC, \bx@GD, \bx@GE, \bx@GF, \bx@GG, \bx@GH, \bx@GI, \bx@GJ, \bx@GK, \bx@GL, \bx@GM, \bx@GN, \bx@GO, \bx@GP, \bx@GQ, \bx@GR, \bx@GS, \bx@GT, \bx@GU, \bx@GV, \bx@GW, \bx@GX, \bx@GY, \bx@GZ}
% \DoNotIndex{\bx@HA, \bx@HB, \bx@HC, \bx@HD, \bx@HE, \bx@HF, \bx@HG, \bx@HH, \bx@HI, \bx@HJ, \bx@HK, \bx@HL, \bx@HM, \bx@HN, \bx@HO, \bx@HP, \bx@HQ, \bx@HR, \bx@HS, \bx@HT, \bx@HU, \bx@HV, \bx@HW, \bx@HX, \bx@HY, \bx@HZ}
% \DoNotIndex{\bx@IA, \bx@IB, \bx@IC, \bx@ID, \bx@IE, \bx@IF, \bx@IG, \bx@IH, \bx@II, \bx@IJ, \bx@IK, \bx@IL, \bx@IM, \bx@IN, \bx@IO, \bx@IP, \bx@IQ, \bx@IR, \bx@IS, \bx@IT, \bx@IU, \bx@IV, \bx@IW, \bx@IX, \bx@IY, \bx@IZ}
% \DoNotIndex{\bx@JA, \bx@JB, \bx@JC, \bx@JD, \bx@JE, \bx@JF, \bx@JG, \bx@JH, \bx@JI, \bx@JJ, \bx@JK, \bx@JL, \bx@JM, \bx@JN, \bx@JO, \bx@JP, \bx@JQ, \bx@JR, \bx@JS, \bx@JT, \bx@JU, \bx@JV, \bx@JW, \bx@JX, \bx@JY, \bx@JZ}
% \DoNotIndex{\morefloats@mx}
%
% \title{The \xpackage{morefloats} package}
% \date{2015/07/22 v1.0h}
% \author{H.-Martin M\"{u}nch (current maintainer;\\
%  invented by Don Hosek, Quixote)\\
%  \xemail{Martin.Muench at Uni-Bonn.de}}
%
% \maketitle
%
% \begin{abstract}
% The default limit of unprocessed floats, $18$,
% can be increased with this \xpackage{morefloats} package.
% Otherwise, |\clear(double)page|, |h(!)|, |H|~from the \xpackage{float} package,
% or |\FloatBarrier| from the \xpackage{picins} package might help.
% \end{abstract}
%
% \bigskip
%
% \noindent Note: \LaTeX{} 2015 provides the |\extrafloats| command.
% \textsc{Don Hosek}, Quixote, 1990/07/27 (Thanks!)
% invented the main code for handling more floats
% before |\extrafloats| was available.
% \textsc{David Carlisle} pointed the maintainer to the new
% |\extrafloats| (Thanks!).
% The current maintainer is \textsc{H.-Martin M\"{u}nch}.\\
%
% \bigskip
%
% \noindent Disclaimer for web links: The author is not responsible for any contents
% referred to in this work unless he has full knowledge of illegal contents.
% If any damage occurs by the use of information presented there, only the
% author of the respective pages might be liable, not the one who has referred
% to these pages.
%
% \bigskip
%
% \noindent {\color{green} Save per page about $200\unit{ml}$ water,
% $2\unit{g}$ CO$_{2}$ and $2\unit{g}$ wood:\\
% Therefore please print only if this is really necessary.}
%
% \newpage
%
% \tableofcontents
%
% \newpage
%
% \section{Introduction\label{sec:Introduction}}
%
% The default limit of unprocessed floats, $18$,
% can be increased with this \xpackage{morefloats} package.\\
% \textquotedblleft{}Of course one immediately begins to wonder:
% \guillemotright{}Why eighteen?!\guillemotleft{} And it turns out that $18$
% one{-}line tables with $10$~point Computer Modern using \xclass{article.cls}
% produces almost exactly one page worth of material.\textquotedblright{}\\
% (user \url{https://tex.stackexchange.com/users/1495/kahen} as comment to\\
% \url{https://tex.stackexchange.com/a/35596/6865} on 2011/11/21)\\
% As alternatives (see also section \ref{sec:alternatives} below)
% |\clear(double)page|, |h(!)|, |H|~from the
% \href{https://www.ctan.org/pkg/float}{\xpackage{float}} package,
% or |\FloatBarrier| from the %
% \href{https://www.ctan.org/pkg/picins}{\xpackage{picins}} package might help.
% If the floats cannot be placed anywhere at all, extending the number of floats
% will just delay the arrival of the corresponding error.
%
% \section{Usage}
%
% \subsection{General usage:}
% Load the package placing
% \begin{quote}
%   |\usepackage[<|\textit{options}|>]{morefloats}|
% \end{quote}
% \noindent in the preamble of your \LaTeXe{} source file (the earlier the better).\\
% \noindent The \xpackage{morefloats} package takes two options: |maxfloats| and
% |morefloats|, where |morefloats| gives the number of additional floats and
% |maxfloats| gives the maximum number of floats. |maxfloats=25| therefore means,
% that there are $18$ (default) floats and $7$ additional floats.
% |morefloats=7| therefore has the same meaning. It is only necessary to give
% one of these two options. At the time being, it is not possible to reduce
% the number of floats (for example to save boxes). If you have code
% accomplishing that, please send it to the package maintainer, thanks.\\
% Version 1.0b used a fixed value of |maxfloats=36|. Therefore for backward
% compatibility this value is taken as the default one.\\
% Example:
% \begin{quote}
%   |\usepackage[maxfloats=25]{morefloats}|
% \end{quote}
% or
% \begin{quote}
%   |\usepackage[morefloats=7]{morefloats}|
% \end{quote}
% or
% \begin{quote}
%   |\usepackage[maxfloats=25,morefloats=7]{morefloats}|
% \end{quote}
%
% \subsection{Situation for \LaTeX{} before 2015:}
% |Float| uses |insert|, and each |insert| uses a group of |count|, |dimen|,
% |skip|, and |box| each. When there are not enough available, no |\newinsert|
% can be created. The
% \href{https://www.ctan.org/pkg/etex-pkg}{\xpackage{etex}} package
% provides access at an extended range of those registers,
% but does not use those for |\newinsert|. Therefore the inserts must be
% reserved first, which forces the use of the extended register range
% for other new |count|, |dimen|, |skip|, and |box|:
% To have more floats available, use |\usepackage{etex}\reserveinserts{...}|
% right after |\documentclass[...]{...}|, where the argument of |\reserveinserts|
% should be at least the maximum number of floats. Add another $10$
% if the \href{https://www.ctan.org/pkg/bigfoot}{\xpackage{bigfoot}} or the
% \href{https://www.ctan.org/pkg/manyfoot}{\xpackage{manyfoot}} package
% is used, but |\reserveinserts| can be about $234$ at most for older
% \LaTeX{} formats.
%
% \subsection{Situation for \LaTeX{} since 2015:}
% Now |\reserveinserts| can be about $2\,147\,483\,647$,
% but |\insert255{}| even then produces an error.
% The \LaTeX{} 2015 \textquotedblleft release provides a new command in the format
% |\extrafloats|\textquotedblright ; \textquotedblleft as it doesn't use
% |\newinsert| (and as the 2015 format uses extended registers by default)
% you can allocate a lot more floats\textquotedblright{} %
% (both \textsc{David Carlisle}, 29. June 2015), \hbox{e.\,g. |\extrafloats{1234}|.}
%
% \section{Alternatives (kind of)\label{sec:alternatives}}
%
% The very old \xpackage{morefloats} with a fixed number of |maxfloats=36| {}%
% \hbox{(i.\,e. $18$ |morefloats|)} has been archived at
% \href{http://mirror.ctan.org/obsolete/macros/latex/contrib/misc/morefloats.sty}{%
%  http://mirror.ctan.org/obsolete/macros/latex/contrib/}\newline%
% \href{http://mirror.ctan.org/obsolete/macros/latex/contrib/misc/morefloats.sty}{%
%  misc/morefloats.sty}.
%
% \bigskip
%
% If you really want to increase the number of (possible) floats,
% this is the right package. On the other hand, if you ran into trouble of
% \texttt{Too many unprocessed floats}, but would also accept less floats,
% there are some other possibilities:
% \begin{description}
%   \item[-] The command |\clearpage| forces \LaTeX{} to output any floating objects
%     that occurred before this command (and go to the next page).
%     |\cleardoublepage| does the same but ensures that the next page with
%     output is one with odd page number.
%   \item[-] Using different float specifiers: |t|~top, |b|~bottom, |p|~page
%     of floats.
%   \item[-] Suggesting \LaTeX{} to put the object where it was placed:
%     |h| (= here) float specifier.
%   \item[-] Telling \LaTeX{} to please put the object where it was placed:
%     |h!| (= here!) float specifier.
%   \item[-] Forcing \LaTeX{} to put the object where it was placed and shut up:
%     The \xpackage{float} package provides the \textquotedblleft style
%     option here, giving floating environments a [H] option which means
%     `PUT IT HERE' (as opposed to the standard [h] option which means
%     `You may put it here if you like')\textquotedblright{} (\xpackage{float}
%     package documentation v1.3d as of 2001/11/08).
%     Changing e.\,g. |\begin{figure}[tbp]...| to |\begin{figure}[H]...|
%     forces the figure to be placed HERE instead of floating away.\\
%     The \xpackage{float} package is available at \url{https://www.ctan.org/pkg/float}.
%   \item[-] The \xpackage{placeins} package provides the command |\FloatBarrier|.
%     Floats occurring before the |\FloatBarrier| are not allowed to float
%     to a later place, and floats occurring after the |\FloatBarrier| are not
%     allowed to float to an earlier place than the |\FloatBarrier|. (There
%     can be more than one |\FloatBarrier| in a document.) -- %
%     The same package also provides an option to automatically add |\FloatBarrier|s to
%     section headings. It is further possible to make
%     |\FloatBarrier|s less strict (see that package's documentation).\\
%     The \xpackage{placeins} package is available at \url{https://www.ctan.org/pkg/placeins}.
%   \item[-] Sometimes also increasing the maximum number (|\maxdeadcycles|)
%     of calls of |\output| without a |\shipout| can help,
%     for example |\maxdeadcycles=123\relax|.
% \end{description}
%
% \newpage
%
% \noindent See also the following entries in the
% \texttt{UK~List of TeX Frequently Asked Questions on the Web}:
% \begin{description}
%   \item[-] \url{http://www.tex.ac.uk/cgi-bin/texfaq2html?label=floats}
%   \item[-] \url{http://www.tex.ac.uk/cgi-bin/texfaq2html?label=tmupfl}
%   \item[-] \url{http://www.tex.ac.uk/cgi-bin/texfaq2html?label=figurehere}
% \end{description}
% and the \textbf{excellent article on \textquotedblleft How to influence the position
% of float environments like figure and table in \hbox{\LaTeX ?\textquotedblright } by
% \textsc{Frank Mittelbach}} at \url{https://tex.stackexchange.com/a/39020/6865}{}!\\
%
% \bigskip
%
% \noindent (You programmed or found another alternative,
%  which is available at CTAN?\\
%  OK, send an e-mail to me with the name, location at CTAN,
%  and a short notice, and I will probably include it in
%  the list above.)
%
% \bigskip
%
% \section{Example}
%
%    \begin{macrocode}
%<*example>
\documentclass[british]{article}[2014/09/29]%      v1.4h
%%%%%%%%%%%%%%%%%%%%%%%%%%%%%%%%%%%%%%%%%%%%%%%%%%%%%%%%%%%%%%%%%%%%%
\usepackage[maxfloats=25]{morefloats}[2015/07/22]% v1.0h
%\maxdeadcycles=200\relax%
%% \maxdeadcycles is the maximum number of calls of \output
%% without a \shipout.
\gdef\unit#1{\mathord{\thinspace\mathrm{#1}}}%
\listfiles
\begin{document}

\makeatletter

\section*{Example for morefloats}
\markboth{Example for morefloats}{Example for morefloats}

This example demonstrates the use of package\newline
\textsf{morefloats}, v1.0h as of 2015/07/22 (HMM).\newline
The package takes options (here:
\verb|maxfloats=|\texttt{\morefloats@maxfloats} is used).\newline
For more details please see the documentation!\newline

To reproduce the\newline
\LaTeX{} \texttt{ Error: Too many unprocessed floats},\newline
comment out the \verb|\usepackage...| in the preamble
(line~3)\newline
(by placing a \% before it).\newline

\bigskip

Save per page about $200\unit{ml}$~water, $2\unit{g}$~CO$_{2}$
and $2\unit{g}$~wood:\newline
Therefore please print only if this is really necessary.\newline
I do NOT think, that it is necessary to print THIS file, really!

\bigskip

There follow \morefloats@maxfloats{} floating tables.

\pagebreak

\@tempcnta=18\relax% default floats
\advance\@tempcnta by \morefloats@morefloats%
% \morefloats@morefloats is the number of additional
% floating tables to create.
\loop
  \ifnum\@tempcnta>0\relax%
  \begin{table}[t]\centering%
    \begin{tabular}{|l|}%
      \hline%
      A table, which will keep floating.\\%
      \hline
    \end{tabular}%
    \caption{A floating Table.}%
  \end{table}%
  \advance\@tempcnta by -1\relax%
\repeat

\makeatother

\end{document}
%</example>
%    \end{macrocode}
%
% \newpage
%
% \StopEventually{}
%
% \section{The implementation}
%
% We start off by checking that we are loading into \LaTeXe{} and
% announcing the name and version of this package.
%
%    \begin{macrocode}
%<*package>
%    \end{macrocode}
%
%    \begin{macrocode}
\NeedsTeXFormat{LaTeX2e}[2011/06/27]
%% The current format at the time of the release of this version of the
%% morefloats package was 2015/01/01, patch level 2.
\ProvidesPackage{morefloats}[2015/07/22 v1.0h
            Raise limit of unprocessed floats (HMM)]

%    \end{macrocode}
%
% \DescribeMacro{Options}
%    \begin{macrocode}
\RequirePackage{kvoptions}[2011/06/30]% v3.11
%% morefloats may work with earlier versions of LaTeX2e and that
%% package, but this was not tested.
%% Please consider updating your LaTeX and package
%% to the most recent version (if they are not already the most
%% recent version).

\SetupKeyvalOptions{family=morefloats,prefix=morefloats@}
\DeclareStringOption{maxfloats}%  \morefloats@maxfloats
\DeclareStringOption{morefloats}% \morefloats@morefloats

\ProcessKeyvalOptions*

%    \end{macrocode}
%
% The \xpackage{morefloats} package takes two options: |maxfloats| and |morefloats|,
% where |morefloats| gives the number of additional floats and |maxfloats| gives
% the maximum number of floats. |maxfloats=37| therefore means, that there are
% $18$ (default) floats and another $19$ additional floats. |morefloats=19| therefore
% has the same meaning. Version~1.0b used a fixed value of |maxfloats=36|.
% Therefore for backward compatibility this value will be taken as the default one.\\
% Now we check whether |maxfloats=...| or |morefloats=...| or both were used,
% and if one option was not used, we supply the according value.
% If no option was used at all, we use the default values.
% Too many requested floats produce error massages by \LaTeX ,
% which might not be easily traced back to this,
% therefore we issue a warning. If option |maxfloats| or |morefloats| is no number,
% the user will received the according error message by \LaTeX{} automatically.
%
%    \begin{macrocode}
\ifx\morefloats@maxfloats\@empty%
  \ifx\morefloats@morefloats\@empty% apply defaults:
    \gdef\morefloats@maxfloats{36}%
    \gdef\morefloats@morefloats{18}%
  \else%
    \ifnum\morefloats@morefloats>1569\relax%
      \PackageWarning{morefloats}{%
        \morefloats@morefloats\space more floats requested.\MessageBreak%
        LaTeX might run out of memory before this\MessageBreak%
        (in which case it will notify you)\MessageBreak%
       }%
    \else%
      \PackageInfo{morefloats}{%
        \morefloats@morefloats\space more floats requested.\MessageBreak%
        LaTeX might run out of memory before this\MessageBreak%
        (in which case it will notify you)\MessageBreak%
       }%
    \fi%
    \@tempcnta=\morefloats@morefloats\relax%
    \advance\@tempcnta by +18%
    \xdef\morefloats@maxfloats{\the\@tempcnta}%
  \fi%
\else%
  \ifx\morefloats@morefloats\@empty%
    \@tempcnta=\morefloats@maxfloats\relax%
    \advance\@tempcnta by -18%
    \xdef\morefloats@morefloats{\the\@tempcnta}%
    \ifnum\morefloats@morefloats<\z@\relax% i.e. \morefloats@maxfloats < 18
      \gdef\morefloats@morefloats{0}%
    \fi%
    \ifnum\morefloats@maxfloats>1587\relax%
      \PackageWarning{morefloats}{%
        \morefloats@maxfloats\space floats requested.\MessageBreak%
        LaTeX might run out of memory before this\MessageBreak%
        (in which case it will notify you)\MessageBreak%
       }%
    \fi%
  \fi%
\fi%

\@tempcnta=\morefloats@maxfloats\relax%
\xdef\morefloats@max{\the\@tempcnta}%

\ifnum\@tempcnta<18\relax%
  \PackageError{morefloats}{Option maxfloats is \the\@tempcnta<18}{%
    maxfloats must be a number equal to or larger than 18\MessageBreak%
    (or not used at all).\MessageBreak%
    Now setting maxfloats=18.\MessageBreak%
   }%
  \gdef\morefloats@max{18}%
\fi%

\@tempcnta=\morefloats@morefloats\relax%
\xdef\morefloats@more{\the\@tempcnta}%

\ifnum\@tempcnta<\z@\relax%
  \PackageError{morefloats}{Option morefloats is \the\@tempcnta<0}{%
    morefloats must be a number equal to or larger than 0\MessageBreak%
    (or not used at all).\MessageBreak%
    Now setting morefloats=0.\MessageBreak%
   }%
  \gdef\morefloats@more{0}%
\fi%

\@tempcnta=18\relax%
\advance\@tempcnta by \morefloats@more%
%    \end{macrocode}
%
% The value of |morefloats| should now be equal to the value of |morefloats@max|.
%
%    \begin{macrocode}
\advance\@tempcnta by -\morefloats@max%
%    \end{macrocode}
%
% Therefore |\@tempcnta| should now be equal to zero.
%
%    \begin{macrocode}
\xdef\morefloats@mx{\the\@tempcnta}%
\ifnum\morefloats@mx=\z@\relax%
  \@tempcnta=\morefloats@maxfloats\relax%
\else%
  \PackageError{morefloats}{%
    Clash between options maxfloats and morefloats}{%
    Option maxfloats must be empty\MessageBreak%
    or the sum of 18 and option value morefloats,\MessageBreak%
    but it is maxfloats=\morefloats@maxfloats\space and %
    morefloats=\morefloats@morefloats .\MessageBreak%
    }%
%    \end{macrocode}
%
% We choose the larger value to be used.
%
%    \begin{macrocode}
  \ifnum\@tempcnta<\z@% \morefloats@max > \morefloats@more
    \@tempcnta=\morefloats@maxfloats\relax%
  \else% \@tempcnta>0, \morefloats@max < \morefloats@more
    \@tempcnta=18\relax%
    \advance\@tempcnta by \morefloats@morefloats%
  \fi%
\fi%
\edef\morefloats@mx{\the\@tempcnta}%
%    \end{macrocode}
%
% Maybe we had to change |\morefloats@maxfloats| or |\morefloats@maxfloats|:
%
%    \begin{macrocode}
\xdef\morefloats@maxfloats{\the\@tempcnta}%
\advance\@tempcnta by -18\relax%
\xdef\morefloats@morefloats{\the\@tempcnta}%
\gdef\morefloats@test{1}%
\ifx\morefloats@morefloats\morefloats@test\relax%
  \PackageInfo{morefloats}{%
    Maximum number of possible floats asked for: \morefloats@maxfloats%
    \MessageBreak%
    (i.e. one more float)\@gobble%
   }%
\else%
  \PackageInfo{morefloats}{%
    Maximum number of possible floats asked for: \morefloats@maxfloats%
    \MessageBreak%
    (i.e. \morefloats@morefloats\space more floats).\MessageBreak%
    LaTeX might run out of memory before this\MessageBreak%
    (in which case it will notify you)%
    \@gobble%
   }%
\fi%


%    \end{macrocode}
%
% The \LaTeX{} 2015 \textquotedblleft release provides a new command in the format
% |\extrafloats| which does a similar job [as earlier versions of this package did],
% although as it doesn't use |\newinsert| (and as the 2015 format uses extended
% registers by default) you can allocate a lot more floats,\textquotedblright{} %
% \hbox{e.\,g. |\extrafloats{1234}|.} Loading the \xpackage{etex} package and
% \xpackage{morefloats} with the new format would
% \textquotedblleft over{-}write the new allocation mechanism and end up with
% fewer floats available.\textquotedblright{} Therefore here it is tested
% \textquotedblleft for the new format and switch[ed] to the new mechanism
% in that case, so that existing documents work as before but using the new allocation
% scheme underneath.\textquotedblright{} (all \textsc{David Carlisle}, 29. June 2015,
% who provided also main parts of the following code)
%
%    \begin{macrocode}
%% Test for new mechanism in LaTeX 2015:
\ifx\e@alloc\@undefined\relax%
  %% This is an old LaTeX format, \extrafloats is not available.
  \PackageWarning{morefloats}{%
    \fmtname\space <\fmtversion> %
    \ifx\patch@level\@undefined\relax%
    \else patch level \patch@level%
    \fi%
    \MessageBreak%
    found. At least\MessageBreak%
    LaTeX2e <2015/01/01> patch level 2\MessageBreak%
    is now available\MessageBreak%
    and can handle even more floats%
    \@gobble%
   }%
\else%
  %% This is new in LaTeX 2015, \extrafloats is available.
  \@ifpackageloaded{etex}%
  {%% etex package loaded:
   %% "it overwrites all the new allocation system
   %% so really \extrafloats shouldn't be expected to work"
   %% (D. Carlisle, 2015/07/16, who also provided the following
   %% \extrafloats redefinition).
   \gdef\extrafloats#1{%
     \ifnum#1>\z@\relax%
       \count@\numexpr\float@count-1\relax%
       \ch@ck0\count@\count\relax%
       \ch@ck1\count@\dimen\relax%
       \ch@ck2\count@\skip\relax%
       \ch@ck4\count@\box\relax%
       \e@alloc@chardef\float@count\count@%
       \expandafter\e@alloc@chardef\csname bx@\the\float@count\endcsname\float@count%
       \@cons\@freelist{\csname bx@\the\float@count\endcsname}%
       \expandafter%
       \extrafloats\expandafter{\numexpr#1-1\relax}%
     \fi%
   }%
  }{% etex package not loaded
   }%
  \extrafloats{\morefloats@morefloats}%
  % The part after the test is no longer needed and therefore not loaded:
  \expandafter\endinput%
\fi%
%% End of the test for LaTeX 2015 (or newer).
%% Not new format, otherwise the last \endinput would have been applied.

%% Test for e-TeX:
\RequirePackage{ifetex}[2011/12/15]% v1.2
\ifetex%
  %% then we can use code similar to the one from David Carlisle,
  %% https://tex.stackexchange.com/a/212483/6865
  \mathchardef\float@count=32767\relax%
  \gdef\extrafloats#1{%
    \ifnum#1>\z@\relax%
      \count@\numexpr\float@count-1\relax%
      \ch@ck0\count@\count\relax%
      \ch@ck1\count@\dimen\relax%
      \ch@ck2\count@\skip\relax%
      \ch@ck4\count@\box\relax%
      \mathchardef\float@count\count@\relax%
      \expandafter\mathchardef\csname bx@\the\float@count\endcsname\float@count%
      \@cons\@freelist{\csname bx@\the\float@count\endcsname}%
      \expandafter%
      \extrafloats\expandafter{\numexpr#1-1\relax}%
    \fi}%
  \extrafloats{\morefloats@morefloats}%
  \expandafter\endinput%
\fi%
%% End of the test for e-TeX.
%% Old format and not e-TeX,
%% otherwise the last \endinput would have been applied.


%    \end{macrocode}
%
% If we ever come to this place, \textquotedblleft everything\textquotedblright{} %
% failed and we need to do things the old fashioned way,
% which severely limits the maximum number of additionally available floats.
%
%    \begin{macrocode}
\PackageWarning{morefloats}{%
  e-TeX is not available here\MessageBreak%
  but it is available in almost all\MessageBreak%
  recent TeX distributions.\MessageBreak%
  Maybe consider updating to one of those%
  \@gobble%
 }%

%    \end{macrocode}
%
% \newpage
%
% \begin{landscape}
%
% |Float| uses |insert|, and each |insert| use a group of |count|, |dimen|, |skip|,
% and |box| each. When there are not enough available, no |\newinsert| can be created.
%
%    \begin{macrocode}
%% Code similar to the one from Heiko Oberdiek,
%% http://permalink.gmane.org/gmane.comp.tex.latex.latex3/2159
                           \@tempcnta=\the\count10 \relax \def\maxfloats@vln{count}    %
\ifnum \count11>\@tempcnta \@tempcnta=\the\count11 \relax \def\maxfloats@vln{dimen} \fi%
\ifnum \count12>\@tempcnta \@tempcnta=\the\count12 \relax \def\maxfloats@vln{skip}  \fi%
\ifnum \count14>\@tempcnta \@tempcnta=\the\count14 \relax \def\maxfloats@vln{box}   \fi%
%% end similar
\@tempcntb=234\relax%
\advance\@tempcntb by -\@tempcnta\relax%
\@tempcnta=\@tempcntb\relax%
\ifnum\morefloats@mx>\@tempcntb\relax%
  \PackageError{morefloats}{Too many floats requested}{%
    Maximum number of possible floats asked for: \morefloats@mx .\MessageBreak%
    There are only \the\@tempcnta\space \maxfloats@vln\space left,\MessageBreak%
    therefore only \the\@tempcntb\space floats will be possible.\MessageBreak%
    Load the morefloats package earlier and/or\MessageBreak%
    reduce the number of used \maxfloats@vln\space registers\MessageBreak%
    to have more floats available!\MessageBreak%
   }%
  \xdef\morefloats@mx{\the\@tempcntb}%
\fi%

%    \end{macrocode}
%
% The task at hand is to increase \LaTeX{}'s default limit of $18$~unprocessed
% floats in memory at once to |maxfloats|.
% An examination of \texttt{latex.tex} reveals that this is accomplished
% by allocating~(!) an insert register for each unprocessed float. A~quick
% check of (the obsolete, now \texttt{ltplain}, update to \LaTeX2e{}!)
% \texttt{lplain.lis} reveals that there is room, in fact, for up to
% $256$ unprocessed floats, but \TeX{}'s main memory could be exhausted
% well before that happened.\\
%
% \LaTeX2e{} uses a |\dimen| for each |\newinsert|, and the number of |\dimen|s
% is also restricted. Therefore only use the number of floats you need!
% To check the number of used registers, you could use the \xpackage{regstats}
% and/or \xpackage{regcount} packages (see subsection~\ref{ss:Downloads}).
%
% \bigskip
%
% \DescribeMacro{Allocating insert registers}
% \DescribeMacro{@freelist}
% \DescribeMacro{@elt}
% \DescribeMacro{newinsert}
% First we allocate the additional insert registers needed.\\
%
% That accomplished, the next step is to define the macro |\@freelist|,
% which is merely a~list of the box registers each preceded by |\@elt|.
% This approach allows processing of the list to be done far more efficiently.
% A similar approach is used by \textsc{Mittelbach \& Sch\"{o}pf}'s \texttt{doc.sty} to
% keep track of control sequences, which should not be indexed.\\
% First for the 18 default \LaTeX{} boxes.\\
% \noindent |\ifnum maxfloats <= 18|, \LaTeX{} already allocated the insert registers. |\fi|\\
%
%    \begin{macrocode}
\global\long\def\@freelist{\@elt\bx@A\@elt\bx@B\@elt\bx@C\@elt\bx@D\@elt\bx@E\@elt\bx@F\@elt\bx@G\@elt\bx@H\@elt%
\bx@I\@elt\bx@J\@elt\bx@K\@elt\bx@L\@elt\bx@M\@elt\bx@N\@elt\bx@O\@elt\bx@P\@elt\bx@Q\@elt\bx@R}

%    \end{macrocode}
%
% Now we need to add |\@elt\bx@...| depending on the number of |morefloats| wanted:\\
% (\textsc{Karl Berry} helped with two out of three |\expandafter|s, thanks!)
%
% \medskip
%
%    \begin{macrocode}
\ifnum \morefloats@mx> 18 \newinsert\bx@S  \expandafter\gdef\expandafter\@freelist\expandafter{\@freelist \@elt\bx@S}
\ifnum \morefloats@mx> 19 \newinsert\bx@T  \expandafter\gdef\expandafter\@freelist\expandafter{\@freelist \@elt\bx@T}
\ifnum \morefloats@mx> 20 \newinsert\bx@U  \expandafter\gdef\expandafter\@freelist\expandafter{\@freelist \@elt\bx@U}
\ifnum \morefloats@mx> 21 \newinsert\bx@V  \expandafter\gdef\expandafter\@freelist\expandafter{\@freelist \@elt\bx@V}
\ifnum \morefloats@mx> 22 \newinsert\bx@W  \expandafter\gdef\expandafter\@freelist\expandafter{\@freelist \@elt\bx@W}
\ifnum \morefloats@mx> 23 \newinsert\bx@X  \expandafter\gdef\expandafter\@freelist\expandafter{\@freelist \@elt\bx@X}
\ifnum \morefloats@mx> 24 \newinsert\bx@Y  \expandafter\gdef\expandafter\@freelist\expandafter{\@freelist \@elt\bx@Y}
\ifnum \morefloats@mx> 25 \newinsert\bx@Z  \expandafter\gdef\expandafter\@freelist\expandafter{\@freelist \@elt\bx@Z}
\ifnum \morefloats@mx> 26 \newinsert\bx@AA \expandafter\gdef\expandafter\@freelist\expandafter{\@freelist \@elt\bx@AA}
\ifnum \morefloats@mx> 27 \newinsert\bx@AB \expandafter\gdef\expandafter\@freelist\expandafter{\@freelist \@elt\bx@AB}
\ifnum \morefloats@mx> 28 \newinsert\bx@AC \expandafter\gdef\expandafter\@freelist\expandafter{\@freelist \@elt\bx@AC}
\ifnum \morefloats@mx> 29 \newinsert\bx@AD \expandafter\gdef\expandafter\@freelist\expandafter{\@freelist \@elt\bx@AD}
\ifnum \morefloats@mx> 30 \newinsert\bx@AE \expandafter\gdef\expandafter\@freelist\expandafter{\@freelist \@elt\bx@AE}
\ifnum \morefloats@mx> 31 \newinsert\bx@AF \expandafter\gdef\expandafter\@freelist\expandafter{\@freelist \@elt\bx@AF}
\ifnum \morefloats@mx> 32 \newinsert\bx@AG \expandafter\gdef\expandafter\@freelist\expandafter{\@freelist \@elt\bx@AG}
\ifnum \morefloats@mx> 33 \newinsert\bx@AH \expandafter\gdef\expandafter\@freelist\expandafter{\@freelist \@elt\bx@AH}
\ifnum \morefloats@mx> 34 \newinsert\bx@AI \expandafter\gdef\expandafter\@freelist\expandafter{\@freelist \@elt\bx@AI}
\ifnum \morefloats@mx> 35 \newinsert\bx@AJ \expandafter\gdef\expandafter\@freelist\expandafter{\@freelist \@elt\bx@AJ}
\ifnum \morefloats@mx> 36 \newinsert\bx@AK \expandafter\gdef\expandafter\@freelist\expandafter{\@freelist \@elt\bx@AK}
\ifnum \morefloats@mx> 37 \newinsert\bx@AL \expandafter\gdef\expandafter\@freelist\expandafter{\@freelist \@elt\bx@AL}
\ifnum \morefloats@mx> 38 \newinsert\bx@AM \expandafter\gdef\expandafter\@freelist\expandafter{\@freelist \@elt\bx@AM}
\ifnum \morefloats@mx> 39 \newinsert\bx@AN \expandafter\gdef\expandafter\@freelist\expandafter{\@freelist \@elt\bx@AN}
\ifnum \morefloats@mx> 40 \newinsert\bx@AO \expandafter\gdef\expandafter\@freelist\expandafter{\@freelist \@elt\bx@AO}
\ifnum \morefloats@mx> 41 \newinsert\bx@AP \expandafter\gdef\expandafter\@freelist\expandafter{\@freelist \@elt\bx@AP}
\ifnum \morefloats@mx> 42 \newinsert\bx@AQ \expandafter\gdef\expandafter\@freelist\expandafter{\@freelist \@elt\bx@AQ}
\ifnum \morefloats@mx> 43 \newinsert\bx@AR \expandafter\gdef\expandafter\@freelist\expandafter{\@freelist \@elt\bx@AR}
\ifnum \morefloats@mx> 44 \newinsert\bx@AS \expandafter\gdef\expandafter\@freelist\expandafter{\@freelist \@elt\bx@AS}
\ifnum \morefloats@mx> 45 \newinsert\bx@AT \expandafter\gdef\expandafter\@freelist\expandafter{\@freelist \@elt\bx@AT}
\ifnum \morefloats@mx> 46 \newinsert\bx@AU \expandafter\gdef\expandafter\@freelist\expandafter{\@freelist \@elt\bx@AU}
\ifnum \morefloats@mx> 47 \newinsert\bx@AV \expandafter\gdef\expandafter\@freelist\expandafter{\@freelist \@elt\bx@AV}
\ifnum \morefloats@mx> 48 \newinsert\bx@AW \expandafter\gdef\expandafter\@freelist\expandafter{\@freelist \@elt\bx@AW}
\ifnum \morefloats@mx> 49 \newinsert\bx@AX \expandafter\gdef\expandafter\@freelist\expandafter{\@freelist \@elt\bx@AX}
\ifnum \morefloats@mx> 50 \newinsert\bx@AY \expandafter\gdef\expandafter\@freelist\expandafter{\@freelist \@elt\bx@AY}
\ifnum \morefloats@mx> 51 \newinsert\bx@AZ \expandafter\gdef\expandafter\@freelist\expandafter{\@freelist \@elt\bx@AZ}
\ifnum \morefloats@mx> 52 \newinsert\bx@BA \expandafter\gdef\expandafter\@freelist\expandafter{\@freelist \@elt\bx@BA}
\ifnum \morefloats@mx> 53 \newinsert\bx@BB \expandafter\gdef\expandafter\@freelist\expandafter{\@freelist \@elt\bx@BB}
\ifnum \morefloats@mx> 54 \newinsert\bx@BC \expandafter\gdef\expandafter\@freelist\expandafter{\@freelist \@elt\bx@BC}
\ifnum \morefloats@mx> 55 \newinsert\bx@BD \expandafter\gdef\expandafter\@freelist\expandafter{\@freelist \@elt\bx@BD}
\ifnum \morefloats@mx> 56 \newinsert\bx@BE \expandafter\gdef\expandafter\@freelist\expandafter{\@freelist \@elt\bx@BE}
\ifnum \morefloats@mx> 57 \newinsert\bx@BF \expandafter\gdef\expandafter\@freelist\expandafter{\@freelist \@elt\bx@BF}
\ifnum \morefloats@mx> 58 \newinsert\bx@BG \expandafter\gdef\expandafter\@freelist\expandafter{\@freelist \@elt\bx@BG}
\ifnum \morefloats@mx> 59 \newinsert\bx@BH \expandafter\gdef\expandafter\@freelist\expandafter{\@freelist \@elt\bx@BH}
\ifnum \morefloats@mx> 60 \newinsert\bx@BI \expandafter\gdef\expandafter\@freelist\expandafter{\@freelist \@elt\bx@BI}
\ifnum \morefloats@mx> 61 \newinsert\bx@BJ \expandafter\gdef\expandafter\@freelist\expandafter{\@freelist \@elt\bx@BJ}
\ifnum \morefloats@mx> 62 \newinsert\bx@BK \expandafter\gdef\expandafter\@freelist\expandafter{\@freelist \@elt\bx@BK}
\ifnum \morefloats@mx> 63 \newinsert\bx@BL \expandafter\gdef\expandafter\@freelist\expandafter{\@freelist \@elt\bx@BL}
\ifnum \morefloats@mx> 64 \newinsert\bx@BM \expandafter\gdef\expandafter\@freelist\expandafter{\@freelist \@elt\bx@BM}
\ifnum \morefloats@mx> 65 \newinsert\bx@BN \expandafter\gdef\expandafter\@freelist\expandafter{\@freelist \@elt\bx@BN}
\ifnum \morefloats@mx> 66 \newinsert\bx@BO \expandafter\gdef\expandafter\@freelist\expandafter{\@freelist \@elt\bx@BO}
\ifnum \morefloats@mx> 67 \newinsert\bx@BP \expandafter\gdef\expandafter\@freelist\expandafter{\@freelist \@elt\bx@BP}
\ifnum \morefloats@mx> 68 \newinsert\bx@BQ \expandafter\gdef\expandafter\@freelist\expandafter{\@freelist \@elt\bx@BQ}
\ifnum \morefloats@mx> 69 \newinsert\bx@BR \expandafter\gdef\expandafter\@freelist\expandafter{\@freelist \@elt\bx@BR}
\ifnum \morefloats@mx> 70 \newinsert\bx@BS \expandafter\gdef\expandafter\@freelist\expandafter{\@freelist \@elt\bx@BS}
\ifnum \morefloats@mx> 71 \newinsert\bx@BT \expandafter\gdef\expandafter\@freelist\expandafter{\@freelist \@elt\bx@BT}
\ifnum \morefloats@mx> 72 \newinsert\bx@BU \expandafter\gdef\expandafter\@freelist\expandafter{\@freelist \@elt\bx@BU}
\ifnum \morefloats@mx> 73 \newinsert\bx@BV \expandafter\gdef\expandafter\@freelist\expandafter{\@freelist \@elt\bx@BV}
\ifnum \morefloats@mx> 74 \newinsert\bx@BW \expandafter\gdef\expandafter\@freelist\expandafter{\@freelist \@elt\bx@BW}
\ifnum \morefloats@mx> 75 \newinsert\bx@BX \expandafter\gdef\expandafter\@freelist\expandafter{\@freelist \@elt\bx@BX}
\ifnum \morefloats@mx> 76 \newinsert\bx@BY \expandafter\gdef\expandafter\@freelist\expandafter{\@freelist \@elt\bx@BY}
\ifnum \morefloats@mx> 77 \newinsert\bx@BZ \expandafter\gdef\expandafter\@freelist\expandafter{\@freelist \@elt\bx@BZ}
\ifnum \morefloats@mx> 78 \newinsert\bx@CA \expandafter\gdef\expandafter\@freelist\expandafter{\@freelist \@elt\bx@CA}
\ifnum \morefloats@mx> 79 \newinsert\bx@CB \expandafter\gdef\expandafter\@freelist\expandafter{\@freelist \@elt\bx@CB}
\ifnum \morefloats@mx> 80 \newinsert\bx@CC \expandafter\gdef\expandafter\@freelist\expandafter{\@freelist \@elt\bx@CC}
\ifnum \morefloats@mx> 81 \newinsert\bx@CD \expandafter\gdef\expandafter\@freelist\expandafter{\@freelist \@elt\bx@CD}
\ifnum \morefloats@mx> 82 \newinsert\bx@CE \expandafter\gdef\expandafter\@freelist\expandafter{\@freelist \@elt\bx@CE}
\ifnum \morefloats@mx> 83 \newinsert\bx@CF \expandafter\gdef\expandafter\@freelist\expandafter{\@freelist \@elt\bx@CF}
\ifnum \morefloats@mx> 84 \newinsert\bx@CG \expandafter\gdef\expandafter\@freelist\expandafter{\@freelist \@elt\bx@CG}
\ifnum \morefloats@mx> 85 \newinsert\bx@CH \expandafter\gdef\expandafter\@freelist\expandafter{\@freelist \@elt\bx@CH}
\ifnum \morefloats@mx> 86 \newinsert\bx@CI \expandafter\gdef\expandafter\@freelist\expandafter{\@freelist \@elt\bx@CI}
\ifnum \morefloats@mx> 87 \newinsert\bx@CJ \expandafter\gdef\expandafter\@freelist\expandafter{\@freelist \@elt\bx@CJ}
\ifnum \morefloats@mx> 88 \newinsert\bx@CK \expandafter\gdef\expandafter\@freelist\expandafter{\@freelist \@elt\bx@CK}
\ifnum \morefloats@mx> 89 \newinsert\bx@CL \expandafter\gdef\expandafter\@freelist\expandafter{\@freelist \@elt\bx@CL}
\ifnum \morefloats@mx> 90 \newinsert\bx@CM \expandafter\gdef\expandafter\@freelist\expandafter{\@freelist \@elt\bx@CM}
\ifnum \morefloats@mx> 91 \newinsert\bx@CN \expandafter\gdef\expandafter\@freelist\expandafter{\@freelist \@elt\bx@CN}
\ifnum \morefloats@mx> 92 \newinsert\bx@CO \expandafter\gdef\expandafter\@freelist\expandafter{\@freelist \@elt\bx@CO}
\ifnum \morefloats@mx> 93 \newinsert\bx@CP \expandafter\gdef\expandafter\@freelist\expandafter{\@freelist \@elt\bx@CP}
\ifnum \morefloats@mx> 94 \newinsert\bx@CQ \expandafter\gdef\expandafter\@freelist\expandafter{\@freelist \@elt\bx@CQ}
\ifnum \morefloats@mx> 95 \newinsert\bx@CR \expandafter\gdef\expandafter\@freelist\expandafter{\@freelist \@elt\bx@CR}
\ifnum \morefloats@mx> 96 \newinsert\bx@CS \expandafter\gdef\expandafter\@freelist\expandafter{\@freelist \@elt\bx@CS}
\ifnum \morefloats@mx> 97 \newinsert\bx@CT \expandafter\gdef\expandafter\@freelist\expandafter{\@freelist \@elt\bx@CT}
\ifnum \morefloats@mx> 98 \newinsert\bx@CU \expandafter\gdef\expandafter\@freelist\expandafter{\@freelist \@elt\bx@CU}
\ifnum \morefloats@mx> 99 \newinsert\bx@CV \expandafter\gdef\expandafter\@freelist\expandafter{\@freelist \@elt\bx@CV}
\ifnum \morefloats@mx>100 \newinsert\bx@CW \expandafter\gdef\expandafter\@freelist\expandafter{\@freelist \@elt\bx@CW}
\ifnum \morefloats@mx>101 \newinsert\bx@CX \expandafter\gdef\expandafter\@freelist\expandafter{\@freelist \@elt\bx@CX}
\ifnum \morefloats@mx>102 \newinsert\bx@CY \expandafter\gdef\expandafter\@freelist\expandafter{\@freelist \@elt\bx@CY}
\ifnum \morefloats@mx>103 \newinsert\bx@CZ \expandafter\gdef\expandafter\@freelist\expandafter{\@freelist \@elt\bx@CZ}
\ifnum \morefloats@mx>104 \newinsert\bx@DA \expandafter\gdef\expandafter\@freelist\expandafter{\@freelist \@elt\bx@DA}
\ifnum \morefloats@mx>105 \newinsert\bx@DB \expandafter\gdef\expandafter\@freelist\expandafter{\@freelist \@elt\bx@DB}
\ifnum \morefloats@mx>106 \newinsert\bx@DC \expandafter\gdef\expandafter\@freelist\expandafter{\@freelist \@elt\bx@DC}
\ifnum \morefloats@mx>107 \newinsert\bx@DD \expandafter\gdef\expandafter\@freelist\expandafter{\@freelist \@elt\bx@DD}
\ifnum \morefloats@mx>108 \newinsert\bx@DE \expandafter\gdef\expandafter\@freelist\expandafter{\@freelist \@elt\bx@DE}
\ifnum \morefloats@mx>109 \newinsert\bx@DF \expandafter\gdef\expandafter\@freelist\expandafter{\@freelist \@elt\bx@DF}
\ifnum \morefloats@mx>110 \newinsert\bx@DG \expandafter\gdef\expandafter\@freelist\expandafter{\@freelist \@elt\bx@DG}
\ifnum \morefloats@mx>111 \newinsert\bx@DH \expandafter\gdef\expandafter\@freelist\expandafter{\@freelist \@elt\bx@DH}
\ifnum \morefloats@mx>112 \newinsert\bx@DI \expandafter\gdef\expandafter\@freelist\expandafter{\@freelist \@elt\bx@DI}
\ifnum \morefloats@mx>113 \newinsert\bx@DJ \expandafter\gdef\expandafter\@freelist\expandafter{\@freelist \@elt\bx@DJ}
\ifnum \morefloats@mx>114 \newinsert\bx@DK \expandafter\gdef\expandafter\@freelist\expandafter{\@freelist \@elt\bx@DK}
\ifnum \morefloats@mx>115 \newinsert\bx@DL \expandafter\gdef\expandafter\@freelist\expandafter{\@freelist \@elt\bx@DL}
\ifnum \morefloats@mx>116 \newinsert\bx@DM \expandafter\gdef\expandafter\@freelist\expandafter{\@freelist \@elt\bx@DM}
\ifnum \morefloats@mx>117 \newinsert\bx@DN \expandafter\gdef\expandafter\@freelist\expandafter{\@freelist \@elt\bx@DN}
\ifnum \morefloats@mx>118 \newinsert\bx@DO \expandafter\gdef\expandafter\@freelist\expandafter{\@freelist \@elt\bx@DO}
\ifnum \morefloats@mx>119 \newinsert\bx@DP \expandafter\gdef\expandafter\@freelist\expandafter{\@freelist \@elt\bx@DP}
\ifnum \morefloats@mx>120 \newinsert\bx@DQ \expandafter\gdef\expandafter\@freelist\expandafter{\@freelist \@elt\bx@DQ}
\ifnum \morefloats@mx>121 \newinsert\bx@DR \expandafter\gdef\expandafter\@freelist\expandafter{\@freelist \@elt\bx@DR}
\ifnum \morefloats@mx>122 \newinsert\bx@DS \expandafter\gdef\expandafter\@freelist\expandafter{\@freelist \@elt\bx@DS}
\ifnum \morefloats@mx>123 \newinsert\bx@DT \expandafter\gdef\expandafter\@freelist\expandafter{\@freelist \@elt\bx@DT}
\ifnum \morefloats@mx>124 \newinsert\bx@DU \expandafter\gdef\expandafter\@freelist\expandafter{\@freelist \@elt\bx@DU}
\ifnum \morefloats@mx>125 \newinsert\bx@DV \expandafter\gdef\expandafter\@freelist\expandafter{\@freelist \@elt\bx@DV}
\ifnum \morefloats@mx>126 \newinsert\bx@DW \expandafter\gdef\expandafter\@freelist\expandafter{\@freelist \@elt\bx@DW}
\ifnum \morefloats@mx>127 \newinsert\bx@DX \expandafter\gdef\expandafter\@freelist\expandafter{\@freelist \@elt\bx@DX}
\ifnum \morefloats@mx>128 \newinsert\bx@DY \expandafter\gdef\expandafter\@freelist\expandafter{\@freelist \@elt\bx@DY}
\ifnum \morefloats@mx>129 \newinsert\bx@DZ \expandafter\gdef\expandafter\@freelist\expandafter{\@freelist \@elt\bx@DZ}
\ifnum \morefloats@mx>130 \newinsert\bx@EA \expandafter\gdef\expandafter\@freelist\expandafter{\@freelist \@elt\bx@EA}
\ifnum \morefloats@mx>131 \newinsert\bx@EB \expandafter\gdef\expandafter\@freelist\expandafter{\@freelist \@elt\bx@EB}
\ifnum \morefloats@mx>132 \newinsert\bx@EC \expandafter\gdef\expandafter\@freelist\expandafter{\@freelist \@elt\bx@EC}
\ifnum \morefloats@mx>133 \newinsert\bx@ED \expandafter\gdef\expandafter\@freelist\expandafter{\@freelist \@elt\bx@ED}
\ifnum \morefloats@mx>134 \newinsert\bx@EE \expandafter\gdef\expandafter\@freelist\expandafter{\@freelist \@elt\bx@EE}
\ifnum \morefloats@mx>135 \newinsert\bx@EF \expandafter\gdef\expandafter\@freelist\expandafter{\@freelist \@elt\bx@EF}
\ifnum \morefloats@mx>136 \newinsert\bx@EG \expandafter\gdef\expandafter\@freelist\expandafter{\@freelist \@elt\bx@EG}
\ifnum \morefloats@mx>137 \newinsert\bx@EH \expandafter\gdef\expandafter\@freelist\expandafter{\@freelist \@elt\bx@EH}
\ifnum \morefloats@mx>138 \newinsert\bx@EI \expandafter\gdef\expandafter\@freelist\expandafter{\@freelist \@elt\bx@EI}
\ifnum \morefloats@mx>139 \newinsert\bx@EJ \expandafter\gdef\expandafter\@freelist\expandafter{\@freelist \@elt\bx@EJ}
\ifnum \morefloats@mx>140 \newinsert\bx@EK \expandafter\gdef\expandafter\@freelist\expandafter{\@freelist \@elt\bx@EK}
\ifnum \morefloats@mx>141 \newinsert\bx@EL \expandafter\gdef\expandafter\@freelist\expandafter{\@freelist \@elt\bx@EL}
\ifnum \morefloats@mx>142 \newinsert\bx@EM \expandafter\gdef\expandafter\@freelist\expandafter{\@freelist \@elt\bx@EM}
\ifnum \morefloats@mx>143 \newinsert\bx@EN \expandafter\gdef\expandafter\@freelist\expandafter{\@freelist \@elt\bx@EN}
\ifnum \morefloats@mx>144 \newinsert\bx@EO \expandafter\gdef\expandafter\@freelist\expandafter{\@freelist \@elt\bx@EO}
\ifnum \morefloats@mx>145 \newinsert\bx@EP \expandafter\gdef\expandafter\@freelist\expandafter{\@freelist \@elt\bx@EP}
\ifnum \morefloats@mx>146 \newinsert\bx@EQ \expandafter\gdef\expandafter\@freelist\expandafter{\@freelist \@elt\bx@EQ}
\ifnum \morefloats@mx>147 \newinsert\bx@ER \expandafter\gdef\expandafter\@freelist\expandafter{\@freelist \@elt\bx@ER}
\ifnum \morefloats@mx>148 \newinsert\bx@ES \expandafter\gdef\expandafter\@freelist\expandafter{\@freelist \@elt\bx@ES}
\ifnum \morefloats@mx>149 \newinsert\bx@ET \expandafter\gdef\expandafter\@freelist\expandafter{\@freelist \@elt\bx@ET}
\ifnum \morefloats@mx>150 \newinsert\bx@EU \expandafter\gdef\expandafter\@freelist\expandafter{\@freelist \@elt\bx@EU}
\ifnum \morefloats@mx>151 \newinsert\bx@EV \expandafter\gdef\expandafter\@freelist\expandafter{\@freelist \@elt\bx@EV}
\ifnum \morefloats@mx>152 \newinsert\bx@EW \expandafter\gdef\expandafter\@freelist\expandafter{\@freelist \@elt\bx@EW}
\ifnum \morefloats@mx>153 \newinsert\bx@EX \expandafter\gdef\expandafter\@freelist\expandafter{\@freelist \@elt\bx@EX}
\ifnum \morefloats@mx>154 \newinsert\bx@EY \expandafter\gdef\expandafter\@freelist\expandafter{\@freelist \@elt\bx@EY}
\ifnum \morefloats@mx>155 \newinsert\bx@EZ \expandafter\gdef\expandafter\@freelist\expandafter{\@freelist \@elt\bx@EZ}
\ifnum \morefloats@mx>156 \newinsert\bx@FA \expandafter\gdef\expandafter\@freelist\expandafter{\@freelist \@elt\bx@FA}
\ifnum \morefloats@mx>157 \newinsert\bx@FB \expandafter\gdef\expandafter\@freelist\expandafter{\@freelist \@elt\bx@FB}
\ifnum \morefloats@mx>158 \newinsert\bx@FC \expandafter\gdef\expandafter\@freelist\expandafter{\@freelist \@elt\bx@FC}
\ifnum \morefloats@mx>159 \newinsert\bx@FD \expandafter\gdef\expandafter\@freelist\expandafter{\@freelist \@elt\bx@FD}
\ifnum \morefloats@mx>160 \newinsert\bx@FE \expandafter\gdef\expandafter\@freelist\expandafter{\@freelist \@elt\bx@FE}
\ifnum \morefloats@mx>161 \newinsert\bx@FF \expandafter\gdef\expandafter\@freelist\expandafter{\@freelist \@elt\bx@FF}
\ifnum \morefloats@mx>162 \newinsert\bx@FG \expandafter\gdef\expandafter\@freelist\expandafter{\@freelist \@elt\bx@FG}
\ifnum \morefloats@mx>163 \newinsert\bx@FH \expandafter\gdef\expandafter\@freelist\expandafter{\@freelist \@elt\bx@FH}
\ifnum \morefloats@mx>164 \newinsert\bx@FI \expandafter\gdef\expandafter\@freelist\expandafter{\@freelist \@elt\bx@FI}
\ifnum \morefloats@mx>165 \newinsert\bx@FJ \expandafter\gdef\expandafter\@freelist\expandafter{\@freelist \@elt\bx@FJ}
\ifnum \morefloats@mx>166 \newinsert\bx@FK \expandafter\gdef\expandafter\@freelist\expandafter{\@freelist \@elt\bx@FK}
\ifnum \morefloats@mx>167 \newinsert\bx@FL \expandafter\gdef\expandafter\@freelist\expandafter{\@freelist \@elt\bx@FL}
\ifnum \morefloats@mx>168 \newinsert\bx@FM \expandafter\gdef\expandafter\@freelist\expandafter{\@freelist \@elt\bx@FM}
\ifnum \morefloats@mx>169 \newinsert\bx@FN \expandafter\gdef\expandafter\@freelist\expandafter{\@freelist \@elt\bx@FN}
\ifnum \morefloats@mx>170 \newinsert\bx@FO \expandafter\gdef\expandafter\@freelist\expandafter{\@freelist \@elt\bx@FO}
\ifnum \morefloats@mx>171 \newinsert\bx@FP \expandafter\gdef\expandafter\@freelist\expandafter{\@freelist \@elt\bx@FP}
\ifnum \morefloats@mx>172 \newinsert\bx@FQ \expandafter\gdef\expandafter\@freelist\expandafter{\@freelist \@elt\bx@FQ}
\ifnum \morefloats@mx>173 \newinsert\bx@FR \expandafter\gdef\expandafter\@freelist\expandafter{\@freelist \@elt\bx@FR}
\ifnum \morefloats@mx>174 \newinsert\bx@FS \expandafter\gdef\expandafter\@freelist\expandafter{\@freelist \@elt\bx@FS}
\ifnum \morefloats@mx>175 \newinsert\bx@FT \expandafter\gdef\expandafter\@freelist\expandafter{\@freelist \@elt\bx@FT}
\ifnum \morefloats@mx>176 \newinsert\bx@FU \expandafter\gdef\expandafter\@freelist\expandafter{\@freelist \@elt\bx@FU}
\ifnum \morefloats@mx>177 \newinsert\bx@FV \expandafter\gdef\expandafter\@freelist\expandafter{\@freelist \@elt\bx@FV}
\ifnum \morefloats@mx>178 \newinsert\bx@FW \expandafter\gdef\expandafter\@freelist\expandafter{\@freelist \@elt\bx@FW}
\ifnum \morefloats@mx>179 \newinsert\bx@FX \expandafter\gdef\expandafter\@freelist\expandafter{\@freelist \@elt\bx@FX}
\ifnum \morefloats@mx>180 \newinsert\bx@FY \expandafter\gdef\expandafter\@freelist\expandafter{\@freelist \@elt\bx@FY}
\ifnum \morefloats@mx>181 \newinsert\bx@FZ \expandafter\gdef\expandafter\@freelist\expandafter{\@freelist \@elt\bx@FZ}
\ifnum \morefloats@mx>182 \newinsert\bx@GA \expandafter\gdef\expandafter\@freelist\expandafter{\@freelist \@elt\bx@GA}
\ifnum \morefloats@mx>183 \newinsert\bx@GB \expandafter\gdef\expandafter\@freelist\expandafter{\@freelist \@elt\bx@GB}
\ifnum \morefloats@mx>184 \newinsert\bx@GC \expandafter\gdef\expandafter\@freelist\expandafter{\@freelist \@elt\bx@GC}
\ifnum \morefloats@mx>185 \newinsert\bx@GD \expandafter\gdef\expandafter\@freelist\expandafter{\@freelist \@elt\bx@GD}
\ifnum \morefloats@mx>186 \newinsert\bx@GE \expandafter\gdef\expandafter\@freelist\expandafter{\@freelist \@elt\bx@GE}
\ifnum \morefloats@mx>187 \newinsert\bx@GF \expandafter\gdef\expandafter\@freelist\expandafter{\@freelist \@elt\bx@GF}
\ifnum \morefloats@mx>188 \newinsert\bx@GG \expandafter\gdef\expandafter\@freelist\expandafter{\@freelist \@elt\bx@GG}
\ifnum \morefloats@mx>189 \newinsert\bx@GH \expandafter\gdef\expandafter\@freelist\expandafter{\@freelist \@elt\bx@GH}
\ifnum \morefloats@mx>190 \newinsert\bx@GI \expandafter\gdef\expandafter\@freelist\expandafter{\@freelist \@elt\bx@GI}
\ifnum \morefloats@mx>191 \newinsert\bx@GJ \expandafter\gdef\expandafter\@freelist\expandafter{\@freelist \@elt\bx@GJ}
\ifnum \morefloats@mx>192 \newinsert\bx@GK \expandafter\gdef\expandafter\@freelist\expandafter{\@freelist \@elt\bx@GK}
\ifnum \morefloats@mx>193 \newinsert\bx@GL \expandafter\gdef\expandafter\@freelist\expandafter{\@freelist \@elt\bx@GL}
\ifnum \morefloats@mx>194 \newinsert\bx@GM \expandafter\gdef\expandafter\@freelist\expandafter{\@freelist \@elt\bx@GM}
\ifnum \morefloats@mx>195 \newinsert\bx@GN \expandafter\gdef\expandafter\@freelist\expandafter{\@freelist \@elt\bx@GN}
\ifnum \morefloats@mx>196 \newinsert\bx@GO \expandafter\gdef\expandafter\@freelist\expandafter{\@freelist \@elt\bx@GO}
\ifnum \morefloats@mx>197 \newinsert\bx@GP \expandafter\gdef\expandafter\@freelist\expandafter{\@freelist \@elt\bx@GP}
\ifnum \morefloats@mx>198 \newinsert\bx@GQ \expandafter\gdef\expandafter\@freelist\expandafter{\@freelist \@elt\bx@GQ}
\ifnum \morefloats@mx>199 \newinsert\bx@GR \expandafter\gdef\expandafter\@freelist\expandafter{\@freelist \@elt\bx@GR}
\ifnum \morefloats@mx>200 \newinsert\bx@GS \expandafter\gdef\expandafter\@freelist\expandafter{\@freelist \@elt\bx@GS}
\ifnum \morefloats@mx>201 \newinsert\bx@GT \expandafter\gdef\expandafter\@freelist\expandafter{\@freelist \@elt\bx@GT}
\ifnum \morefloats@mx>202 \newinsert\bx@GU \expandafter\gdef\expandafter\@freelist\expandafter{\@freelist \@elt\bx@GU}
\ifnum \morefloats@mx>203 \newinsert\bx@GV \expandafter\gdef\expandafter\@freelist\expandafter{\@freelist \@elt\bx@GV}
\ifnum \morefloats@mx>204 \newinsert\bx@GW \expandafter\gdef\expandafter\@freelist\expandafter{\@freelist \@elt\bx@GW}
\ifnum \morefloats@mx>205 \newinsert\bx@GX \expandafter\gdef\expandafter\@freelist\expandafter{\@freelist \@elt\bx@GX}
\ifnum \morefloats@mx>206 \newinsert\bx@GY \expandafter\gdef\expandafter\@freelist\expandafter{\@freelist \@elt\bx@GY}
\ifnum \morefloats@mx>207 \newinsert\bx@GZ \expandafter\gdef\expandafter\@freelist\expandafter{\@freelist \@elt\bx@GZ}
\ifnum \morefloats@mx>208 \newinsert\bx@HA \expandafter\gdef\expandafter\@freelist\expandafter{\@freelist \@elt\bx@HA}
\ifnum \morefloats@mx>209 \newinsert\bx@HB \expandafter\gdef\expandafter\@freelist\expandafter{\@freelist \@elt\bx@HB}
\ifnum \morefloats@mx>210 \newinsert\bx@HC \expandafter\gdef\expandafter\@freelist\expandafter{\@freelist \@elt\bx@HC}
\ifnum \morefloats@mx>211 \newinsert\bx@HD \expandafter\gdef\expandafter\@freelist\expandafter{\@freelist \@elt\bx@HD}
\ifnum \morefloats@mx>212 \newinsert\bx@HE \expandafter\gdef\expandafter\@freelist\expandafter{\@freelist \@elt\bx@HE}
\ifnum \morefloats@mx>213 \newinsert\bx@HF \expandafter\gdef\expandafter\@freelist\expandafter{\@freelist \@elt\bx@HF}
\ifnum \morefloats@mx>214 \newinsert\bx@HG \expandafter\gdef\expandafter\@freelist\expandafter{\@freelist \@elt\bx@HG}
\ifnum \morefloats@mx>215 \newinsert\bx@HH \expandafter\gdef\expandafter\@freelist\expandafter{\@freelist \@elt\bx@HH}
\ifnum \morefloats@mx>216 \newinsert\bx@HI \expandafter\gdef\expandafter\@freelist\expandafter{\@freelist \@elt\bx@HI}
\ifnum \morefloats@mx>217 \newinsert\bx@HJ \expandafter\gdef\expandafter\@freelist\expandafter{\@freelist \@elt\bx@HJ}
\ifnum \morefloats@mx>218 \newinsert\bx@HK \expandafter\gdef\expandafter\@freelist\expandafter{\@freelist \@elt\bx@HK}
\ifnum \morefloats@mx>219 \newinsert\bx@HL \expandafter\gdef\expandafter\@freelist\expandafter{\@freelist \@elt\bx@HL}
\ifnum \morefloats@mx>220 \newinsert\bx@HM \expandafter\gdef\expandafter\@freelist\expandafter{\@freelist \@elt\bx@HM}
\ifnum \morefloats@mx>221 \newinsert\bx@HN \expandafter\gdef\expandafter\@freelist\expandafter{\@freelist \@elt\bx@HN}
\ifnum \morefloats@mx>222 \newinsert\bx@HO \expandafter\gdef\expandafter\@freelist\expandafter{\@freelist \@elt\bx@HO}
\ifnum \morefloats@mx>223 \newinsert\bx@HP \expandafter\gdef\expandafter\@freelist\expandafter{\@freelist \@elt\bx@HP}
\ifnum \morefloats@mx>224 \newinsert\bx@HQ \expandafter\gdef\expandafter\@freelist\expandafter{\@freelist \@elt\bx@HQ}
\ifnum \morefloats@mx>225 \newinsert\bx@HR \expandafter\gdef\expandafter\@freelist\expandafter{\@freelist \@elt\bx@HR}
\ifnum \morefloats@mx>226 \newinsert\bx@HS \expandafter\gdef\expandafter\@freelist\expandafter{\@freelist \@elt\bx@HS}
\ifnum \morefloats@mx>227 \newinsert\bx@HT \expandafter\gdef\expandafter\@freelist\expandafter{\@freelist \@elt\bx@HT}
\ifnum \morefloats@mx>228 \newinsert\bx@HU \expandafter\gdef\expandafter\@freelist\expandafter{\@freelist \@elt\bx@HU}
\ifnum \morefloats@mx>229 \newinsert\bx@HV \expandafter\gdef\expandafter\@freelist\expandafter{\@freelist \@elt\bx@HV}
\ifnum \morefloats@mx>230 \newinsert\bx@HW \expandafter\gdef\expandafter\@freelist\expandafter{\@freelist \@elt\bx@HW}
\ifnum \morefloats@mx>231 \newinsert\bx@HX \expandafter\gdef\expandafter\@freelist\expandafter{\@freelist \@elt\bx@HX}
\ifnum \morefloats@mx>232 \newinsert\bx@HY \expandafter\gdef\expandafter\@freelist\expandafter{\@freelist \@elt\bx@HY}
\ifnum \morefloats@mx>233 \newinsert\bx@HZ \expandafter\gdef\expandafter\@freelist\expandafter{\@freelist \@elt\bx@HZ}
\ifnum \morefloats@mx>234 \newinsert\bx@IA \expandafter\gdef\expandafter\@freelist\expandafter{\@freelist \@elt\bx@IA}
\ifnum \morefloats@mx>235 \newinsert\bx@IB \expandafter\gdef\expandafter\@freelist\expandafter{\@freelist \@elt\bx@IB}
\ifnum \morefloats@mx>236 \newinsert\bx@IC \expandafter\gdef\expandafter\@freelist\expandafter{\@freelist \@elt\bx@IC}
\ifnum \morefloats@mx>237 \newinsert\bx@ID \expandafter\gdef\expandafter\@freelist\expandafter{\@freelist \@elt\bx@ID}
\ifnum \morefloats@mx>238 \newinsert\bx@IE \expandafter\gdef\expandafter\@freelist\expandafter{\@freelist \@elt\bx@IE}
\ifnum \morefloats@mx>239 \newinsert\bx@IF \expandafter\gdef\expandafter\@freelist\expandafter{\@freelist \@elt\bx@IF}
\ifnum \morefloats@mx>240 \newinsert\bx@IG \expandafter\gdef\expandafter\@freelist\expandafter{\@freelist \@elt\bx@IG}
\ifnum \morefloats@mx>241 \newinsert\bx@IH \expandafter\gdef\expandafter\@freelist\expandafter{\@freelist \@elt\bx@IH}
\ifnum \morefloats@mx>242 \newinsert\bx@II \expandafter\gdef\expandafter\@freelist\expandafter{\@freelist \@elt\bx@II}
\ifnum \morefloats@mx>243 \newinsert\bx@IJ \expandafter\gdef\expandafter\@freelist\expandafter{\@freelist \@elt\bx@IJ}
\ifnum \morefloats@mx>244 \newinsert\bx@IK \expandafter\gdef\expandafter\@freelist\expandafter{\@freelist \@elt\bx@IK}
\ifnum \morefloats@mx>245 \newinsert\bx@IL \expandafter\gdef\expandafter\@freelist\expandafter{\@freelist \@elt\bx@IL}
\ifnum \morefloats@mx>246 \newinsert\bx@IM \expandafter\gdef\expandafter\@freelist\expandafter{\@freelist \@elt\bx@IM}
\ifnum \morefloats@mx>247 \newinsert\bx@IN \expandafter\gdef\expandafter\@freelist\expandafter{\@freelist \@elt\bx@IN}
\ifnum \morefloats@mx>248 \newinsert\bx@IO \expandafter\gdef\expandafter\@freelist\expandafter{\@freelist \@elt\bx@IO}
\ifnum \morefloats@mx>249 \newinsert\bx@IP \expandafter\gdef\expandafter\@freelist\expandafter{\@freelist \@elt\bx@IP}
\ifnum \morefloats@mx>250 \newinsert\bx@IQ \expandafter\gdef\expandafter\@freelist\expandafter{\@freelist \@elt\bx@IQ}
\ifnum \morefloats@mx>251 \newinsert\bx@IR \expandafter\gdef\expandafter\@freelist\expandafter{\@freelist \@elt\bx@IR}
\ifnum \morefloats@mx>252 \newinsert\bx@IS \expandafter\gdef\expandafter\@freelist\expandafter{\@freelist \@elt\bx@IS}
\ifnum \morefloats@mx>253 \newinsert\bx@IT \expandafter\gdef\expandafter\@freelist\expandafter{\@freelist \@elt\bx@IT}
\ifnum \morefloats@mx>254 \newinsert\bx@IU \expandafter\gdef\expandafter\@freelist\expandafter{\@freelist \@elt\bx@IU}
\ifnum \morefloats@mx>255 \newinsert\bx@IV \expandafter\gdef\expandafter\@freelist\expandafter{\@freelist \@elt\bx@IV}
%    \end{macrocode}
%
% \newpage
%
%    \begin{macrocode}
\ifnum \morefloats@mx>256\relax%
  \PackageError{morefloats}{Too many floats called for}{%
    You requested more than 256 floats.\MessageBreak%
    (\morefloats@mx\space to be precise.)\MessageBreak%
    LaTeX before 2015 could not process\MessageBreak%
    more than 256 floats, therefore the morefloats\MessageBreak%
    package only provides 256 floats.\MessageBreak%
    If you need more floats,\MessageBreak%
    update to a current (>=2015) LaTeX distribution.\MessageBreak%
    I expected LaTeX (prior 2015) to run out of dimensions\MessageBreak%
    or memory long before reaching 256 floats anyway.\MessageBreak%
   }%
\fi \fi \fi \fi \fi \fi \fi \fi \fi \fi \fi \fi \fi \fi \fi \fi \fi \fi
\fi \fi \fi \fi \fi \fi \fi \fi \fi \fi \fi \fi \fi \fi \fi \fi \fi \fi
\fi \fi \fi \fi \fi \fi \fi \fi \fi \fi \fi \fi \fi \fi \fi \fi \fi \fi
\fi \fi \fi \fi \fi \fi \fi \fi \fi \fi \fi \fi \fi \fi \fi \fi \fi \fi
\fi \fi \fi \fi \fi \fi \fi \fi \fi \fi \fi \fi \fi \fi \fi \fi \fi \fi
\fi \fi \fi \fi \fi \fi \fi \fi \fi \fi \fi \fi \fi \fi \fi \fi \fi \fi
\fi \fi \fi \fi \fi \fi \fi \fi \fi \fi \fi \fi \fi \fi \fi \fi \fi \fi
\fi \fi \fi \fi \fi \fi \fi \fi \fi \fi \fi \fi \fi \fi \fi \fi \fi \fi
\fi \fi \fi \fi \fi \fi \fi \fi \fi \fi \fi \fi \fi \fi \fi \fi \fi \fi
\fi \fi \fi \fi \fi \fi \fi \fi \fi \fi \fi \fi \fi \fi \fi \fi \fi \fi
\fi \fi \fi \fi \fi \fi \fi \fi \fi \fi \fi \fi \fi \fi \fi \fi \fi \fi
\fi \fi \fi \fi \fi \fi \fi \fi \fi \fi \fi \fi \fi \fi \fi \fi \fi \fi
\fi \fi \fi \fi \fi \fi \fi \fi \fi \fi \fi \fi \fi \fi \fi \fi \fi \fi
\fi \fi \fi \fi \fi

%    \end{macrocode}
%
%    \begin{macrocode}
%</package>
%    \end{macrocode}
%
% \end{landscape}
% \newpage
%
% \section{Installation}
%
% \subsection{Downloads\label{ss:Downloads}}
%
% Everything is available at \url{https://www.ctan.org},
% but may need additional packages themselves.\\
%
% \DescribeMacro{morefloats.dtx}
% For unpacking the |morefloats.dtx| file and constructing the documentation it is required:
% \begin{description}
% \item[-] \TeX Format \LaTeXe{}: \url{https://www.CTAN.org}
%
% \item[-] document class \xclass{ltxdoc}, 2015/03/26, v2.0w,
%   \url{https://www.ctan.org/pkg/ltxdoc}
%
% \item[-] package \xpackage{fontenc}, 2005/09/27, v1.99g,
%   \url{https://ctan.org/pkg/fontenc}
%
% \item[-] package \xpackage{pdflscape}, 2008/08/11, v0.10,
%   \url{https://ctan.org/pkg/pdflscape}
%
% \item[-] package \xpackage{holtxdoc}, 2012/03/21, v0.24,
%   \url{https://ctan.org/pkg/holtxdoc}
%
% \item[-] package \xpackage{hypdoc}, 2011/08/19, v1.11,
%   \url{https://ctan.org/pkg/hypdoc}
% \end{description}
%
% \DescribeMacro{morefloats.sty}
% The \texttt{morefloats.sty} for \LaTeXe{} \hbox{(i.\,e. each} document using
% the \xpackage{morefloats} package) requires:
% \begin{description}
% \item[-] \TeX Format \LaTeXe{}, \url{https://www.CTAN.org/}
%
% \item[-] package \xpackage{kvoptions}, 2011/06/30, v3.11,
%   \url{https://ctan.org/pkg/kvoptions}
%
% \item[-] package \xpackage{ifetex}, 2011/12/15, v1.2,
%   \url{https://ctan.org/pkg/ifetex}, is used in some cases
% \end{description}
%
% \DescribeMacro{regstats}
% \DescribeMacro{regcount}
% To check the number of used registers it was mentioned:
% \begin{description}
% \item[-] package \xpackage{regstats}, \url{https://ctan.org/pkg/regstats}
% \item[-] package \xpackage{regcount}, \url{https://ctan.org/pkg/regcount}
% \end{description}
%
% \DescribeMacro{Oberdiek}
% \DescribeMacro{holtxdoc}
% \DescribeMacro{hypdoc}
% All packages of \textsc{Heiko Oberdiek}'s bundle `oberdiek'
% (especially \xpackage{holtxdoc}, \xpackage{hypdoc}, and \xpackage{kvoptions})
% are also available in a TDS compliant ZIP archive:\\
% \url{http://mirror.ctan.org/install/macros/latex/contrib/oberdiek.tds.zip}.\\
% It is probably best to download and use this, because the packages in there
% are quite probably both recent and compatible among themselves.\\
%
% \DescribeMacro{hyperref}
% \noindent \xpackage{hyperref} is not included in that bundle and needs to be
% downloaded separately,\\
% \url{http://mirror.ctan.org/install/macros/latex/contrib/hyperref.tds.zip}.\\
%
% \DescribeMacro{M\"{u}nch}
% A hyperlinked list of my (other) packages can be found at
% \url{https://www.ctan.org/author/muench-hm}.\\
%
% \subsection{Package, unpacking TDS}
% \paragraph{Package.} This package is available on \url{https://www.CTAN.org}.
% \begin{description}
% \item[\url{http://mirror.ctan.org/macros/latex/contrib/morefloats/morefloats.dtx}]\hspace*{0.1cm}
%       The source file.
% \item[\url{http://mirror.ctan.org/macros/latex/contrib/morefloats/morefloats.pdf}]\hspace*{0.1cm}
%       The documentation.
% \item[\url{http://mirror.ctan.org/macros/latex/contrib/morefloats/README}]\hspace*{0.1cm}\\
%       \hspace*{1em}The README file.
% \end{description}
%
% \noindent There is also a |morefloats.tds.zip| available:
% \begin{description}
% \item[\url{http://mirror.ctan.org/install/macros/latex/contrib/morefloats.tds.zip}]\hspace*{0.1cm}
%       Everything in TDS compliant, compiled format.
% \end{description}
% which additionally contains\\
% \begin{tabular}{ll}
% morefloats.ins & The installation file.\\
% morefloats.drv & The driver to generate the documentation.\\
% morefloats.sty & The \xext{sty}le file.\\
% morefloats-example.tex & The example file.\\
% morefloats-example.pdf & The compiled example file.
% \end{tabular}
%
% \bigskip
%
% \noindent For required other packages, please see the preceding subsection.
%
% \paragraph{Unpacking.} The  \xfile{.dtx} file is a self-extracting
% \docstrip{} archive. The files are extracted by running the
% \xfile{.dtx} through \plainTeX{}:
% \begin{quote}
%   \verb|tex morefloats.dtx|
% \end{quote}
%
% About generating the documentation see paragraph~\ref{GenDoc} below.\\
%
% \paragraph{TDS.} Now the different files must be moved into
% the different directories in your installation TDS tree
% (also known as \xfile{texmf} tree):
% \begin{quote}
% \def\t{^^A
% \begin{tabular}{@{}>{\ttfamily}l@{ $\rightarrow$ }>{\ttfamily}l@{}}
%   morefloats.sty & tex/latex/morefloats/morefloats.sty\\
%   morefloats.pdf & doc/latex/morefloats/morefloats.pdf\\
%   morefloats-example.tex & doc/latex/morefloats/morefloats-example.tex\\
%   morefloats-example.pdf & doc/latex/morefloats/morefloats-example.pdf\\
%   morefloats.dtx & source/latex/morefloats/morefloats.dtx\\
% \end{tabular}^^A
% }^^A
% \sbox0{\t}^^A
% \ifdim\wd0>\linewidth
%   \begingroup
%     \advance\linewidth by\leftmargin
%     \advance\linewidth by\rightmargin
%   \edef\x{\endgroup
%     \def\noexpand\lw{\the\linewidth}^^A
%   }\x
%   \def\lwbox{^^A
%     \leavevmode
%     \hbox to \linewidth{^^A
%       \kern-\leftmargin\relax
%       \hss
%       \usebox0
%       \hss
%       \kern-\rightmargin\relax
%     }^^A
%   }^^A
%   \ifdim\wd0>\lw
%     \sbox0{\small\t}^^A
%     \ifdim\wd0>\linewidth
%       \ifdim\wd0>\lw
%         \sbox0{\footnotesize\t}^^A
%         \ifdim\wd0>\linewidth
%           \ifdim\wd0>\lw
%             \sbox0{\scriptsize\t}^^A
%             \ifdim\wd0>\linewidth
%               \ifdim\wd0>\lw
%                 \sbox0{\tiny\t}^^A
%                 \ifdim\wd0>\linewidth
%                   \lwbox
%                 \else
%                   \usebox0
%                 \fi
%               \else
%                 \lwbox
%               \fi
%             \else
%               \usebox0
%             \fi
%           \else
%             \lwbox
%           \fi
%         \else
%           \usebox0
%         \fi
%       \else
%         \lwbox
%       \fi
%     \else
%       \usebox0
%     \fi
%   \else
%     \lwbox
%   \fi
% \else
%   \usebox0
% \fi
% \end{quote}
% If you have a \xfile{docstrip.cfg} that configures and enables \docstrip's
% TDS installing feature, then some files can already be in the right
% place, see the documentation of \docstrip{}.
%
% \subsection{Refresh file name databases}
%
% If your \TeX~distribution (\TeX{} Live, \mikTeX, \teTeX, \dots) relies on
% file name databases, you must refresh these. For example, \teTeX{} users run
% \verb|texhash| or \verb|mktexlsr|.
%
% \subsection{Some details for the interested}
%
% \paragraph{Unpacking with \LaTeX{}.}
% The \xfile{.dtx} chooses its action depending on the format:
% \begin{description}
% \item[\plainTeX:] Run \docstrip{} and extract the files.
% \item[\LaTeX:] Generate the documentation.
% \end{description}
% If you insist on using \LaTeX{} for \docstrip{} (really,
% \docstrip{} does not need \LaTeX ), then inform the autodetect routine
% about your intention:
% \begin{quote}
%   \verb|latex \let\install=y% \iffalse meta-comment
%
% File: morefloats.dtx
% Version: 2015/07/22 v1.0h
%
% Copyright (C) 2010 - 2015 by
%    H.-Martin M"unch <Martin dot Muench at Uni-Bonn dot de>
% Portions of code copyrighted by other people as marked.
%
% LaTeX 2015 provides the extrafloats command.
% Don Hosek, Quixote, 1990/07/27 (Thanks!)
% invented the main code for handling more floats
% before extrafloats was available.
% Maintenance has been taken over in September 2010
% by H.-Martin M\"{u}nch.
% David Carlisle pointed the maintainer to the new
% extrafloats command (Thanks!).
%
% This work may be distributed and/or modified under the
% conditions of the LaTeX Project Public License, either
% version 1.3c of this license or (at your option) any later
% version. This version of this license is in
%    http://www.latex-project.org/lppl/lppl-1-3c.txt
% and the latest version of this license is in
%    http://www.latex-project.org/lppl.txt
% and version 1.3c or later is part of all distributions of
% LaTeX version 2005/12/01 or later.
%
% This work has the LPPL maintenance status "maintained".
%
% The Current Maintainer of this work is H.-Martin Muench.
%
% This work consists of the main source file morefloats.dtx,
% the README, and the derived files
%    morefloats.sty, morefloats.pdf,
%    morefloats.ins, morefloats.drv,
%    morefloats-example.tex, morefloats-example.pdf.
%
% 'morefloats' is available on CTAN:
% https://www.ctan.org/pkg/morefloats
%
% Also a TDS.ZIP file is provided that contains all the files
% already sorted in a TDS tree:
% http://mirror.ctan.org/install/macros/latex/contrib/morefloats.tds.zip
%
%<*ignore>
\begingroup
  \catcode123=1 %
  \catcode125=2 %
  \def\x{LaTeX2e}%
\expandafter\endgroup
\ifcase 0\ifx\install y1\fi\expandafter
         \ifx\csname processbatchFile\endcsname\relax\else1\fi
         \ifx\fmtname\x\else 1\fi\relax
\else\csname fi\endcsname
%</ignore>
%<*install>
\input docstrip.tex
\Msg{*******************************************************************************}
\Msg{* Installation                                                                *}
\Msg{* Package: morefloats 2015/07/22 v1.0h Raise limit of unprocessed floats (HMM)*}
\Msg{*******************************************************************************}

\keepsilent
\askforoverwritefalse

\let\MetaPrefix\relax
\preamble

This is a generated file.

Project: morefloats
Version: 2015/07/22 v1.0h

Copyright (C) 2010 - 2015 by
    H.-Martin M"unch <Martin dot Muench at Uni-Bonn dot de>
Portions of code copyrighted by other people as marked.

The usual disclaimer applies:
If it doesn't work right that's your problem.
(Nevertheless, send an e-mail to the maintainer
 when you find an error in this package.)

This work may be distributed and/or modified under the
conditions of the LaTeX Project Public License, either
version 1.3c of this license or (at your option) any later
version. This version of this license is in
   http://www.latex-project.org/lppl/lppl-1-3c.txt
and the latest version of this license is in
   http://www.latex-project.org/lppl.txt
and version 1.3c or later is part of all distributions of
LaTeX version 2005/12/01 or later.

This work has the LPPL maintenance status "maintained".

The Current Maintainer of this work is H.-Martin Muench.

LaTeX 2015 provides the extrafloats command.
Don Hosek, Quixote, 1990/07/27 (Thanks!)
invented the main code for handling more floats
before extrafloats was available.
Maintenance has been taken over in September 2010
by H.-Martin Muench.
David Carlisle pointed the maintainer to the new
extrafloats command (Thanks!).

This work consists of the main source file morefloats.dtx,
the README, and the derived files
   morefloats.sty, morefloats.pdf,
   morefloats.ins, morefloats.drv,
   morefloats-example.tex, morefloats-example.pdf.

In memoriam
 Claudia Simone Barth + 1996/01/30
 Tommy Muench + 2014/01/02
 Hans-Klaus Muench + 2014/08/24

\endpreamble
\let\MetaPrefix\DoubleperCent

\generate{%
  \file{morefloats.ins}{\from{morefloats.dtx}{install}}%
  \file{morefloats.drv}{\from{morefloats.dtx}{driver}}%
  \usedir{tex/latex/morefloats}%
  \file{morefloats.sty}{\from{morefloats.dtx}{package}}%
  \usedir{doc/latex/morefloats}%
  \file{morefloats-example.tex}{\from{morefloats.dtx}{example}}%
}

\catcode32=13\relax% active space
\let =\space%
\Msg{************************************************************************}
\Msg{*}
\Msg{* To finish the installation you have to move the following}
\Msg{* file into a directory searched by TeX:}
\Msg{*}
\Msg{*  morefloats.sty}
\Msg{*}
\Msg{* To produce the documentation run the file `morefloats.drv'}
\Msg{* through (pdf)LaTeX, e.g.}
\Msg{*  pdflatex morefloats.drv}
\Msg{*  makeindex -s gind.ist morefloats.idx}
\Msg{*  pdflatex morefloats.drv}
\Msg{*  makeindex -s gind.ist morefloats.idx}
\Msg{*  pdflatex morefloats.drv}
\Msg{*}
\Msg{* At least three runs are necessary e.g. to get the}
\Msg{*  references right!}
\Msg{*}
\Msg{* Happy TeXing!}
\Msg{*}
\Msg{************************************************************************}

\endbatchfile
%</install>
%<*ignore>
\fi
%</ignore>
%
% \section{The documentation driver file}
%
% The next bit of code contains the documentation driver file for
% \TeX , i.\,e., the file that will produce the documentation you
% are currently reading. It will be extracted from this file by the
% \texttt{docstrip} programme. That is, run \LaTeX{} on \texttt{docstrip}
% and specify the \texttt{driver} option when \texttt{docstrip}
% asks for options.
%
%    \begin{macrocode}
%<*driver>
\NeedsTeXFormat{LaTeX2e}[2015/01/01]
\ProvidesFile{morefloats.drv}%
  [2015/07/22 v1.0h Raise limit of unprocessed floats (HMM)]
\documentclass{ltxdoc}[2015/03/26]%   v2.0w
\usepackage[T1]{fontenc}[2005/09/27]% v1.99g
\usepackage{pdflscape}[2008/08/11]%   v0.10
\usepackage{holtxdoc}[2012/03/21]%    v0.24
%% morefloats should work with earlier versions of LaTeX2e and
%% may work with earlier versions of the class and those packages,
%% but this was not tested.
%% Please consider updating your LaTeX, class, and packages
%% to the most recent version (if they are not already the most
%% recent version).
\hypersetup{%
 pdfsubject={LaTeX2e package for increasing the limit of unprocessed floats (HMM)},%
 pdfkeywords={LaTeX, morefloats, floats, H.-Martin Muench},%
 pdfencoding=auto,%
 pdflang={en},%
 breaklinks=true,%
 linktoc=all,%
 pdfstartview=FitH,%
 pdfpagelayout=OneColumn,%
 bookmarksnumbered=true,%
 bookmarksopen=true,%
 bookmarksopenlevel=2,%
 pdfmenubar=true,%
 pdftoolbar=true,%
 pdfwindowui=true,%
 pdfnewwindow=true%
}
\CodelineIndex
\hyphenation{docu-ment}
\gdef\unit#1{\mathord{\thinspace\mathrm{#1}}}%
\begin{document}
  \DocInput{morefloats.dtx}%
\end{document}
%</driver>
%    \end{macrocode}
%
% \fi
%
% \CheckSum{3565}
%
% \CharacterTable
%  {Upper-case    \A\B\C\D\E\F\G\H\I\J\K\L\M\N\O\P\Q\R\S\T\U\V\W\X\Y\Z
%   Lower-case    \a\b\c\d\e\f\g\h\i\j\k\l\m\n\o\p\q\r\s\t\u\v\w\x\y\z
%   Digits        \0\1\2\3\4\5\6\7\8\9
%   Exclamation   \!     Double quote  \"     Hash (number) \#
%   Dollar        \$     Percent       \%     Ampersand     \&
%   Acute accent  \'     Left paren    \(     Right paren   \)
%   Asterisk      \*     Plus          \+     Comma         \,
%   Minus         \-     Point         \.     Solidus       \/
%   Colon         \:     Semicolon     \;     Less than     \<
%   Equals        \=     Greater than  \>     Question mark \?
%   Commercial at \@     Left bracket  \[     Backslash     \\
%   Right bracket \]     Circumflex    \^     Underscore    \_
%   Grave accent  \`     Left brace    \{     Vertical bar  \|
%   Right brace   \}     Tilde         \~}
%
% \GetFileInfo{morefloats.drv}
%
% \begingroup
%   \def\x{\#,\$,\^,\_,\~,\ ,\&,\{,\},\%}%
%   \makeatletter
%   \@onelevel@sanitize\x
% \expandafter\endgroup
% \expandafter\DoNotIndex\expandafter{\x}
% \expandafter\DoNotIndex\expandafter{\string\ }
% \begingroup
%   \makeatletter
%     \lccode`9=32\relax
%     \lowercase{%^^A
%       \edef\x{\noexpand\DoNotIndex{\@backslashchar9}}%^^A
%     }%^^A
%   \expandafter\endgroup\x
% \DoNotIndex{\\,\,}
% \DoNotIndex{\def,\edef,\gdef, \xdef}
% \DoNotIndex{\ifnum, \ifx}
% \DoNotIndex{\begin, \end, \LaTeX, \LateXe}
% \DoNotIndex{\bigskip, \caption, \centering, \hline, \MessageBreak}
% \DoNotIndex{\documentclass, \markboth, \mathrm, \mathord}
% \DoNotIndex{\NeedsTeXFormat, \usepackage, \ProvidesPackage, \RequirePackage}
% \DoNotIndex{\newline, \newpage, \pagebreak}
% \DoNotIndex{\section, \subsection, \space, \thinspace}
% \DoNotIndex{\textsf, \texttt}
% \DoNotIndex{\the, \@tempcnta,\@tempcntb}
% \DoNotIndex{\@elt,\@freelist, \newinsert}
% \DoNotIndex{\bx@A,  \bx@B,  \bx@C,  \bx@D,  \bx@E,  \bx@F,  \bx@G,  \bx@H,  \bx@I,  \bx@J,  \bx@K,  \bx@L,  \bx@M,  \bx@N,  \bx@O,  \bx@P,  \bx@Q,  \bx@R,  \bx@S,  \bx@T,  \bx@U,  \bx@V,  \bx@W,  \bx@X,  \bx@Y,  \bx@Z}
% \DoNotIndex{\bx@AA, \bx@AB, \bx@AC, \bx@AD, \bx@AE, \bx@AF, \bx@AG, \bx@AH, \bx@AI, \bx@AJ, \bx@AK, \bx@AL, \bx@AM, \bx@AN, \bx@AO, \bx@AP, \bx@AQ, \bx@AR, \bx@AS, \bx@AT, \bx@AU, \bx@AV, \bx@AW, \bx@AX, \bx@AY, \bx@AZ}
% \DoNotIndex{\bx@BA, \bx@BB, \bx@BC, \bx@BD, \bx@BE, \bx@BF, \bx@BG, \bx@BH, \bx@BI, \bx@BJ, \bx@BK, \bx@BL, \bx@BM, \bx@BN, \bx@BO, \bx@BP, \bx@BQ, \bx@BR, \bx@BS, \bx@BT, \bx@BU, \bx@BV, \bx@BW, \bx@BX, \bx@BY, \bx@BZ}
% \DoNotIndex{\bx@CA, \bx@CB, \bx@CC, \bx@CD, \bx@CE, \bx@CF, \bx@CG, \bx@CH, \bx@CI, \bx@CJ, \bx@CK, \bx@CL, \bx@CM, \bx@CN, \bx@CO, \bx@CP, \bx@CQ, \bx@CR, \bx@CS, \bx@CT, \bx@CU, \bx@CV, \bx@CW, \bx@CX, \bx@CY, \bx@CZ}
% \DoNotIndex{\bx@DA, \bx@DB, \bx@DC, \bx@DD, \bx@DE, \bx@DF, \bx@DG, \bx@DH, \bx@DI, \bx@DJ, \bx@DK, \bx@DL, \bx@DM, \bx@DN, \bx@DO, \bx@DP, \bx@DQ, \bx@DR, \bx@DS, \bx@DT, \bx@DU, \bx@DV, \bx@DW, \bx@DX, \bx@DY, \bx@DZ}
% \DoNotIndex{\bx@EA, \bx@EB, \bx@EC, \bx@ED, \bx@EE, \bx@EF, \bx@EG, \bx@EH, \bx@EI, \bx@EJ, \bx@EK, \bx@EL, \bx@EM, \bx@EN, \bx@EO, \bx@EP, \bx@EQ, \bx@ER, \bx@ES, \bx@ET, \bx@EU, \bx@EV, \bx@EW, \bx@EX, \bx@EY, \bx@EZ}
% \DoNotIndex{\bx@FA, \bx@FB, \bx@FC, \bx@FD, \bx@FE, \bx@FF, \bx@FG, \bx@FH, \bx@FI, \bx@FJ, \bx@FK, \bx@FL, \bx@FM, \bx@FN, \bx@FO, \bx@FP, \bx@FQ, \bx@FR, \bx@FS, \bx@FT, \bx@FU, \bx@FV, \bx@FW, \bx@FX, \bx@FY, \bx@FZ}
% \DoNotIndex{\bx@GA, \bx@GB, \bx@GC, \bx@GD, \bx@GE, \bx@GF, \bx@GG, \bx@GH, \bx@GI, \bx@GJ, \bx@GK, \bx@GL, \bx@GM, \bx@GN, \bx@GO, \bx@GP, \bx@GQ, \bx@GR, \bx@GS, \bx@GT, \bx@GU, \bx@GV, \bx@GW, \bx@GX, \bx@GY, \bx@GZ}
% \DoNotIndex{\bx@HA, \bx@HB, \bx@HC, \bx@HD, \bx@HE, \bx@HF, \bx@HG, \bx@HH, \bx@HI, \bx@HJ, \bx@HK, \bx@HL, \bx@HM, \bx@HN, \bx@HO, \bx@HP, \bx@HQ, \bx@HR, \bx@HS, \bx@HT, \bx@HU, \bx@HV, \bx@HW, \bx@HX, \bx@HY, \bx@HZ}
% \DoNotIndex{\bx@IA, \bx@IB, \bx@IC, \bx@ID, \bx@IE, \bx@IF, \bx@IG, \bx@IH, \bx@II, \bx@IJ, \bx@IK, \bx@IL, \bx@IM, \bx@IN, \bx@IO, \bx@IP, \bx@IQ, \bx@IR, \bx@IS, \bx@IT, \bx@IU, \bx@IV, \bx@IW, \bx@IX, \bx@IY, \bx@IZ}
% \DoNotIndex{\bx@JA, \bx@JB, \bx@JC, \bx@JD, \bx@JE, \bx@JF, \bx@JG, \bx@JH, \bx@JI, \bx@JJ, \bx@JK, \bx@JL, \bx@JM, \bx@JN, \bx@JO, \bx@JP, \bx@JQ, \bx@JR, \bx@JS, \bx@JT, \bx@JU, \bx@JV, \bx@JW, \bx@JX, \bx@JY, \bx@JZ}
% \DoNotIndex{\morefloats@mx}
%
% \title{The \xpackage{morefloats} package}
% \date{2015/07/22 v1.0h}
% \author{H.-Martin M\"{u}nch (current maintainer;\\
%  invented by Don Hosek, Quixote)\\
%  \xemail{Martin.Muench at Uni-Bonn.de}}
%
% \maketitle
%
% \begin{abstract}
% The default limit of unprocessed floats, $18$,
% can be increased with this \xpackage{morefloats} package.
% Otherwise, |\clear(double)page|, |h(!)|, |H|~from the \xpackage{float} package,
% or |\FloatBarrier| from the \xpackage{picins} package might help.
% \end{abstract}
%
% \bigskip
%
% \noindent Note: \LaTeX{} 2015 provides the |\extrafloats| command.
% \textsc{Don Hosek}, Quixote, 1990/07/27 (Thanks!)
% invented the main code for handling more floats
% before |\extrafloats| was available.
% \textsc{David Carlisle} pointed the maintainer to the new
% |\extrafloats| (Thanks!).
% The current maintainer is \textsc{H.-Martin M\"{u}nch}.\\
%
% \bigskip
%
% \noindent Disclaimer for web links: The author is not responsible for any contents
% referred to in this work unless he has full knowledge of illegal contents.
% If any damage occurs by the use of information presented there, only the
% author of the respective pages might be liable, not the one who has referred
% to these pages.
%
% \bigskip
%
% \noindent {\color{green} Save per page about $200\unit{ml}$ water,
% $2\unit{g}$ CO$_{2}$ and $2\unit{g}$ wood:\\
% Therefore please print only if this is really necessary.}
%
% \newpage
%
% \tableofcontents
%
% \newpage
%
% \section{Introduction\label{sec:Introduction}}
%
% The default limit of unprocessed floats, $18$,
% can be increased with this \xpackage{morefloats} package.\\
% \textquotedblleft{}Of course one immediately begins to wonder:
% \guillemotright{}Why eighteen?!\guillemotleft{} And it turns out that $18$
% one{-}line tables with $10$~point Computer Modern using \xclass{article.cls}
% produces almost exactly one page worth of material.\textquotedblright{}\\
% (user \url{https://tex.stackexchange.com/users/1495/kahen} as comment to\\
% \url{https://tex.stackexchange.com/a/35596/6865} on 2011/11/21)\\
% As alternatives (see also section \ref{sec:alternatives} below)
% |\clear(double)page|, |h(!)|, |H|~from the
% \href{https://www.ctan.org/pkg/float}{\xpackage{float}} package,
% or |\FloatBarrier| from the %
% \href{https://www.ctan.org/pkg/picins}{\xpackage{picins}} package might help.
% If the floats cannot be placed anywhere at all, extending the number of floats
% will just delay the arrival of the corresponding error.
%
% \section{Usage}
%
% \subsection{General usage:}
% Load the package placing
% \begin{quote}
%   |\usepackage[<|\textit{options}|>]{morefloats}|
% \end{quote}
% \noindent in the preamble of your \LaTeXe{} source file (the earlier the better).\\
% \noindent The \xpackage{morefloats} package takes two options: |maxfloats| and
% |morefloats|, where |morefloats| gives the number of additional floats and
% |maxfloats| gives the maximum number of floats. |maxfloats=25| therefore means,
% that there are $18$ (default) floats and $7$ additional floats.
% |morefloats=7| therefore has the same meaning. It is only necessary to give
% one of these two options. At the time being, it is not possible to reduce
% the number of floats (for example to save boxes). If you have code
% accomplishing that, please send it to the package maintainer, thanks.\\
% Version 1.0b used a fixed value of |maxfloats=36|. Therefore for backward
% compatibility this value is taken as the default one.\\
% Example:
% \begin{quote}
%   |\usepackage[maxfloats=25]{morefloats}|
% \end{quote}
% or
% \begin{quote}
%   |\usepackage[morefloats=7]{morefloats}|
% \end{quote}
% or
% \begin{quote}
%   |\usepackage[maxfloats=25,morefloats=7]{morefloats}|
% \end{quote}
%
% \subsection{Situation for \LaTeX{} before 2015:}
% |Float| uses |insert|, and each |insert| uses a group of |count|, |dimen|,
% |skip|, and |box| each. When there are not enough available, no |\newinsert|
% can be created. The
% \href{https://www.ctan.org/pkg/etex-pkg}{\xpackage{etex}} package
% provides access at an extended range of those registers,
% but does not use those for |\newinsert|. Therefore the inserts must be
% reserved first, which forces the use of the extended register range
% for other new |count|, |dimen|, |skip|, and |box|:
% To have more floats available, use |\usepackage{etex}\reserveinserts{...}|
% right after |\documentclass[...]{...}|, where the argument of |\reserveinserts|
% should be at least the maximum number of floats. Add another $10$
% if the \href{https://www.ctan.org/pkg/bigfoot}{\xpackage{bigfoot}} or the
% \href{https://www.ctan.org/pkg/manyfoot}{\xpackage{manyfoot}} package
% is used, but |\reserveinserts| can be about $234$ at most for older
% \LaTeX{} formats.
%
% \subsection{Situation for \LaTeX{} since 2015:}
% Now |\reserveinserts| can be about $2\,147\,483\,647$,
% but |\insert255{}| even then produces an error.
% The \LaTeX{} 2015 \textquotedblleft release provides a new command in the format
% |\extrafloats|\textquotedblright ; \textquotedblleft as it doesn't use
% |\newinsert| (and as the 2015 format uses extended registers by default)
% you can allocate a lot more floats\textquotedblright{} %
% (both \textsc{David Carlisle}, 29. June 2015), \hbox{e.\,g. |\extrafloats{1234}|.}
%
% \section{Alternatives (kind of)\label{sec:alternatives}}
%
% The very old \xpackage{morefloats} with a fixed number of |maxfloats=36| {}%
% \hbox{(i.\,e. $18$ |morefloats|)} has been archived at
% \href{http://mirror.ctan.org/obsolete/macros/latex/contrib/misc/morefloats.sty}{%
%  http://mirror.ctan.org/obsolete/macros/latex/contrib/}\newline%
% \href{http://mirror.ctan.org/obsolete/macros/latex/contrib/misc/morefloats.sty}{%
%  misc/morefloats.sty}.
%
% \bigskip
%
% If you really want to increase the number of (possible) floats,
% this is the right package. On the other hand, if you ran into trouble of
% \texttt{Too many unprocessed floats}, but would also accept less floats,
% there are some other possibilities:
% \begin{description}
%   \item[-] The command |\clearpage| forces \LaTeX{} to output any floating objects
%     that occurred before this command (and go to the next page).
%     |\cleardoublepage| does the same but ensures that the next page with
%     output is one with odd page number.
%   \item[-] Using different float specifiers: |t|~top, |b|~bottom, |p|~page
%     of floats.
%   \item[-] Suggesting \LaTeX{} to put the object where it was placed:
%     |h| (= here) float specifier.
%   \item[-] Telling \LaTeX{} to please put the object where it was placed:
%     |h!| (= here!) float specifier.
%   \item[-] Forcing \LaTeX{} to put the object where it was placed and shut up:
%     The \xpackage{float} package provides the \textquotedblleft style
%     option here, giving floating environments a [H] option which means
%     `PUT IT HERE' (as opposed to the standard [h] option which means
%     `You may put it here if you like')\textquotedblright{} (\xpackage{float}
%     package documentation v1.3d as of 2001/11/08).
%     Changing e.\,g. |\begin{figure}[tbp]...| to |\begin{figure}[H]...|
%     forces the figure to be placed HERE instead of floating away.\\
%     The \xpackage{float} package is available at \url{https://www.ctan.org/pkg/float}.
%   \item[-] The \xpackage{placeins} package provides the command |\FloatBarrier|.
%     Floats occurring before the |\FloatBarrier| are not allowed to float
%     to a later place, and floats occurring after the |\FloatBarrier| are not
%     allowed to float to an earlier place than the |\FloatBarrier|. (There
%     can be more than one |\FloatBarrier| in a document.) -- %
%     The same package also provides an option to automatically add |\FloatBarrier|s to
%     section headings. It is further possible to make
%     |\FloatBarrier|s less strict (see that package's documentation).\\
%     The \xpackage{placeins} package is available at \url{https://www.ctan.org/pkg/placeins}.
%   \item[-] Sometimes also increasing the maximum number (|\maxdeadcycles|)
%     of calls of |\output| without a |\shipout| can help,
%     for example |\maxdeadcycles=123\relax|.
% \end{description}
%
% \newpage
%
% \noindent See also the following entries in the
% \texttt{UK~List of TeX Frequently Asked Questions on the Web}:
% \begin{description}
%   \item[-] \url{http://www.tex.ac.uk/cgi-bin/texfaq2html?label=floats}
%   \item[-] \url{http://www.tex.ac.uk/cgi-bin/texfaq2html?label=tmupfl}
%   \item[-] \url{http://www.tex.ac.uk/cgi-bin/texfaq2html?label=figurehere}
% \end{description}
% and the \textbf{excellent article on \textquotedblleft How to influence the position
% of float environments like figure and table in \hbox{\LaTeX ?\textquotedblright } by
% \textsc{Frank Mittelbach}} at \url{https://tex.stackexchange.com/a/39020/6865}{}!\\
%
% \bigskip
%
% \noindent (You programmed or found another alternative,
%  which is available at CTAN?\\
%  OK, send an e-mail to me with the name, location at CTAN,
%  and a short notice, and I will probably include it in
%  the list above.)
%
% \bigskip
%
% \section{Example}
%
%    \begin{macrocode}
%<*example>
\documentclass[british]{article}[2014/09/29]%      v1.4h
%%%%%%%%%%%%%%%%%%%%%%%%%%%%%%%%%%%%%%%%%%%%%%%%%%%%%%%%%%%%%%%%%%%%%
\usepackage[maxfloats=25]{morefloats}[2015/07/22]% v1.0h
%\maxdeadcycles=200\relax%
%% \maxdeadcycles is the maximum number of calls of \output
%% without a \shipout.
\gdef\unit#1{\mathord{\thinspace\mathrm{#1}}}%
\listfiles
\begin{document}

\makeatletter

\section*{Example for morefloats}
\markboth{Example for morefloats}{Example for morefloats}

This example demonstrates the use of package\newline
\textsf{morefloats}, v1.0h as of 2015/07/22 (HMM).\newline
The package takes options (here:
\verb|maxfloats=|\texttt{\morefloats@maxfloats} is used).\newline
For more details please see the documentation!\newline

To reproduce the\newline
\LaTeX{} \texttt{ Error: Too many unprocessed floats},\newline
comment out the \verb|\usepackage...| in the preamble
(line~3)\newline
(by placing a \% before it).\newline

\bigskip

Save per page about $200\unit{ml}$~water, $2\unit{g}$~CO$_{2}$
and $2\unit{g}$~wood:\newline
Therefore please print only if this is really necessary.\newline
I do NOT think, that it is necessary to print THIS file, really!

\bigskip

There follow \morefloats@maxfloats{} floating tables.

\pagebreak

\@tempcnta=18\relax% default floats
\advance\@tempcnta by \morefloats@morefloats%
% \morefloats@morefloats is the number of additional
% floating tables to create.
\loop
  \ifnum\@tempcnta>0\relax%
  \begin{table}[t]\centering%
    \begin{tabular}{|l|}%
      \hline%
      A table, which will keep floating.\\%
      \hline
    \end{tabular}%
    \caption{A floating Table.}%
  \end{table}%
  \advance\@tempcnta by -1\relax%
\repeat

\makeatother

\end{document}
%</example>
%    \end{macrocode}
%
% \newpage
%
% \StopEventually{}
%
% \section{The implementation}
%
% We start off by checking that we are loading into \LaTeXe{} and
% announcing the name and version of this package.
%
%    \begin{macrocode}
%<*package>
%    \end{macrocode}
%
%    \begin{macrocode}
\NeedsTeXFormat{LaTeX2e}[2011/06/27]
%% The current format at the time of the release of this version of the
%% morefloats package was 2015/01/01, patch level 2.
\ProvidesPackage{morefloats}[2015/07/22 v1.0h
            Raise limit of unprocessed floats (HMM)]

%    \end{macrocode}
%
% \DescribeMacro{Options}
%    \begin{macrocode}
\RequirePackage{kvoptions}[2011/06/30]% v3.11
%% morefloats may work with earlier versions of LaTeX2e and that
%% package, but this was not tested.
%% Please consider updating your LaTeX and package
%% to the most recent version (if they are not already the most
%% recent version).

\SetupKeyvalOptions{family=morefloats,prefix=morefloats@}
\DeclareStringOption{maxfloats}%  \morefloats@maxfloats
\DeclareStringOption{morefloats}% \morefloats@morefloats

\ProcessKeyvalOptions*

%    \end{macrocode}
%
% The \xpackage{morefloats} package takes two options: |maxfloats| and |morefloats|,
% where |morefloats| gives the number of additional floats and |maxfloats| gives
% the maximum number of floats. |maxfloats=37| therefore means, that there are
% $18$ (default) floats and another $19$ additional floats. |morefloats=19| therefore
% has the same meaning. Version~1.0b used a fixed value of |maxfloats=36|.
% Therefore for backward compatibility this value will be taken as the default one.\\
% Now we check whether |maxfloats=...| or |morefloats=...| or both were used,
% and if one option was not used, we supply the according value.
% If no option was used at all, we use the default values.
% Too many requested floats produce error massages by \LaTeX ,
% which might not be easily traced back to this,
% therefore we issue a warning. If option |maxfloats| or |morefloats| is no number,
% the user will received the according error message by \LaTeX{} automatically.
%
%    \begin{macrocode}
\ifx\morefloats@maxfloats\@empty%
  \ifx\morefloats@morefloats\@empty% apply defaults:
    \gdef\morefloats@maxfloats{36}%
    \gdef\morefloats@morefloats{18}%
  \else%
    \ifnum\morefloats@morefloats>1569\relax%
      \PackageWarning{morefloats}{%
        \morefloats@morefloats\space more floats requested.\MessageBreak%
        LaTeX might run out of memory before this\MessageBreak%
        (in which case it will notify you)\MessageBreak%
       }%
    \else%
      \PackageInfo{morefloats}{%
        \morefloats@morefloats\space more floats requested.\MessageBreak%
        LaTeX might run out of memory before this\MessageBreak%
        (in which case it will notify you)\MessageBreak%
       }%
    \fi%
    \@tempcnta=\morefloats@morefloats\relax%
    \advance\@tempcnta by +18%
    \xdef\morefloats@maxfloats{\the\@tempcnta}%
  \fi%
\else%
  \ifx\morefloats@morefloats\@empty%
    \@tempcnta=\morefloats@maxfloats\relax%
    \advance\@tempcnta by -18%
    \xdef\morefloats@morefloats{\the\@tempcnta}%
    \ifnum\morefloats@morefloats<\z@\relax% i.e. \morefloats@maxfloats < 18
      \gdef\morefloats@morefloats{0}%
    \fi%
    \ifnum\morefloats@maxfloats>1587\relax%
      \PackageWarning{morefloats}{%
        \morefloats@maxfloats\space floats requested.\MessageBreak%
        LaTeX might run out of memory before this\MessageBreak%
        (in which case it will notify you)\MessageBreak%
       }%
    \fi%
  \fi%
\fi%

\@tempcnta=\morefloats@maxfloats\relax%
\xdef\morefloats@max{\the\@tempcnta}%

\ifnum\@tempcnta<18\relax%
  \PackageError{morefloats}{Option maxfloats is \the\@tempcnta<18}{%
    maxfloats must be a number equal to or larger than 18\MessageBreak%
    (or not used at all).\MessageBreak%
    Now setting maxfloats=18.\MessageBreak%
   }%
  \gdef\morefloats@max{18}%
\fi%

\@tempcnta=\morefloats@morefloats\relax%
\xdef\morefloats@more{\the\@tempcnta}%

\ifnum\@tempcnta<\z@\relax%
  \PackageError{morefloats}{Option morefloats is \the\@tempcnta<0}{%
    morefloats must be a number equal to or larger than 0\MessageBreak%
    (or not used at all).\MessageBreak%
    Now setting morefloats=0.\MessageBreak%
   }%
  \gdef\morefloats@more{0}%
\fi%

\@tempcnta=18\relax%
\advance\@tempcnta by \morefloats@more%
%    \end{macrocode}
%
% The value of |morefloats| should now be equal to the value of |morefloats@max|.
%
%    \begin{macrocode}
\advance\@tempcnta by -\morefloats@max%
%    \end{macrocode}
%
% Therefore |\@tempcnta| should now be equal to zero.
%
%    \begin{macrocode}
\xdef\morefloats@mx{\the\@tempcnta}%
\ifnum\morefloats@mx=\z@\relax%
  \@tempcnta=\morefloats@maxfloats\relax%
\else%
  \PackageError{morefloats}{%
    Clash between options maxfloats and morefloats}{%
    Option maxfloats must be empty\MessageBreak%
    or the sum of 18 and option value morefloats,\MessageBreak%
    but it is maxfloats=\morefloats@maxfloats\space and %
    morefloats=\morefloats@morefloats .\MessageBreak%
    }%
%    \end{macrocode}
%
% We choose the larger value to be used.
%
%    \begin{macrocode}
  \ifnum\@tempcnta<\z@% \morefloats@max > \morefloats@more
    \@tempcnta=\morefloats@maxfloats\relax%
  \else% \@tempcnta>0, \morefloats@max < \morefloats@more
    \@tempcnta=18\relax%
    \advance\@tempcnta by \morefloats@morefloats%
  \fi%
\fi%
\edef\morefloats@mx{\the\@tempcnta}%
%    \end{macrocode}
%
% Maybe we had to change |\morefloats@maxfloats| or |\morefloats@maxfloats|:
%
%    \begin{macrocode}
\xdef\morefloats@maxfloats{\the\@tempcnta}%
\advance\@tempcnta by -18\relax%
\xdef\morefloats@morefloats{\the\@tempcnta}%
\gdef\morefloats@test{1}%
\ifx\morefloats@morefloats\morefloats@test\relax%
  \PackageInfo{morefloats}{%
    Maximum number of possible floats asked for: \morefloats@maxfloats%
    \MessageBreak%
    (i.e. one more float)\@gobble%
   }%
\else%
  \PackageInfo{morefloats}{%
    Maximum number of possible floats asked for: \morefloats@maxfloats%
    \MessageBreak%
    (i.e. \morefloats@morefloats\space more floats).\MessageBreak%
    LaTeX might run out of memory before this\MessageBreak%
    (in which case it will notify you)%
    \@gobble%
   }%
\fi%


%    \end{macrocode}
%
% The \LaTeX{} 2015 \textquotedblleft release provides a new command in the format
% |\extrafloats| which does a similar job [as earlier versions of this package did],
% although as it doesn't use |\newinsert| (and as the 2015 format uses extended
% registers by default) you can allocate a lot more floats,\textquotedblright{} %
% \hbox{e.\,g. |\extrafloats{1234}|.} Loading the \xpackage{etex} package and
% \xpackage{morefloats} with the new format would
% \textquotedblleft over{-}write the new allocation mechanism and end up with
% fewer floats available.\textquotedblright{} Therefore here it is tested
% \textquotedblleft for the new format and switch[ed] to the new mechanism
% in that case, so that existing documents work as before but using the new allocation
% scheme underneath.\textquotedblright{} (all \textsc{David Carlisle}, 29. June 2015,
% who provided also main parts of the following code)
%
%    \begin{macrocode}
%% Test for new mechanism in LaTeX 2015:
\ifx\e@alloc\@undefined\relax%
  %% This is an old LaTeX format, \extrafloats is not available.
  \PackageWarning{morefloats}{%
    \fmtname\space <\fmtversion> %
    \ifx\patch@level\@undefined\relax%
    \else patch level \patch@level%
    \fi%
    \MessageBreak%
    found. At least\MessageBreak%
    LaTeX2e <2015/01/01> patch level 2\MessageBreak%
    is now available\MessageBreak%
    and can handle even more floats%
    \@gobble%
   }%
\else%
  %% This is new in LaTeX 2015, \extrafloats is available.
  \@ifpackageloaded{etex}%
  {%% etex package loaded:
   %% "it overwrites all the new allocation system
   %% so really \extrafloats shouldn't be expected to work"
   %% (D. Carlisle, 2015/07/16, who also provided the following
   %% \extrafloats redefinition).
   \gdef\extrafloats#1{%
     \ifnum#1>\z@\relax%
       \count@\numexpr\float@count-1\relax%
       \ch@ck0\count@\count\relax%
       \ch@ck1\count@\dimen\relax%
       \ch@ck2\count@\skip\relax%
       \ch@ck4\count@\box\relax%
       \e@alloc@chardef\float@count\count@%
       \expandafter\e@alloc@chardef\csname bx@\the\float@count\endcsname\float@count%
       \@cons\@freelist{\csname bx@\the\float@count\endcsname}%
       \expandafter%
       \extrafloats\expandafter{\numexpr#1-1\relax}%
     \fi%
   }%
  }{% etex package not loaded
   }%
  \extrafloats{\morefloats@morefloats}%
  % The part after the test is no longer needed and therefore not loaded:
  \expandafter\endinput%
\fi%
%% End of the test for LaTeX 2015 (or newer).
%% Not new format, otherwise the last \endinput would have been applied.

%% Test for e-TeX:
\RequirePackage{ifetex}[2011/12/15]% v1.2
\ifetex%
  %% then we can use code similar to the one from David Carlisle,
  %% https://tex.stackexchange.com/a/212483/6865
  \mathchardef\float@count=32767\relax%
  \gdef\extrafloats#1{%
    \ifnum#1>\z@\relax%
      \count@\numexpr\float@count-1\relax%
      \ch@ck0\count@\count\relax%
      \ch@ck1\count@\dimen\relax%
      \ch@ck2\count@\skip\relax%
      \ch@ck4\count@\box\relax%
      \mathchardef\float@count\count@\relax%
      \expandafter\mathchardef\csname bx@\the\float@count\endcsname\float@count%
      \@cons\@freelist{\csname bx@\the\float@count\endcsname}%
      \expandafter%
      \extrafloats\expandafter{\numexpr#1-1\relax}%
    \fi}%
  \extrafloats{\morefloats@morefloats}%
  \expandafter\endinput%
\fi%
%% End of the test for e-TeX.
%% Old format and not e-TeX,
%% otherwise the last \endinput would have been applied.


%    \end{macrocode}
%
% If we ever come to this place, \textquotedblleft everything\textquotedblright{} %
% failed and we need to do things the old fashioned way,
% which severely limits the maximum number of additionally available floats.
%
%    \begin{macrocode}
\PackageWarning{morefloats}{%
  e-TeX is not available here\MessageBreak%
  but it is available in almost all\MessageBreak%
  recent TeX distributions.\MessageBreak%
  Maybe consider updating to one of those%
  \@gobble%
 }%

%    \end{macrocode}
%
% \newpage
%
% \begin{landscape}
%
% |Float| uses |insert|, and each |insert| use a group of |count|, |dimen|, |skip|,
% and |box| each. When there are not enough available, no |\newinsert| can be created.
%
%    \begin{macrocode}
%% Code similar to the one from Heiko Oberdiek,
%% http://permalink.gmane.org/gmane.comp.tex.latex.latex3/2159
                           \@tempcnta=\the\count10 \relax \def\maxfloats@vln{count}    %
\ifnum \count11>\@tempcnta \@tempcnta=\the\count11 \relax \def\maxfloats@vln{dimen} \fi%
\ifnum \count12>\@tempcnta \@tempcnta=\the\count12 \relax \def\maxfloats@vln{skip}  \fi%
\ifnum \count14>\@tempcnta \@tempcnta=\the\count14 \relax \def\maxfloats@vln{box}   \fi%
%% end similar
\@tempcntb=234\relax%
\advance\@tempcntb by -\@tempcnta\relax%
\@tempcnta=\@tempcntb\relax%
\ifnum\morefloats@mx>\@tempcntb\relax%
  \PackageError{morefloats}{Too many floats requested}{%
    Maximum number of possible floats asked for: \morefloats@mx .\MessageBreak%
    There are only \the\@tempcnta\space \maxfloats@vln\space left,\MessageBreak%
    therefore only \the\@tempcntb\space floats will be possible.\MessageBreak%
    Load the morefloats package earlier and/or\MessageBreak%
    reduce the number of used \maxfloats@vln\space registers\MessageBreak%
    to have more floats available!\MessageBreak%
   }%
  \xdef\morefloats@mx{\the\@tempcntb}%
\fi%

%    \end{macrocode}
%
% The task at hand is to increase \LaTeX{}'s default limit of $18$~unprocessed
% floats in memory at once to |maxfloats|.
% An examination of \texttt{latex.tex} reveals that this is accomplished
% by allocating~(!) an insert register for each unprocessed float. A~quick
% check of (the obsolete, now \texttt{ltplain}, update to \LaTeX2e{}!)
% \texttt{lplain.lis} reveals that there is room, in fact, for up to
% $256$ unprocessed floats, but \TeX{}'s main memory could be exhausted
% well before that happened.\\
%
% \LaTeX2e{} uses a |\dimen| for each |\newinsert|, and the number of |\dimen|s
% is also restricted. Therefore only use the number of floats you need!
% To check the number of used registers, you could use the \xpackage{regstats}
% and/or \xpackage{regcount} packages (see subsection~\ref{ss:Downloads}).
%
% \bigskip
%
% \DescribeMacro{Allocating insert registers}
% \DescribeMacro{@freelist}
% \DescribeMacro{@elt}
% \DescribeMacro{newinsert}
% First we allocate the additional insert registers needed.\\
%
% That accomplished, the next step is to define the macro |\@freelist|,
% which is merely a~list of the box registers each preceded by |\@elt|.
% This approach allows processing of the list to be done far more efficiently.
% A similar approach is used by \textsc{Mittelbach \& Sch\"{o}pf}'s \texttt{doc.sty} to
% keep track of control sequences, which should not be indexed.\\
% First for the 18 default \LaTeX{} boxes.\\
% \noindent |\ifnum maxfloats <= 18|, \LaTeX{} already allocated the insert registers. |\fi|\\
%
%    \begin{macrocode}
\global\long\def\@freelist{\@elt\bx@A\@elt\bx@B\@elt\bx@C\@elt\bx@D\@elt\bx@E\@elt\bx@F\@elt\bx@G\@elt\bx@H\@elt%
\bx@I\@elt\bx@J\@elt\bx@K\@elt\bx@L\@elt\bx@M\@elt\bx@N\@elt\bx@O\@elt\bx@P\@elt\bx@Q\@elt\bx@R}

%    \end{macrocode}
%
% Now we need to add |\@elt\bx@...| depending on the number of |morefloats| wanted:\\
% (\textsc{Karl Berry} helped with two out of three |\expandafter|s, thanks!)
%
% \medskip
%
%    \begin{macrocode}
\ifnum \morefloats@mx> 18 \newinsert\bx@S  \expandafter\gdef\expandafter\@freelist\expandafter{\@freelist \@elt\bx@S}
\ifnum \morefloats@mx> 19 \newinsert\bx@T  \expandafter\gdef\expandafter\@freelist\expandafter{\@freelist \@elt\bx@T}
\ifnum \morefloats@mx> 20 \newinsert\bx@U  \expandafter\gdef\expandafter\@freelist\expandafter{\@freelist \@elt\bx@U}
\ifnum \morefloats@mx> 21 \newinsert\bx@V  \expandafter\gdef\expandafter\@freelist\expandafter{\@freelist \@elt\bx@V}
\ifnum \morefloats@mx> 22 \newinsert\bx@W  \expandafter\gdef\expandafter\@freelist\expandafter{\@freelist \@elt\bx@W}
\ifnum \morefloats@mx> 23 \newinsert\bx@X  \expandafter\gdef\expandafter\@freelist\expandafter{\@freelist \@elt\bx@X}
\ifnum \morefloats@mx> 24 \newinsert\bx@Y  \expandafter\gdef\expandafter\@freelist\expandafter{\@freelist \@elt\bx@Y}
\ifnum \morefloats@mx> 25 \newinsert\bx@Z  \expandafter\gdef\expandafter\@freelist\expandafter{\@freelist \@elt\bx@Z}
\ifnum \morefloats@mx> 26 \newinsert\bx@AA \expandafter\gdef\expandafter\@freelist\expandafter{\@freelist \@elt\bx@AA}
\ifnum \morefloats@mx> 27 \newinsert\bx@AB \expandafter\gdef\expandafter\@freelist\expandafter{\@freelist \@elt\bx@AB}
\ifnum \morefloats@mx> 28 \newinsert\bx@AC \expandafter\gdef\expandafter\@freelist\expandafter{\@freelist \@elt\bx@AC}
\ifnum \morefloats@mx> 29 \newinsert\bx@AD \expandafter\gdef\expandafter\@freelist\expandafter{\@freelist \@elt\bx@AD}
\ifnum \morefloats@mx> 30 \newinsert\bx@AE \expandafter\gdef\expandafter\@freelist\expandafter{\@freelist \@elt\bx@AE}
\ifnum \morefloats@mx> 31 \newinsert\bx@AF \expandafter\gdef\expandafter\@freelist\expandafter{\@freelist \@elt\bx@AF}
\ifnum \morefloats@mx> 32 \newinsert\bx@AG \expandafter\gdef\expandafter\@freelist\expandafter{\@freelist \@elt\bx@AG}
\ifnum \morefloats@mx> 33 \newinsert\bx@AH \expandafter\gdef\expandafter\@freelist\expandafter{\@freelist \@elt\bx@AH}
\ifnum \morefloats@mx> 34 \newinsert\bx@AI \expandafter\gdef\expandafter\@freelist\expandafter{\@freelist \@elt\bx@AI}
\ifnum \morefloats@mx> 35 \newinsert\bx@AJ \expandafter\gdef\expandafter\@freelist\expandafter{\@freelist \@elt\bx@AJ}
\ifnum \morefloats@mx> 36 \newinsert\bx@AK \expandafter\gdef\expandafter\@freelist\expandafter{\@freelist \@elt\bx@AK}
\ifnum \morefloats@mx> 37 \newinsert\bx@AL \expandafter\gdef\expandafter\@freelist\expandafter{\@freelist \@elt\bx@AL}
\ifnum \morefloats@mx> 38 \newinsert\bx@AM \expandafter\gdef\expandafter\@freelist\expandafter{\@freelist \@elt\bx@AM}
\ifnum \morefloats@mx> 39 \newinsert\bx@AN \expandafter\gdef\expandafter\@freelist\expandafter{\@freelist \@elt\bx@AN}
\ifnum \morefloats@mx> 40 \newinsert\bx@AO \expandafter\gdef\expandafter\@freelist\expandafter{\@freelist \@elt\bx@AO}
\ifnum \morefloats@mx> 41 \newinsert\bx@AP \expandafter\gdef\expandafter\@freelist\expandafter{\@freelist \@elt\bx@AP}
\ifnum \morefloats@mx> 42 \newinsert\bx@AQ \expandafter\gdef\expandafter\@freelist\expandafter{\@freelist \@elt\bx@AQ}
\ifnum \morefloats@mx> 43 \newinsert\bx@AR \expandafter\gdef\expandafter\@freelist\expandafter{\@freelist \@elt\bx@AR}
\ifnum \morefloats@mx> 44 \newinsert\bx@AS \expandafter\gdef\expandafter\@freelist\expandafter{\@freelist \@elt\bx@AS}
\ifnum \morefloats@mx> 45 \newinsert\bx@AT \expandafter\gdef\expandafter\@freelist\expandafter{\@freelist \@elt\bx@AT}
\ifnum \morefloats@mx> 46 \newinsert\bx@AU \expandafter\gdef\expandafter\@freelist\expandafter{\@freelist \@elt\bx@AU}
\ifnum \morefloats@mx> 47 \newinsert\bx@AV \expandafter\gdef\expandafter\@freelist\expandafter{\@freelist \@elt\bx@AV}
\ifnum \morefloats@mx> 48 \newinsert\bx@AW \expandafter\gdef\expandafter\@freelist\expandafter{\@freelist \@elt\bx@AW}
\ifnum \morefloats@mx> 49 \newinsert\bx@AX \expandafter\gdef\expandafter\@freelist\expandafter{\@freelist \@elt\bx@AX}
\ifnum \morefloats@mx> 50 \newinsert\bx@AY \expandafter\gdef\expandafter\@freelist\expandafter{\@freelist \@elt\bx@AY}
\ifnum \morefloats@mx> 51 \newinsert\bx@AZ \expandafter\gdef\expandafter\@freelist\expandafter{\@freelist \@elt\bx@AZ}
\ifnum \morefloats@mx> 52 \newinsert\bx@BA \expandafter\gdef\expandafter\@freelist\expandafter{\@freelist \@elt\bx@BA}
\ifnum \morefloats@mx> 53 \newinsert\bx@BB \expandafter\gdef\expandafter\@freelist\expandafter{\@freelist \@elt\bx@BB}
\ifnum \morefloats@mx> 54 \newinsert\bx@BC \expandafter\gdef\expandafter\@freelist\expandafter{\@freelist \@elt\bx@BC}
\ifnum \morefloats@mx> 55 \newinsert\bx@BD \expandafter\gdef\expandafter\@freelist\expandafter{\@freelist \@elt\bx@BD}
\ifnum \morefloats@mx> 56 \newinsert\bx@BE \expandafter\gdef\expandafter\@freelist\expandafter{\@freelist \@elt\bx@BE}
\ifnum \morefloats@mx> 57 \newinsert\bx@BF \expandafter\gdef\expandafter\@freelist\expandafter{\@freelist \@elt\bx@BF}
\ifnum \morefloats@mx> 58 \newinsert\bx@BG \expandafter\gdef\expandafter\@freelist\expandafter{\@freelist \@elt\bx@BG}
\ifnum \morefloats@mx> 59 \newinsert\bx@BH \expandafter\gdef\expandafter\@freelist\expandafter{\@freelist \@elt\bx@BH}
\ifnum \morefloats@mx> 60 \newinsert\bx@BI \expandafter\gdef\expandafter\@freelist\expandafter{\@freelist \@elt\bx@BI}
\ifnum \morefloats@mx> 61 \newinsert\bx@BJ \expandafter\gdef\expandafter\@freelist\expandafter{\@freelist \@elt\bx@BJ}
\ifnum \morefloats@mx> 62 \newinsert\bx@BK \expandafter\gdef\expandafter\@freelist\expandafter{\@freelist \@elt\bx@BK}
\ifnum \morefloats@mx> 63 \newinsert\bx@BL \expandafter\gdef\expandafter\@freelist\expandafter{\@freelist \@elt\bx@BL}
\ifnum \morefloats@mx> 64 \newinsert\bx@BM \expandafter\gdef\expandafter\@freelist\expandafter{\@freelist \@elt\bx@BM}
\ifnum \morefloats@mx> 65 \newinsert\bx@BN \expandafter\gdef\expandafter\@freelist\expandafter{\@freelist \@elt\bx@BN}
\ifnum \morefloats@mx> 66 \newinsert\bx@BO \expandafter\gdef\expandafter\@freelist\expandafter{\@freelist \@elt\bx@BO}
\ifnum \morefloats@mx> 67 \newinsert\bx@BP \expandafter\gdef\expandafter\@freelist\expandafter{\@freelist \@elt\bx@BP}
\ifnum \morefloats@mx> 68 \newinsert\bx@BQ \expandafter\gdef\expandafter\@freelist\expandafter{\@freelist \@elt\bx@BQ}
\ifnum \morefloats@mx> 69 \newinsert\bx@BR \expandafter\gdef\expandafter\@freelist\expandafter{\@freelist \@elt\bx@BR}
\ifnum \morefloats@mx> 70 \newinsert\bx@BS \expandafter\gdef\expandafter\@freelist\expandafter{\@freelist \@elt\bx@BS}
\ifnum \morefloats@mx> 71 \newinsert\bx@BT \expandafter\gdef\expandafter\@freelist\expandafter{\@freelist \@elt\bx@BT}
\ifnum \morefloats@mx> 72 \newinsert\bx@BU \expandafter\gdef\expandafter\@freelist\expandafter{\@freelist \@elt\bx@BU}
\ifnum \morefloats@mx> 73 \newinsert\bx@BV \expandafter\gdef\expandafter\@freelist\expandafter{\@freelist \@elt\bx@BV}
\ifnum \morefloats@mx> 74 \newinsert\bx@BW \expandafter\gdef\expandafter\@freelist\expandafter{\@freelist \@elt\bx@BW}
\ifnum \morefloats@mx> 75 \newinsert\bx@BX \expandafter\gdef\expandafter\@freelist\expandafter{\@freelist \@elt\bx@BX}
\ifnum \morefloats@mx> 76 \newinsert\bx@BY \expandafter\gdef\expandafter\@freelist\expandafter{\@freelist \@elt\bx@BY}
\ifnum \morefloats@mx> 77 \newinsert\bx@BZ \expandafter\gdef\expandafter\@freelist\expandafter{\@freelist \@elt\bx@BZ}
\ifnum \morefloats@mx> 78 \newinsert\bx@CA \expandafter\gdef\expandafter\@freelist\expandafter{\@freelist \@elt\bx@CA}
\ifnum \morefloats@mx> 79 \newinsert\bx@CB \expandafter\gdef\expandafter\@freelist\expandafter{\@freelist \@elt\bx@CB}
\ifnum \morefloats@mx> 80 \newinsert\bx@CC \expandafter\gdef\expandafter\@freelist\expandafter{\@freelist \@elt\bx@CC}
\ifnum \morefloats@mx> 81 \newinsert\bx@CD \expandafter\gdef\expandafter\@freelist\expandafter{\@freelist \@elt\bx@CD}
\ifnum \morefloats@mx> 82 \newinsert\bx@CE \expandafter\gdef\expandafter\@freelist\expandafter{\@freelist \@elt\bx@CE}
\ifnum \morefloats@mx> 83 \newinsert\bx@CF \expandafter\gdef\expandafter\@freelist\expandafter{\@freelist \@elt\bx@CF}
\ifnum \morefloats@mx> 84 \newinsert\bx@CG \expandafter\gdef\expandafter\@freelist\expandafter{\@freelist \@elt\bx@CG}
\ifnum \morefloats@mx> 85 \newinsert\bx@CH \expandafter\gdef\expandafter\@freelist\expandafter{\@freelist \@elt\bx@CH}
\ifnum \morefloats@mx> 86 \newinsert\bx@CI \expandafter\gdef\expandafter\@freelist\expandafter{\@freelist \@elt\bx@CI}
\ifnum \morefloats@mx> 87 \newinsert\bx@CJ \expandafter\gdef\expandafter\@freelist\expandafter{\@freelist \@elt\bx@CJ}
\ifnum \morefloats@mx> 88 \newinsert\bx@CK \expandafter\gdef\expandafter\@freelist\expandafter{\@freelist \@elt\bx@CK}
\ifnum \morefloats@mx> 89 \newinsert\bx@CL \expandafter\gdef\expandafter\@freelist\expandafter{\@freelist \@elt\bx@CL}
\ifnum \morefloats@mx> 90 \newinsert\bx@CM \expandafter\gdef\expandafter\@freelist\expandafter{\@freelist \@elt\bx@CM}
\ifnum \morefloats@mx> 91 \newinsert\bx@CN \expandafter\gdef\expandafter\@freelist\expandafter{\@freelist \@elt\bx@CN}
\ifnum \morefloats@mx> 92 \newinsert\bx@CO \expandafter\gdef\expandafter\@freelist\expandafter{\@freelist \@elt\bx@CO}
\ifnum \morefloats@mx> 93 \newinsert\bx@CP \expandafter\gdef\expandafter\@freelist\expandafter{\@freelist \@elt\bx@CP}
\ifnum \morefloats@mx> 94 \newinsert\bx@CQ \expandafter\gdef\expandafter\@freelist\expandafter{\@freelist \@elt\bx@CQ}
\ifnum \morefloats@mx> 95 \newinsert\bx@CR \expandafter\gdef\expandafter\@freelist\expandafter{\@freelist \@elt\bx@CR}
\ifnum \morefloats@mx> 96 \newinsert\bx@CS \expandafter\gdef\expandafter\@freelist\expandafter{\@freelist \@elt\bx@CS}
\ifnum \morefloats@mx> 97 \newinsert\bx@CT \expandafter\gdef\expandafter\@freelist\expandafter{\@freelist \@elt\bx@CT}
\ifnum \morefloats@mx> 98 \newinsert\bx@CU \expandafter\gdef\expandafter\@freelist\expandafter{\@freelist \@elt\bx@CU}
\ifnum \morefloats@mx> 99 \newinsert\bx@CV \expandafter\gdef\expandafter\@freelist\expandafter{\@freelist \@elt\bx@CV}
\ifnum \morefloats@mx>100 \newinsert\bx@CW \expandafter\gdef\expandafter\@freelist\expandafter{\@freelist \@elt\bx@CW}
\ifnum \morefloats@mx>101 \newinsert\bx@CX \expandafter\gdef\expandafter\@freelist\expandafter{\@freelist \@elt\bx@CX}
\ifnum \morefloats@mx>102 \newinsert\bx@CY \expandafter\gdef\expandafter\@freelist\expandafter{\@freelist \@elt\bx@CY}
\ifnum \morefloats@mx>103 \newinsert\bx@CZ \expandafter\gdef\expandafter\@freelist\expandafter{\@freelist \@elt\bx@CZ}
\ifnum \morefloats@mx>104 \newinsert\bx@DA \expandafter\gdef\expandafter\@freelist\expandafter{\@freelist \@elt\bx@DA}
\ifnum \morefloats@mx>105 \newinsert\bx@DB \expandafter\gdef\expandafter\@freelist\expandafter{\@freelist \@elt\bx@DB}
\ifnum \morefloats@mx>106 \newinsert\bx@DC \expandafter\gdef\expandafter\@freelist\expandafter{\@freelist \@elt\bx@DC}
\ifnum \morefloats@mx>107 \newinsert\bx@DD \expandafter\gdef\expandafter\@freelist\expandafter{\@freelist \@elt\bx@DD}
\ifnum \morefloats@mx>108 \newinsert\bx@DE \expandafter\gdef\expandafter\@freelist\expandafter{\@freelist \@elt\bx@DE}
\ifnum \morefloats@mx>109 \newinsert\bx@DF \expandafter\gdef\expandafter\@freelist\expandafter{\@freelist \@elt\bx@DF}
\ifnum \morefloats@mx>110 \newinsert\bx@DG \expandafter\gdef\expandafter\@freelist\expandafter{\@freelist \@elt\bx@DG}
\ifnum \morefloats@mx>111 \newinsert\bx@DH \expandafter\gdef\expandafter\@freelist\expandafter{\@freelist \@elt\bx@DH}
\ifnum \morefloats@mx>112 \newinsert\bx@DI \expandafter\gdef\expandafter\@freelist\expandafter{\@freelist \@elt\bx@DI}
\ifnum \morefloats@mx>113 \newinsert\bx@DJ \expandafter\gdef\expandafter\@freelist\expandafter{\@freelist \@elt\bx@DJ}
\ifnum \morefloats@mx>114 \newinsert\bx@DK \expandafter\gdef\expandafter\@freelist\expandafter{\@freelist \@elt\bx@DK}
\ifnum \morefloats@mx>115 \newinsert\bx@DL \expandafter\gdef\expandafter\@freelist\expandafter{\@freelist \@elt\bx@DL}
\ifnum \morefloats@mx>116 \newinsert\bx@DM \expandafter\gdef\expandafter\@freelist\expandafter{\@freelist \@elt\bx@DM}
\ifnum \morefloats@mx>117 \newinsert\bx@DN \expandafter\gdef\expandafter\@freelist\expandafter{\@freelist \@elt\bx@DN}
\ifnum \morefloats@mx>118 \newinsert\bx@DO \expandafter\gdef\expandafter\@freelist\expandafter{\@freelist \@elt\bx@DO}
\ifnum \morefloats@mx>119 \newinsert\bx@DP \expandafter\gdef\expandafter\@freelist\expandafter{\@freelist \@elt\bx@DP}
\ifnum \morefloats@mx>120 \newinsert\bx@DQ \expandafter\gdef\expandafter\@freelist\expandafter{\@freelist \@elt\bx@DQ}
\ifnum \morefloats@mx>121 \newinsert\bx@DR \expandafter\gdef\expandafter\@freelist\expandafter{\@freelist \@elt\bx@DR}
\ifnum \morefloats@mx>122 \newinsert\bx@DS \expandafter\gdef\expandafter\@freelist\expandafter{\@freelist \@elt\bx@DS}
\ifnum \morefloats@mx>123 \newinsert\bx@DT \expandafter\gdef\expandafter\@freelist\expandafter{\@freelist \@elt\bx@DT}
\ifnum \morefloats@mx>124 \newinsert\bx@DU \expandafter\gdef\expandafter\@freelist\expandafter{\@freelist \@elt\bx@DU}
\ifnum \morefloats@mx>125 \newinsert\bx@DV \expandafter\gdef\expandafter\@freelist\expandafter{\@freelist \@elt\bx@DV}
\ifnum \morefloats@mx>126 \newinsert\bx@DW \expandafter\gdef\expandafter\@freelist\expandafter{\@freelist \@elt\bx@DW}
\ifnum \morefloats@mx>127 \newinsert\bx@DX \expandafter\gdef\expandafter\@freelist\expandafter{\@freelist \@elt\bx@DX}
\ifnum \morefloats@mx>128 \newinsert\bx@DY \expandafter\gdef\expandafter\@freelist\expandafter{\@freelist \@elt\bx@DY}
\ifnum \morefloats@mx>129 \newinsert\bx@DZ \expandafter\gdef\expandafter\@freelist\expandafter{\@freelist \@elt\bx@DZ}
\ifnum \morefloats@mx>130 \newinsert\bx@EA \expandafter\gdef\expandafter\@freelist\expandafter{\@freelist \@elt\bx@EA}
\ifnum \morefloats@mx>131 \newinsert\bx@EB \expandafter\gdef\expandafter\@freelist\expandafter{\@freelist \@elt\bx@EB}
\ifnum \morefloats@mx>132 \newinsert\bx@EC \expandafter\gdef\expandafter\@freelist\expandafter{\@freelist \@elt\bx@EC}
\ifnum \morefloats@mx>133 \newinsert\bx@ED \expandafter\gdef\expandafter\@freelist\expandafter{\@freelist \@elt\bx@ED}
\ifnum \morefloats@mx>134 \newinsert\bx@EE \expandafter\gdef\expandafter\@freelist\expandafter{\@freelist \@elt\bx@EE}
\ifnum \morefloats@mx>135 \newinsert\bx@EF \expandafter\gdef\expandafter\@freelist\expandafter{\@freelist \@elt\bx@EF}
\ifnum \morefloats@mx>136 \newinsert\bx@EG \expandafter\gdef\expandafter\@freelist\expandafter{\@freelist \@elt\bx@EG}
\ifnum \morefloats@mx>137 \newinsert\bx@EH \expandafter\gdef\expandafter\@freelist\expandafter{\@freelist \@elt\bx@EH}
\ifnum \morefloats@mx>138 \newinsert\bx@EI \expandafter\gdef\expandafter\@freelist\expandafter{\@freelist \@elt\bx@EI}
\ifnum \morefloats@mx>139 \newinsert\bx@EJ \expandafter\gdef\expandafter\@freelist\expandafter{\@freelist \@elt\bx@EJ}
\ifnum \morefloats@mx>140 \newinsert\bx@EK \expandafter\gdef\expandafter\@freelist\expandafter{\@freelist \@elt\bx@EK}
\ifnum \morefloats@mx>141 \newinsert\bx@EL \expandafter\gdef\expandafter\@freelist\expandafter{\@freelist \@elt\bx@EL}
\ifnum \morefloats@mx>142 \newinsert\bx@EM \expandafter\gdef\expandafter\@freelist\expandafter{\@freelist \@elt\bx@EM}
\ifnum \morefloats@mx>143 \newinsert\bx@EN \expandafter\gdef\expandafter\@freelist\expandafter{\@freelist \@elt\bx@EN}
\ifnum \morefloats@mx>144 \newinsert\bx@EO \expandafter\gdef\expandafter\@freelist\expandafter{\@freelist \@elt\bx@EO}
\ifnum \morefloats@mx>145 \newinsert\bx@EP \expandafter\gdef\expandafter\@freelist\expandafter{\@freelist \@elt\bx@EP}
\ifnum \morefloats@mx>146 \newinsert\bx@EQ \expandafter\gdef\expandafter\@freelist\expandafter{\@freelist \@elt\bx@EQ}
\ifnum \morefloats@mx>147 \newinsert\bx@ER \expandafter\gdef\expandafter\@freelist\expandafter{\@freelist \@elt\bx@ER}
\ifnum \morefloats@mx>148 \newinsert\bx@ES \expandafter\gdef\expandafter\@freelist\expandafter{\@freelist \@elt\bx@ES}
\ifnum \morefloats@mx>149 \newinsert\bx@ET \expandafter\gdef\expandafter\@freelist\expandafter{\@freelist \@elt\bx@ET}
\ifnum \morefloats@mx>150 \newinsert\bx@EU \expandafter\gdef\expandafter\@freelist\expandafter{\@freelist \@elt\bx@EU}
\ifnum \morefloats@mx>151 \newinsert\bx@EV \expandafter\gdef\expandafter\@freelist\expandafter{\@freelist \@elt\bx@EV}
\ifnum \morefloats@mx>152 \newinsert\bx@EW \expandafter\gdef\expandafter\@freelist\expandafter{\@freelist \@elt\bx@EW}
\ifnum \morefloats@mx>153 \newinsert\bx@EX \expandafter\gdef\expandafter\@freelist\expandafter{\@freelist \@elt\bx@EX}
\ifnum \morefloats@mx>154 \newinsert\bx@EY \expandafter\gdef\expandafter\@freelist\expandafter{\@freelist \@elt\bx@EY}
\ifnum \morefloats@mx>155 \newinsert\bx@EZ \expandafter\gdef\expandafter\@freelist\expandafter{\@freelist \@elt\bx@EZ}
\ifnum \morefloats@mx>156 \newinsert\bx@FA \expandafter\gdef\expandafter\@freelist\expandafter{\@freelist \@elt\bx@FA}
\ifnum \morefloats@mx>157 \newinsert\bx@FB \expandafter\gdef\expandafter\@freelist\expandafter{\@freelist \@elt\bx@FB}
\ifnum \morefloats@mx>158 \newinsert\bx@FC \expandafter\gdef\expandafter\@freelist\expandafter{\@freelist \@elt\bx@FC}
\ifnum \morefloats@mx>159 \newinsert\bx@FD \expandafter\gdef\expandafter\@freelist\expandafter{\@freelist \@elt\bx@FD}
\ifnum \morefloats@mx>160 \newinsert\bx@FE \expandafter\gdef\expandafter\@freelist\expandafter{\@freelist \@elt\bx@FE}
\ifnum \morefloats@mx>161 \newinsert\bx@FF \expandafter\gdef\expandafter\@freelist\expandafter{\@freelist \@elt\bx@FF}
\ifnum \morefloats@mx>162 \newinsert\bx@FG \expandafter\gdef\expandafter\@freelist\expandafter{\@freelist \@elt\bx@FG}
\ifnum \morefloats@mx>163 \newinsert\bx@FH \expandafter\gdef\expandafter\@freelist\expandafter{\@freelist \@elt\bx@FH}
\ifnum \morefloats@mx>164 \newinsert\bx@FI \expandafter\gdef\expandafter\@freelist\expandafter{\@freelist \@elt\bx@FI}
\ifnum \morefloats@mx>165 \newinsert\bx@FJ \expandafter\gdef\expandafter\@freelist\expandafter{\@freelist \@elt\bx@FJ}
\ifnum \morefloats@mx>166 \newinsert\bx@FK \expandafter\gdef\expandafter\@freelist\expandafter{\@freelist \@elt\bx@FK}
\ifnum \morefloats@mx>167 \newinsert\bx@FL \expandafter\gdef\expandafter\@freelist\expandafter{\@freelist \@elt\bx@FL}
\ifnum \morefloats@mx>168 \newinsert\bx@FM \expandafter\gdef\expandafter\@freelist\expandafter{\@freelist \@elt\bx@FM}
\ifnum \morefloats@mx>169 \newinsert\bx@FN \expandafter\gdef\expandafter\@freelist\expandafter{\@freelist \@elt\bx@FN}
\ifnum \morefloats@mx>170 \newinsert\bx@FO \expandafter\gdef\expandafter\@freelist\expandafter{\@freelist \@elt\bx@FO}
\ifnum \morefloats@mx>171 \newinsert\bx@FP \expandafter\gdef\expandafter\@freelist\expandafter{\@freelist \@elt\bx@FP}
\ifnum \morefloats@mx>172 \newinsert\bx@FQ \expandafter\gdef\expandafter\@freelist\expandafter{\@freelist \@elt\bx@FQ}
\ifnum \morefloats@mx>173 \newinsert\bx@FR \expandafter\gdef\expandafter\@freelist\expandafter{\@freelist \@elt\bx@FR}
\ifnum \morefloats@mx>174 \newinsert\bx@FS \expandafter\gdef\expandafter\@freelist\expandafter{\@freelist \@elt\bx@FS}
\ifnum \morefloats@mx>175 \newinsert\bx@FT \expandafter\gdef\expandafter\@freelist\expandafter{\@freelist \@elt\bx@FT}
\ifnum \morefloats@mx>176 \newinsert\bx@FU \expandafter\gdef\expandafter\@freelist\expandafter{\@freelist \@elt\bx@FU}
\ifnum \morefloats@mx>177 \newinsert\bx@FV \expandafter\gdef\expandafter\@freelist\expandafter{\@freelist \@elt\bx@FV}
\ifnum \morefloats@mx>178 \newinsert\bx@FW \expandafter\gdef\expandafter\@freelist\expandafter{\@freelist \@elt\bx@FW}
\ifnum \morefloats@mx>179 \newinsert\bx@FX \expandafter\gdef\expandafter\@freelist\expandafter{\@freelist \@elt\bx@FX}
\ifnum \morefloats@mx>180 \newinsert\bx@FY \expandafter\gdef\expandafter\@freelist\expandafter{\@freelist \@elt\bx@FY}
\ifnum \morefloats@mx>181 \newinsert\bx@FZ \expandafter\gdef\expandafter\@freelist\expandafter{\@freelist \@elt\bx@FZ}
\ifnum \morefloats@mx>182 \newinsert\bx@GA \expandafter\gdef\expandafter\@freelist\expandafter{\@freelist \@elt\bx@GA}
\ifnum \morefloats@mx>183 \newinsert\bx@GB \expandafter\gdef\expandafter\@freelist\expandafter{\@freelist \@elt\bx@GB}
\ifnum \morefloats@mx>184 \newinsert\bx@GC \expandafter\gdef\expandafter\@freelist\expandafter{\@freelist \@elt\bx@GC}
\ifnum \morefloats@mx>185 \newinsert\bx@GD \expandafter\gdef\expandafter\@freelist\expandafter{\@freelist \@elt\bx@GD}
\ifnum \morefloats@mx>186 \newinsert\bx@GE \expandafter\gdef\expandafter\@freelist\expandafter{\@freelist \@elt\bx@GE}
\ifnum \morefloats@mx>187 \newinsert\bx@GF \expandafter\gdef\expandafter\@freelist\expandafter{\@freelist \@elt\bx@GF}
\ifnum \morefloats@mx>188 \newinsert\bx@GG \expandafter\gdef\expandafter\@freelist\expandafter{\@freelist \@elt\bx@GG}
\ifnum \morefloats@mx>189 \newinsert\bx@GH \expandafter\gdef\expandafter\@freelist\expandafter{\@freelist \@elt\bx@GH}
\ifnum \morefloats@mx>190 \newinsert\bx@GI \expandafter\gdef\expandafter\@freelist\expandafter{\@freelist \@elt\bx@GI}
\ifnum \morefloats@mx>191 \newinsert\bx@GJ \expandafter\gdef\expandafter\@freelist\expandafter{\@freelist \@elt\bx@GJ}
\ifnum \morefloats@mx>192 \newinsert\bx@GK \expandafter\gdef\expandafter\@freelist\expandafter{\@freelist \@elt\bx@GK}
\ifnum \morefloats@mx>193 \newinsert\bx@GL \expandafter\gdef\expandafter\@freelist\expandafter{\@freelist \@elt\bx@GL}
\ifnum \morefloats@mx>194 \newinsert\bx@GM \expandafter\gdef\expandafter\@freelist\expandafter{\@freelist \@elt\bx@GM}
\ifnum \morefloats@mx>195 \newinsert\bx@GN \expandafter\gdef\expandafter\@freelist\expandafter{\@freelist \@elt\bx@GN}
\ifnum \morefloats@mx>196 \newinsert\bx@GO \expandafter\gdef\expandafter\@freelist\expandafter{\@freelist \@elt\bx@GO}
\ifnum \morefloats@mx>197 \newinsert\bx@GP \expandafter\gdef\expandafter\@freelist\expandafter{\@freelist \@elt\bx@GP}
\ifnum \morefloats@mx>198 \newinsert\bx@GQ \expandafter\gdef\expandafter\@freelist\expandafter{\@freelist \@elt\bx@GQ}
\ifnum \morefloats@mx>199 \newinsert\bx@GR \expandafter\gdef\expandafter\@freelist\expandafter{\@freelist \@elt\bx@GR}
\ifnum \morefloats@mx>200 \newinsert\bx@GS \expandafter\gdef\expandafter\@freelist\expandafter{\@freelist \@elt\bx@GS}
\ifnum \morefloats@mx>201 \newinsert\bx@GT \expandafter\gdef\expandafter\@freelist\expandafter{\@freelist \@elt\bx@GT}
\ifnum \morefloats@mx>202 \newinsert\bx@GU \expandafter\gdef\expandafter\@freelist\expandafter{\@freelist \@elt\bx@GU}
\ifnum \morefloats@mx>203 \newinsert\bx@GV \expandafter\gdef\expandafter\@freelist\expandafter{\@freelist \@elt\bx@GV}
\ifnum \morefloats@mx>204 \newinsert\bx@GW \expandafter\gdef\expandafter\@freelist\expandafter{\@freelist \@elt\bx@GW}
\ifnum \morefloats@mx>205 \newinsert\bx@GX \expandafter\gdef\expandafter\@freelist\expandafter{\@freelist \@elt\bx@GX}
\ifnum \morefloats@mx>206 \newinsert\bx@GY \expandafter\gdef\expandafter\@freelist\expandafter{\@freelist \@elt\bx@GY}
\ifnum \morefloats@mx>207 \newinsert\bx@GZ \expandafter\gdef\expandafter\@freelist\expandafter{\@freelist \@elt\bx@GZ}
\ifnum \morefloats@mx>208 \newinsert\bx@HA \expandafter\gdef\expandafter\@freelist\expandafter{\@freelist \@elt\bx@HA}
\ifnum \morefloats@mx>209 \newinsert\bx@HB \expandafter\gdef\expandafter\@freelist\expandafter{\@freelist \@elt\bx@HB}
\ifnum \morefloats@mx>210 \newinsert\bx@HC \expandafter\gdef\expandafter\@freelist\expandafter{\@freelist \@elt\bx@HC}
\ifnum \morefloats@mx>211 \newinsert\bx@HD \expandafter\gdef\expandafter\@freelist\expandafter{\@freelist \@elt\bx@HD}
\ifnum \morefloats@mx>212 \newinsert\bx@HE \expandafter\gdef\expandafter\@freelist\expandafter{\@freelist \@elt\bx@HE}
\ifnum \morefloats@mx>213 \newinsert\bx@HF \expandafter\gdef\expandafter\@freelist\expandafter{\@freelist \@elt\bx@HF}
\ifnum \morefloats@mx>214 \newinsert\bx@HG \expandafter\gdef\expandafter\@freelist\expandafter{\@freelist \@elt\bx@HG}
\ifnum \morefloats@mx>215 \newinsert\bx@HH \expandafter\gdef\expandafter\@freelist\expandafter{\@freelist \@elt\bx@HH}
\ifnum \morefloats@mx>216 \newinsert\bx@HI \expandafter\gdef\expandafter\@freelist\expandafter{\@freelist \@elt\bx@HI}
\ifnum \morefloats@mx>217 \newinsert\bx@HJ \expandafter\gdef\expandafter\@freelist\expandafter{\@freelist \@elt\bx@HJ}
\ifnum \morefloats@mx>218 \newinsert\bx@HK \expandafter\gdef\expandafter\@freelist\expandafter{\@freelist \@elt\bx@HK}
\ifnum \morefloats@mx>219 \newinsert\bx@HL \expandafter\gdef\expandafter\@freelist\expandafter{\@freelist \@elt\bx@HL}
\ifnum \morefloats@mx>220 \newinsert\bx@HM \expandafter\gdef\expandafter\@freelist\expandafter{\@freelist \@elt\bx@HM}
\ifnum \morefloats@mx>221 \newinsert\bx@HN \expandafter\gdef\expandafter\@freelist\expandafter{\@freelist \@elt\bx@HN}
\ifnum \morefloats@mx>222 \newinsert\bx@HO \expandafter\gdef\expandafter\@freelist\expandafter{\@freelist \@elt\bx@HO}
\ifnum \morefloats@mx>223 \newinsert\bx@HP \expandafter\gdef\expandafter\@freelist\expandafter{\@freelist \@elt\bx@HP}
\ifnum \morefloats@mx>224 \newinsert\bx@HQ \expandafter\gdef\expandafter\@freelist\expandafter{\@freelist \@elt\bx@HQ}
\ifnum \morefloats@mx>225 \newinsert\bx@HR \expandafter\gdef\expandafter\@freelist\expandafter{\@freelist \@elt\bx@HR}
\ifnum \morefloats@mx>226 \newinsert\bx@HS \expandafter\gdef\expandafter\@freelist\expandafter{\@freelist \@elt\bx@HS}
\ifnum \morefloats@mx>227 \newinsert\bx@HT \expandafter\gdef\expandafter\@freelist\expandafter{\@freelist \@elt\bx@HT}
\ifnum \morefloats@mx>228 \newinsert\bx@HU \expandafter\gdef\expandafter\@freelist\expandafter{\@freelist \@elt\bx@HU}
\ifnum \morefloats@mx>229 \newinsert\bx@HV \expandafter\gdef\expandafter\@freelist\expandafter{\@freelist \@elt\bx@HV}
\ifnum \morefloats@mx>230 \newinsert\bx@HW \expandafter\gdef\expandafter\@freelist\expandafter{\@freelist \@elt\bx@HW}
\ifnum \morefloats@mx>231 \newinsert\bx@HX \expandafter\gdef\expandafter\@freelist\expandafter{\@freelist \@elt\bx@HX}
\ifnum \morefloats@mx>232 \newinsert\bx@HY \expandafter\gdef\expandafter\@freelist\expandafter{\@freelist \@elt\bx@HY}
\ifnum \morefloats@mx>233 \newinsert\bx@HZ \expandafter\gdef\expandafter\@freelist\expandafter{\@freelist \@elt\bx@HZ}
\ifnum \morefloats@mx>234 \newinsert\bx@IA \expandafter\gdef\expandafter\@freelist\expandafter{\@freelist \@elt\bx@IA}
\ifnum \morefloats@mx>235 \newinsert\bx@IB \expandafter\gdef\expandafter\@freelist\expandafter{\@freelist \@elt\bx@IB}
\ifnum \morefloats@mx>236 \newinsert\bx@IC \expandafter\gdef\expandafter\@freelist\expandafter{\@freelist \@elt\bx@IC}
\ifnum \morefloats@mx>237 \newinsert\bx@ID \expandafter\gdef\expandafter\@freelist\expandafter{\@freelist \@elt\bx@ID}
\ifnum \morefloats@mx>238 \newinsert\bx@IE \expandafter\gdef\expandafter\@freelist\expandafter{\@freelist \@elt\bx@IE}
\ifnum \morefloats@mx>239 \newinsert\bx@IF \expandafter\gdef\expandafter\@freelist\expandafter{\@freelist \@elt\bx@IF}
\ifnum \morefloats@mx>240 \newinsert\bx@IG \expandafter\gdef\expandafter\@freelist\expandafter{\@freelist \@elt\bx@IG}
\ifnum \morefloats@mx>241 \newinsert\bx@IH \expandafter\gdef\expandafter\@freelist\expandafter{\@freelist \@elt\bx@IH}
\ifnum \morefloats@mx>242 \newinsert\bx@II \expandafter\gdef\expandafter\@freelist\expandafter{\@freelist \@elt\bx@II}
\ifnum \morefloats@mx>243 \newinsert\bx@IJ \expandafter\gdef\expandafter\@freelist\expandafter{\@freelist \@elt\bx@IJ}
\ifnum \morefloats@mx>244 \newinsert\bx@IK \expandafter\gdef\expandafter\@freelist\expandafter{\@freelist \@elt\bx@IK}
\ifnum \morefloats@mx>245 \newinsert\bx@IL \expandafter\gdef\expandafter\@freelist\expandafter{\@freelist \@elt\bx@IL}
\ifnum \morefloats@mx>246 \newinsert\bx@IM \expandafter\gdef\expandafter\@freelist\expandafter{\@freelist \@elt\bx@IM}
\ifnum \morefloats@mx>247 \newinsert\bx@IN \expandafter\gdef\expandafter\@freelist\expandafter{\@freelist \@elt\bx@IN}
\ifnum \morefloats@mx>248 \newinsert\bx@IO \expandafter\gdef\expandafter\@freelist\expandafter{\@freelist \@elt\bx@IO}
\ifnum \morefloats@mx>249 \newinsert\bx@IP \expandafter\gdef\expandafter\@freelist\expandafter{\@freelist \@elt\bx@IP}
\ifnum \morefloats@mx>250 \newinsert\bx@IQ \expandafter\gdef\expandafter\@freelist\expandafter{\@freelist \@elt\bx@IQ}
\ifnum \morefloats@mx>251 \newinsert\bx@IR \expandafter\gdef\expandafter\@freelist\expandafter{\@freelist \@elt\bx@IR}
\ifnum \morefloats@mx>252 \newinsert\bx@IS \expandafter\gdef\expandafter\@freelist\expandafter{\@freelist \@elt\bx@IS}
\ifnum \morefloats@mx>253 \newinsert\bx@IT \expandafter\gdef\expandafter\@freelist\expandafter{\@freelist \@elt\bx@IT}
\ifnum \morefloats@mx>254 \newinsert\bx@IU \expandafter\gdef\expandafter\@freelist\expandafter{\@freelist \@elt\bx@IU}
\ifnum \morefloats@mx>255 \newinsert\bx@IV \expandafter\gdef\expandafter\@freelist\expandafter{\@freelist \@elt\bx@IV}
%    \end{macrocode}
%
% \newpage
%
%    \begin{macrocode}
\ifnum \morefloats@mx>256\relax%
  \PackageError{morefloats}{Too many floats called for}{%
    You requested more than 256 floats.\MessageBreak%
    (\morefloats@mx\space to be precise.)\MessageBreak%
    LaTeX before 2015 could not process\MessageBreak%
    more than 256 floats, therefore the morefloats\MessageBreak%
    package only provides 256 floats.\MessageBreak%
    If you need more floats,\MessageBreak%
    update to a current (>=2015) LaTeX distribution.\MessageBreak%
    I expected LaTeX (prior 2015) to run out of dimensions\MessageBreak%
    or memory long before reaching 256 floats anyway.\MessageBreak%
   }%
\fi \fi \fi \fi \fi \fi \fi \fi \fi \fi \fi \fi \fi \fi \fi \fi \fi \fi
\fi \fi \fi \fi \fi \fi \fi \fi \fi \fi \fi \fi \fi \fi \fi \fi \fi \fi
\fi \fi \fi \fi \fi \fi \fi \fi \fi \fi \fi \fi \fi \fi \fi \fi \fi \fi
\fi \fi \fi \fi \fi \fi \fi \fi \fi \fi \fi \fi \fi \fi \fi \fi \fi \fi
\fi \fi \fi \fi \fi \fi \fi \fi \fi \fi \fi \fi \fi \fi \fi \fi \fi \fi
\fi \fi \fi \fi \fi \fi \fi \fi \fi \fi \fi \fi \fi \fi \fi \fi \fi \fi
\fi \fi \fi \fi \fi \fi \fi \fi \fi \fi \fi \fi \fi \fi \fi \fi \fi \fi
\fi \fi \fi \fi \fi \fi \fi \fi \fi \fi \fi \fi \fi \fi \fi \fi \fi \fi
\fi \fi \fi \fi \fi \fi \fi \fi \fi \fi \fi \fi \fi \fi \fi \fi \fi \fi
\fi \fi \fi \fi \fi \fi \fi \fi \fi \fi \fi \fi \fi \fi \fi \fi \fi \fi
\fi \fi \fi \fi \fi \fi \fi \fi \fi \fi \fi \fi \fi \fi \fi \fi \fi \fi
\fi \fi \fi \fi \fi \fi \fi \fi \fi \fi \fi \fi \fi \fi \fi \fi \fi \fi
\fi \fi \fi \fi \fi \fi \fi \fi \fi \fi \fi \fi \fi \fi \fi \fi \fi \fi
\fi \fi \fi \fi \fi

%    \end{macrocode}
%
%    \begin{macrocode}
%</package>
%    \end{macrocode}
%
% \end{landscape}
% \newpage
%
% \section{Installation}
%
% \subsection{Downloads\label{ss:Downloads}}
%
% Everything is available at \url{https://www.ctan.org},
% but may need additional packages themselves.\\
%
% \DescribeMacro{morefloats.dtx}
% For unpacking the |morefloats.dtx| file and constructing the documentation it is required:
% \begin{description}
% \item[-] \TeX Format \LaTeXe{}: \url{https://www.CTAN.org}
%
% \item[-] document class \xclass{ltxdoc}, 2015/03/26, v2.0w,
%   \url{https://www.ctan.org/pkg/ltxdoc}
%
% \item[-] package \xpackage{fontenc}, 2005/09/27, v1.99g,
%   \url{https://ctan.org/pkg/fontenc}
%
% \item[-] package \xpackage{pdflscape}, 2008/08/11, v0.10,
%   \url{https://ctan.org/pkg/pdflscape}
%
% \item[-] package \xpackage{holtxdoc}, 2012/03/21, v0.24,
%   \url{https://ctan.org/pkg/holtxdoc}
%
% \item[-] package \xpackage{hypdoc}, 2011/08/19, v1.11,
%   \url{https://ctan.org/pkg/hypdoc}
% \end{description}
%
% \DescribeMacro{morefloats.sty}
% The \texttt{morefloats.sty} for \LaTeXe{} \hbox{(i.\,e. each} document using
% the \xpackage{morefloats} package) requires:
% \begin{description}
% \item[-] \TeX Format \LaTeXe{}, \url{https://www.CTAN.org/}
%
% \item[-] package \xpackage{kvoptions}, 2011/06/30, v3.11,
%   \url{https://ctan.org/pkg/kvoptions}
%
% \item[-] package \xpackage{ifetex}, 2011/12/15, v1.2,
%   \url{https://ctan.org/pkg/ifetex}, is used in some cases
% \end{description}
%
% \DescribeMacro{regstats}
% \DescribeMacro{regcount}
% To check the number of used registers it was mentioned:
% \begin{description}
% \item[-] package \xpackage{regstats}, \url{https://ctan.org/pkg/regstats}
% \item[-] package \xpackage{regcount}, \url{https://ctan.org/pkg/regcount}
% \end{description}
%
% \DescribeMacro{Oberdiek}
% \DescribeMacro{holtxdoc}
% \DescribeMacro{hypdoc}
% All packages of \textsc{Heiko Oberdiek}'s bundle `oberdiek'
% (especially \xpackage{holtxdoc}, \xpackage{hypdoc}, and \xpackage{kvoptions})
% are also available in a TDS compliant ZIP archive:\\
% \url{http://mirror.ctan.org/install/macros/latex/contrib/oberdiek.tds.zip}.\\
% It is probably best to download and use this, because the packages in there
% are quite probably both recent and compatible among themselves.\\
%
% \DescribeMacro{hyperref}
% \noindent \xpackage{hyperref} is not included in that bundle and needs to be
% downloaded separately,\\
% \url{http://mirror.ctan.org/install/macros/latex/contrib/hyperref.tds.zip}.\\
%
% \DescribeMacro{M\"{u}nch}
% A hyperlinked list of my (other) packages can be found at
% \url{https://www.ctan.org/author/muench-hm}.\\
%
% \subsection{Package, unpacking TDS}
% \paragraph{Package.} This package is available on \url{https://www.CTAN.org}.
% \begin{description}
% \item[\url{http://mirror.ctan.org/macros/latex/contrib/morefloats/morefloats.dtx}]\hspace*{0.1cm}
%       The source file.
% \item[\url{http://mirror.ctan.org/macros/latex/contrib/morefloats/morefloats.pdf}]\hspace*{0.1cm}
%       The documentation.
% \item[\url{http://mirror.ctan.org/macros/latex/contrib/morefloats/README}]\hspace*{0.1cm}\\
%       \hspace*{1em}The README file.
% \end{description}
%
% \noindent There is also a |morefloats.tds.zip| available:
% \begin{description}
% \item[\url{http://mirror.ctan.org/install/macros/latex/contrib/morefloats.tds.zip}]\hspace*{0.1cm}
%       Everything in TDS compliant, compiled format.
% \end{description}
% which additionally contains\\
% \begin{tabular}{ll}
% morefloats.ins & The installation file.\\
% morefloats.drv & The driver to generate the documentation.\\
% morefloats.sty & The \xext{sty}le file.\\
% morefloats-example.tex & The example file.\\
% morefloats-example.pdf & The compiled example file.
% \end{tabular}
%
% \bigskip
%
% \noindent For required other packages, please see the preceding subsection.
%
% \paragraph{Unpacking.} The  \xfile{.dtx} file is a self-extracting
% \docstrip{} archive. The files are extracted by running the
% \xfile{.dtx} through \plainTeX{}:
% \begin{quote}
%   \verb|tex morefloats.dtx|
% \end{quote}
%
% About generating the documentation see paragraph~\ref{GenDoc} below.\\
%
% \paragraph{TDS.} Now the different files must be moved into
% the different directories in your installation TDS tree
% (also known as \xfile{texmf} tree):
% \begin{quote}
% \def\t{^^A
% \begin{tabular}{@{}>{\ttfamily}l@{ $\rightarrow$ }>{\ttfamily}l@{}}
%   morefloats.sty & tex/latex/morefloats/morefloats.sty\\
%   morefloats.pdf & doc/latex/morefloats/morefloats.pdf\\
%   morefloats-example.tex & doc/latex/morefloats/morefloats-example.tex\\
%   morefloats-example.pdf & doc/latex/morefloats/morefloats-example.pdf\\
%   morefloats.dtx & source/latex/morefloats/morefloats.dtx\\
% \end{tabular}^^A
% }^^A
% \sbox0{\t}^^A
% \ifdim\wd0>\linewidth
%   \begingroup
%     \advance\linewidth by\leftmargin
%     \advance\linewidth by\rightmargin
%   \edef\x{\endgroup
%     \def\noexpand\lw{\the\linewidth}^^A
%   }\x
%   \def\lwbox{^^A
%     \leavevmode
%     \hbox to \linewidth{^^A
%       \kern-\leftmargin\relax
%       \hss
%       \usebox0
%       \hss
%       \kern-\rightmargin\relax
%     }^^A
%   }^^A
%   \ifdim\wd0>\lw
%     \sbox0{\small\t}^^A
%     \ifdim\wd0>\linewidth
%       \ifdim\wd0>\lw
%         \sbox0{\footnotesize\t}^^A
%         \ifdim\wd0>\linewidth
%           \ifdim\wd0>\lw
%             \sbox0{\scriptsize\t}^^A
%             \ifdim\wd0>\linewidth
%               \ifdim\wd0>\lw
%                 \sbox0{\tiny\t}^^A
%                 \ifdim\wd0>\linewidth
%                   \lwbox
%                 \else
%                   \usebox0
%                 \fi
%               \else
%                 \lwbox
%               \fi
%             \else
%               \usebox0
%             \fi
%           \else
%             \lwbox
%           \fi
%         \else
%           \usebox0
%         \fi
%       \else
%         \lwbox
%       \fi
%     \else
%       \usebox0
%     \fi
%   \else
%     \lwbox
%   \fi
% \else
%   \usebox0
% \fi
% \end{quote}
% If you have a \xfile{docstrip.cfg} that configures and enables \docstrip's
% TDS installing feature, then some files can already be in the right
% place, see the documentation of \docstrip{}.
%
% \subsection{Refresh file name databases}
%
% If your \TeX~distribution (\TeX{} Live, \mikTeX, \teTeX, \dots) relies on
% file name databases, you must refresh these. For example, \teTeX{} users run
% \verb|texhash| or \verb|mktexlsr|.
%
% \subsection{Some details for the interested}
%
% \paragraph{Unpacking with \LaTeX{}.}
% The \xfile{.dtx} chooses its action depending on the format:
% \begin{description}
% \item[\plainTeX:] Run \docstrip{} and extract the files.
% \item[\LaTeX:] Generate the documentation.
% \end{description}
% If you insist on using \LaTeX{} for \docstrip{} (really,
% \docstrip{} does not need \LaTeX ), then inform the autodetect routine
% about your intention:
% \begin{quote}
%   \verb|latex \let\install=y% \iffalse meta-comment
%
% File: morefloats.dtx
% Version: 2015/07/22 v1.0h
%
% Copyright (C) 2010 - 2015 by
%    H.-Martin M"unch <Martin dot Muench at Uni-Bonn dot de>
% Portions of code copyrighted by other people as marked.
%
% LaTeX 2015 provides the extrafloats command.
% Don Hosek, Quixote, 1990/07/27 (Thanks!)
% invented the main code for handling more floats
% before extrafloats was available.
% Maintenance has been taken over in September 2010
% by H.-Martin M\"{u}nch.
% David Carlisle pointed the maintainer to the new
% extrafloats command (Thanks!).
%
% This work may be distributed and/or modified under the
% conditions of the LaTeX Project Public License, either
% version 1.3c of this license or (at your option) any later
% version. This version of this license is in
%    http://www.latex-project.org/lppl/lppl-1-3c.txt
% and the latest version of this license is in
%    http://www.latex-project.org/lppl.txt
% and version 1.3c or later is part of all distributions of
% LaTeX version 2005/12/01 or later.
%
% This work has the LPPL maintenance status "maintained".
%
% The Current Maintainer of this work is H.-Martin Muench.
%
% This work consists of the main source file morefloats.dtx,
% the README, and the derived files
%    morefloats.sty, morefloats.pdf,
%    morefloats.ins, morefloats.drv,
%    morefloats-example.tex, morefloats-example.pdf.
%
% 'morefloats' is available on CTAN:
% https://www.ctan.org/pkg/morefloats
%
% Also a TDS.ZIP file is provided that contains all the files
% already sorted in a TDS tree:
% http://mirror.ctan.org/install/macros/latex/contrib/morefloats.tds.zip
%
%<*ignore>
\begingroup
  \catcode123=1 %
  \catcode125=2 %
  \def\x{LaTeX2e}%
\expandafter\endgroup
\ifcase 0\ifx\install y1\fi\expandafter
         \ifx\csname processbatchFile\endcsname\relax\else1\fi
         \ifx\fmtname\x\else 1\fi\relax
\else\csname fi\endcsname
%</ignore>
%<*install>
\input docstrip.tex
\Msg{*******************************************************************************}
\Msg{* Installation                                                                *}
\Msg{* Package: morefloats 2015/07/22 v1.0h Raise limit of unprocessed floats (HMM)*}
\Msg{*******************************************************************************}

\keepsilent
\askforoverwritefalse

\let\MetaPrefix\relax
\preamble

This is a generated file.

Project: morefloats
Version: 2015/07/22 v1.0h

Copyright (C) 2010 - 2015 by
    H.-Martin M"unch <Martin dot Muench at Uni-Bonn dot de>
Portions of code copyrighted by other people as marked.

The usual disclaimer applies:
If it doesn't work right that's your problem.
(Nevertheless, send an e-mail to the maintainer
 when you find an error in this package.)

This work may be distributed and/or modified under the
conditions of the LaTeX Project Public License, either
version 1.3c of this license or (at your option) any later
version. This version of this license is in
   http://www.latex-project.org/lppl/lppl-1-3c.txt
and the latest version of this license is in
   http://www.latex-project.org/lppl.txt
and version 1.3c or later is part of all distributions of
LaTeX version 2005/12/01 or later.

This work has the LPPL maintenance status "maintained".

The Current Maintainer of this work is H.-Martin Muench.

LaTeX 2015 provides the extrafloats command.
Don Hosek, Quixote, 1990/07/27 (Thanks!)
invented the main code for handling more floats
before extrafloats was available.
Maintenance has been taken over in September 2010
by H.-Martin Muench.
David Carlisle pointed the maintainer to the new
extrafloats command (Thanks!).

This work consists of the main source file morefloats.dtx,
the README, and the derived files
   morefloats.sty, morefloats.pdf,
   morefloats.ins, morefloats.drv,
   morefloats-example.tex, morefloats-example.pdf.

In memoriam
 Claudia Simone Barth + 1996/01/30
 Tommy Muench + 2014/01/02
 Hans-Klaus Muench + 2014/08/24

\endpreamble
\let\MetaPrefix\DoubleperCent

\generate{%
  \file{morefloats.ins}{\from{morefloats.dtx}{install}}%
  \file{morefloats.drv}{\from{morefloats.dtx}{driver}}%
  \usedir{tex/latex/morefloats}%
  \file{morefloats.sty}{\from{morefloats.dtx}{package}}%
  \usedir{doc/latex/morefloats}%
  \file{morefloats-example.tex}{\from{morefloats.dtx}{example}}%
}

\catcode32=13\relax% active space
\let =\space%
\Msg{************************************************************************}
\Msg{*}
\Msg{* To finish the installation you have to move the following}
\Msg{* file into a directory searched by TeX:}
\Msg{*}
\Msg{*  morefloats.sty}
\Msg{*}
\Msg{* To produce the documentation run the file `morefloats.drv'}
\Msg{* through (pdf)LaTeX, e.g.}
\Msg{*  pdflatex morefloats.drv}
\Msg{*  makeindex -s gind.ist morefloats.idx}
\Msg{*  pdflatex morefloats.drv}
\Msg{*  makeindex -s gind.ist morefloats.idx}
\Msg{*  pdflatex morefloats.drv}
\Msg{*}
\Msg{* At least three runs are necessary e.g. to get the}
\Msg{*  references right!}
\Msg{*}
\Msg{* Happy TeXing!}
\Msg{*}
\Msg{************************************************************************}

\endbatchfile
%</install>
%<*ignore>
\fi
%</ignore>
%
% \section{The documentation driver file}
%
% The next bit of code contains the documentation driver file for
% \TeX , i.\,e., the file that will produce the documentation you
% are currently reading. It will be extracted from this file by the
% \texttt{docstrip} programme. That is, run \LaTeX{} on \texttt{docstrip}
% and specify the \texttt{driver} option when \texttt{docstrip}
% asks for options.
%
%    \begin{macrocode}
%<*driver>
\NeedsTeXFormat{LaTeX2e}[2015/01/01]
\ProvidesFile{morefloats.drv}%
  [2015/07/22 v1.0h Raise limit of unprocessed floats (HMM)]
\documentclass{ltxdoc}[2015/03/26]%   v2.0w
\usepackage[T1]{fontenc}[2005/09/27]% v1.99g
\usepackage{pdflscape}[2008/08/11]%   v0.10
\usepackage{holtxdoc}[2012/03/21]%    v0.24
%% morefloats should work with earlier versions of LaTeX2e and
%% may work with earlier versions of the class and those packages,
%% but this was not tested.
%% Please consider updating your LaTeX, class, and packages
%% to the most recent version (if they are not already the most
%% recent version).
\hypersetup{%
 pdfsubject={LaTeX2e package for increasing the limit of unprocessed floats (HMM)},%
 pdfkeywords={LaTeX, morefloats, floats, H.-Martin Muench},%
 pdfencoding=auto,%
 pdflang={en},%
 breaklinks=true,%
 linktoc=all,%
 pdfstartview=FitH,%
 pdfpagelayout=OneColumn,%
 bookmarksnumbered=true,%
 bookmarksopen=true,%
 bookmarksopenlevel=2,%
 pdfmenubar=true,%
 pdftoolbar=true,%
 pdfwindowui=true,%
 pdfnewwindow=true%
}
\CodelineIndex
\hyphenation{docu-ment}
\gdef\unit#1{\mathord{\thinspace\mathrm{#1}}}%
\begin{document}
  \DocInput{morefloats.dtx}%
\end{document}
%</driver>
%    \end{macrocode}
%
% \fi
%
% \CheckSum{3565}
%
% \CharacterTable
%  {Upper-case    \A\B\C\D\E\F\G\H\I\J\K\L\M\N\O\P\Q\R\S\T\U\V\W\X\Y\Z
%   Lower-case    \a\b\c\d\e\f\g\h\i\j\k\l\m\n\o\p\q\r\s\t\u\v\w\x\y\z
%   Digits        \0\1\2\3\4\5\6\7\8\9
%   Exclamation   \!     Double quote  \"     Hash (number) \#
%   Dollar        \$     Percent       \%     Ampersand     \&
%   Acute accent  \'     Left paren    \(     Right paren   \)
%   Asterisk      \*     Plus          \+     Comma         \,
%   Minus         \-     Point         \.     Solidus       \/
%   Colon         \:     Semicolon     \;     Less than     \<
%   Equals        \=     Greater than  \>     Question mark \?
%   Commercial at \@     Left bracket  \[     Backslash     \\
%   Right bracket \]     Circumflex    \^     Underscore    \_
%   Grave accent  \`     Left brace    \{     Vertical bar  \|
%   Right brace   \}     Tilde         \~}
%
% \GetFileInfo{morefloats.drv}
%
% \begingroup
%   \def\x{\#,\$,\^,\_,\~,\ ,\&,\{,\},\%}%
%   \makeatletter
%   \@onelevel@sanitize\x
% \expandafter\endgroup
% \expandafter\DoNotIndex\expandafter{\x}
% \expandafter\DoNotIndex\expandafter{\string\ }
% \begingroup
%   \makeatletter
%     \lccode`9=32\relax
%     \lowercase{%^^A
%       \edef\x{\noexpand\DoNotIndex{\@backslashchar9}}%^^A
%     }%^^A
%   \expandafter\endgroup\x
% \DoNotIndex{\\,\,}
% \DoNotIndex{\def,\edef,\gdef, \xdef}
% \DoNotIndex{\ifnum, \ifx}
% \DoNotIndex{\begin, \end, \LaTeX, \LateXe}
% \DoNotIndex{\bigskip, \caption, \centering, \hline, \MessageBreak}
% \DoNotIndex{\documentclass, \markboth, \mathrm, \mathord}
% \DoNotIndex{\NeedsTeXFormat, \usepackage, \ProvidesPackage, \RequirePackage}
% \DoNotIndex{\newline, \newpage, \pagebreak}
% \DoNotIndex{\section, \subsection, \space, \thinspace}
% \DoNotIndex{\textsf, \texttt}
% \DoNotIndex{\the, \@tempcnta,\@tempcntb}
% \DoNotIndex{\@elt,\@freelist, \newinsert}
% \DoNotIndex{\bx@A,  \bx@B,  \bx@C,  \bx@D,  \bx@E,  \bx@F,  \bx@G,  \bx@H,  \bx@I,  \bx@J,  \bx@K,  \bx@L,  \bx@M,  \bx@N,  \bx@O,  \bx@P,  \bx@Q,  \bx@R,  \bx@S,  \bx@T,  \bx@U,  \bx@V,  \bx@W,  \bx@X,  \bx@Y,  \bx@Z}
% \DoNotIndex{\bx@AA, \bx@AB, \bx@AC, \bx@AD, \bx@AE, \bx@AF, \bx@AG, \bx@AH, \bx@AI, \bx@AJ, \bx@AK, \bx@AL, \bx@AM, \bx@AN, \bx@AO, \bx@AP, \bx@AQ, \bx@AR, \bx@AS, \bx@AT, \bx@AU, \bx@AV, \bx@AW, \bx@AX, \bx@AY, \bx@AZ}
% \DoNotIndex{\bx@BA, \bx@BB, \bx@BC, \bx@BD, \bx@BE, \bx@BF, \bx@BG, \bx@BH, \bx@BI, \bx@BJ, \bx@BK, \bx@BL, \bx@BM, \bx@BN, \bx@BO, \bx@BP, \bx@BQ, \bx@BR, \bx@BS, \bx@BT, \bx@BU, \bx@BV, \bx@BW, \bx@BX, \bx@BY, \bx@BZ}
% \DoNotIndex{\bx@CA, \bx@CB, \bx@CC, \bx@CD, \bx@CE, \bx@CF, \bx@CG, \bx@CH, \bx@CI, \bx@CJ, \bx@CK, \bx@CL, \bx@CM, \bx@CN, \bx@CO, \bx@CP, \bx@CQ, \bx@CR, \bx@CS, \bx@CT, \bx@CU, \bx@CV, \bx@CW, \bx@CX, \bx@CY, \bx@CZ}
% \DoNotIndex{\bx@DA, \bx@DB, \bx@DC, \bx@DD, \bx@DE, \bx@DF, \bx@DG, \bx@DH, \bx@DI, \bx@DJ, \bx@DK, \bx@DL, \bx@DM, \bx@DN, \bx@DO, \bx@DP, \bx@DQ, \bx@DR, \bx@DS, \bx@DT, \bx@DU, \bx@DV, \bx@DW, \bx@DX, \bx@DY, \bx@DZ}
% \DoNotIndex{\bx@EA, \bx@EB, \bx@EC, \bx@ED, \bx@EE, \bx@EF, \bx@EG, \bx@EH, \bx@EI, \bx@EJ, \bx@EK, \bx@EL, \bx@EM, \bx@EN, \bx@EO, \bx@EP, \bx@EQ, \bx@ER, \bx@ES, \bx@ET, \bx@EU, \bx@EV, \bx@EW, \bx@EX, \bx@EY, \bx@EZ}
% \DoNotIndex{\bx@FA, \bx@FB, \bx@FC, \bx@FD, \bx@FE, \bx@FF, \bx@FG, \bx@FH, \bx@FI, \bx@FJ, \bx@FK, \bx@FL, \bx@FM, \bx@FN, \bx@FO, \bx@FP, \bx@FQ, \bx@FR, \bx@FS, \bx@FT, \bx@FU, \bx@FV, \bx@FW, \bx@FX, \bx@FY, \bx@FZ}
% \DoNotIndex{\bx@GA, \bx@GB, \bx@GC, \bx@GD, \bx@GE, \bx@GF, \bx@GG, \bx@GH, \bx@GI, \bx@GJ, \bx@GK, \bx@GL, \bx@GM, \bx@GN, \bx@GO, \bx@GP, \bx@GQ, \bx@GR, \bx@GS, \bx@GT, \bx@GU, \bx@GV, \bx@GW, \bx@GX, \bx@GY, \bx@GZ}
% \DoNotIndex{\bx@HA, \bx@HB, \bx@HC, \bx@HD, \bx@HE, \bx@HF, \bx@HG, \bx@HH, \bx@HI, \bx@HJ, \bx@HK, \bx@HL, \bx@HM, \bx@HN, \bx@HO, \bx@HP, \bx@HQ, \bx@HR, \bx@HS, \bx@HT, \bx@HU, \bx@HV, \bx@HW, \bx@HX, \bx@HY, \bx@HZ}
% \DoNotIndex{\bx@IA, \bx@IB, \bx@IC, \bx@ID, \bx@IE, \bx@IF, \bx@IG, \bx@IH, \bx@II, \bx@IJ, \bx@IK, \bx@IL, \bx@IM, \bx@IN, \bx@IO, \bx@IP, \bx@IQ, \bx@IR, \bx@IS, \bx@IT, \bx@IU, \bx@IV, \bx@IW, \bx@IX, \bx@IY, \bx@IZ}
% \DoNotIndex{\bx@JA, \bx@JB, \bx@JC, \bx@JD, \bx@JE, \bx@JF, \bx@JG, \bx@JH, \bx@JI, \bx@JJ, \bx@JK, \bx@JL, \bx@JM, \bx@JN, \bx@JO, \bx@JP, \bx@JQ, \bx@JR, \bx@JS, \bx@JT, \bx@JU, \bx@JV, \bx@JW, \bx@JX, \bx@JY, \bx@JZ}
% \DoNotIndex{\morefloats@mx}
%
% \title{The \xpackage{morefloats} package}
% \date{2015/07/22 v1.0h}
% \author{H.-Martin M\"{u}nch (current maintainer;\\
%  invented by Don Hosek, Quixote)\\
%  \xemail{Martin.Muench at Uni-Bonn.de}}
%
% \maketitle
%
% \begin{abstract}
% The default limit of unprocessed floats, $18$,
% can be increased with this \xpackage{morefloats} package.
% Otherwise, |\clear(double)page|, |h(!)|, |H|~from the \xpackage{float} package,
% or |\FloatBarrier| from the \xpackage{picins} package might help.
% \end{abstract}
%
% \bigskip
%
% \noindent Note: \LaTeX{} 2015 provides the |\extrafloats| command.
% \textsc{Don Hosek}, Quixote, 1990/07/27 (Thanks!)
% invented the main code for handling more floats
% before |\extrafloats| was available.
% \textsc{David Carlisle} pointed the maintainer to the new
% |\extrafloats| (Thanks!).
% The current maintainer is \textsc{H.-Martin M\"{u}nch}.\\
%
% \bigskip
%
% \noindent Disclaimer for web links: The author is not responsible for any contents
% referred to in this work unless he has full knowledge of illegal contents.
% If any damage occurs by the use of information presented there, only the
% author of the respective pages might be liable, not the one who has referred
% to these pages.
%
% \bigskip
%
% \noindent {\color{green} Save per page about $200\unit{ml}$ water,
% $2\unit{g}$ CO$_{2}$ and $2\unit{g}$ wood:\\
% Therefore please print only if this is really necessary.}
%
% \newpage
%
% \tableofcontents
%
% \newpage
%
% \section{Introduction\label{sec:Introduction}}
%
% The default limit of unprocessed floats, $18$,
% can be increased with this \xpackage{morefloats} package.\\
% \textquotedblleft{}Of course one immediately begins to wonder:
% \guillemotright{}Why eighteen?!\guillemotleft{} And it turns out that $18$
% one{-}line tables with $10$~point Computer Modern using \xclass{article.cls}
% produces almost exactly one page worth of material.\textquotedblright{}\\
% (user \url{https://tex.stackexchange.com/users/1495/kahen} as comment to\\
% \url{https://tex.stackexchange.com/a/35596/6865} on 2011/11/21)\\
% As alternatives (see also section \ref{sec:alternatives} below)
% |\clear(double)page|, |h(!)|, |H|~from the
% \href{https://www.ctan.org/pkg/float}{\xpackage{float}} package,
% or |\FloatBarrier| from the %
% \href{https://www.ctan.org/pkg/picins}{\xpackage{picins}} package might help.
% If the floats cannot be placed anywhere at all, extending the number of floats
% will just delay the arrival of the corresponding error.
%
% \section{Usage}
%
% \subsection{General usage:}
% Load the package placing
% \begin{quote}
%   |\usepackage[<|\textit{options}|>]{morefloats}|
% \end{quote}
% \noindent in the preamble of your \LaTeXe{} source file (the earlier the better).\\
% \noindent The \xpackage{morefloats} package takes two options: |maxfloats| and
% |morefloats|, where |morefloats| gives the number of additional floats and
% |maxfloats| gives the maximum number of floats. |maxfloats=25| therefore means,
% that there are $18$ (default) floats and $7$ additional floats.
% |morefloats=7| therefore has the same meaning. It is only necessary to give
% one of these two options. At the time being, it is not possible to reduce
% the number of floats (for example to save boxes). If you have code
% accomplishing that, please send it to the package maintainer, thanks.\\
% Version 1.0b used a fixed value of |maxfloats=36|. Therefore for backward
% compatibility this value is taken as the default one.\\
% Example:
% \begin{quote}
%   |\usepackage[maxfloats=25]{morefloats}|
% \end{quote}
% or
% \begin{quote}
%   |\usepackage[morefloats=7]{morefloats}|
% \end{quote}
% or
% \begin{quote}
%   |\usepackage[maxfloats=25,morefloats=7]{morefloats}|
% \end{quote}
%
% \subsection{Situation for \LaTeX{} before 2015:}
% |Float| uses |insert|, and each |insert| uses a group of |count|, |dimen|,
% |skip|, and |box| each. When there are not enough available, no |\newinsert|
% can be created. The
% \href{https://www.ctan.org/pkg/etex-pkg}{\xpackage{etex}} package
% provides access at an extended range of those registers,
% but does not use those for |\newinsert|. Therefore the inserts must be
% reserved first, which forces the use of the extended register range
% for other new |count|, |dimen|, |skip|, and |box|:
% To have more floats available, use |\usepackage{etex}\reserveinserts{...}|
% right after |\documentclass[...]{...}|, where the argument of |\reserveinserts|
% should be at least the maximum number of floats. Add another $10$
% if the \href{https://www.ctan.org/pkg/bigfoot}{\xpackage{bigfoot}} or the
% \href{https://www.ctan.org/pkg/manyfoot}{\xpackage{manyfoot}} package
% is used, but |\reserveinserts| can be about $234$ at most for older
% \LaTeX{} formats.
%
% \subsection{Situation for \LaTeX{} since 2015:}
% Now |\reserveinserts| can be about $2\,147\,483\,647$,
% but |\insert255{}| even then produces an error.
% The \LaTeX{} 2015 \textquotedblleft release provides a new command in the format
% |\extrafloats|\textquotedblright ; \textquotedblleft as it doesn't use
% |\newinsert| (and as the 2015 format uses extended registers by default)
% you can allocate a lot more floats\textquotedblright{} %
% (both \textsc{David Carlisle}, 29. June 2015), \hbox{e.\,g. |\extrafloats{1234}|.}
%
% \section{Alternatives (kind of)\label{sec:alternatives}}
%
% The very old \xpackage{morefloats} with a fixed number of |maxfloats=36| {}%
% \hbox{(i.\,e. $18$ |morefloats|)} has been archived at
% \href{http://mirror.ctan.org/obsolete/macros/latex/contrib/misc/morefloats.sty}{%
%  http://mirror.ctan.org/obsolete/macros/latex/contrib/}\newline%
% \href{http://mirror.ctan.org/obsolete/macros/latex/contrib/misc/morefloats.sty}{%
%  misc/morefloats.sty}.
%
% \bigskip
%
% If you really want to increase the number of (possible) floats,
% this is the right package. On the other hand, if you ran into trouble of
% \texttt{Too many unprocessed floats}, but would also accept less floats,
% there are some other possibilities:
% \begin{description}
%   \item[-] The command |\clearpage| forces \LaTeX{} to output any floating objects
%     that occurred before this command (and go to the next page).
%     |\cleardoublepage| does the same but ensures that the next page with
%     output is one with odd page number.
%   \item[-] Using different float specifiers: |t|~top, |b|~bottom, |p|~page
%     of floats.
%   \item[-] Suggesting \LaTeX{} to put the object where it was placed:
%     |h| (= here) float specifier.
%   \item[-] Telling \LaTeX{} to please put the object where it was placed:
%     |h!| (= here!) float specifier.
%   \item[-] Forcing \LaTeX{} to put the object where it was placed and shut up:
%     The \xpackage{float} package provides the \textquotedblleft style
%     option here, giving floating environments a [H] option which means
%     `PUT IT HERE' (as opposed to the standard [h] option which means
%     `You may put it here if you like')\textquotedblright{} (\xpackage{float}
%     package documentation v1.3d as of 2001/11/08).
%     Changing e.\,g. |\begin{figure}[tbp]...| to |\begin{figure}[H]...|
%     forces the figure to be placed HERE instead of floating away.\\
%     The \xpackage{float} package is available at \url{https://www.ctan.org/pkg/float}.
%   \item[-] The \xpackage{placeins} package provides the command |\FloatBarrier|.
%     Floats occurring before the |\FloatBarrier| are not allowed to float
%     to a later place, and floats occurring after the |\FloatBarrier| are not
%     allowed to float to an earlier place than the |\FloatBarrier|. (There
%     can be more than one |\FloatBarrier| in a document.) -- %
%     The same package also provides an option to automatically add |\FloatBarrier|s to
%     section headings. It is further possible to make
%     |\FloatBarrier|s less strict (see that package's documentation).\\
%     The \xpackage{placeins} package is available at \url{https://www.ctan.org/pkg/placeins}.
%   \item[-] Sometimes also increasing the maximum number (|\maxdeadcycles|)
%     of calls of |\output| without a |\shipout| can help,
%     for example |\maxdeadcycles=123\relax|.
% \end{description}
%
% \newpage
%
% \noindent See also the following entries in the
% \texttt{UK~List of TeX Frequently Asked Questions on the Web}:
% \begin{description}
%   \item[-] \url{http://www.tex.ac.uk/cgi-bin/texfaq2html?label=floats}
%   \item[-] \url{http://www.tex.ac.uk/cgi-bin/texfaq2html?label=tmupfl}
%   \item[-] \url{http://www.tex.ac.uk/cgi-bin/texfaq2html?label=figurehere}
% \end{description}
% and the \textbf{excellent article on \textquotedblleft How to influence the position
% of float environments like figure and table in \hbox{\LaTeX ?\textquotedblright } by
% \textsc{Frank Mittelbach}} at \url{https://tex.stackexchange.com/a/39020/6865}{}!\\
%
% \bigskip
%
% \noindent (You programmed or found another alternative,
%  which is available at CTAN?\\
%  OK, send an e-mail to me with the name, location at CTAN,
%  and a short notice, and I will probably include it in
%  the list above.)
%
% \bigskip
%
% \section{Example}
%
%    \begin{macrocode}
%<*example>
\documentclass[british]{article}[2014/09/29]%      v1.4h
%%%%%%%%%%%%%%%%%%%%%%%%%%%%%%%%%%%%%%%%%%%%%%%%%%%%%%%%%%%%%%%%%%%%%
\usepackage[maxfloats=25]{morefloats}[2015/07/22]% v1.0h
%\maxdeadcycles=200\relax%
%% \maxdeadcycles is the maximum number of calls of \output
%% without a \shipout.
\gdef\unit#1{\mathord{\thinspace\mathrm{#1}}}%
\listfiles
\begin{document}

\makeatletter

\section*{Example for morefloats}
\markboth{Example for morefloats}{Example for morefloats}

This example demonstrates the use of package\newline
\textsf{morefloats}, v1.0h as of 2015/07/22 (HMM).\newline
The package takes options (here:
\verb|maxfloats=|\texttt{\morefloats@maxfloats} is used).\newline
For more details please see the documentation!\newline

To reproduce the\newline
\LaTeX{} \texttt{ Error: Too many unprocessed floats},\newline
comment out the \verb|\usepackage...| in the preamble
(line~3)\newline
(by placing a \% before it).\newline

\bigskip

Save per page about $200\unit{ml}$~water, $2\unit{g}$~CO$_{2}$
and $2\unit{g}$~wood:\newline
Therefore please print only if this is really necessary.\newline
I do NOT think, that it is necessary to print THIS file, really!

\bigskip

There follow \morefloats@maxfloats{} floating tables.

\pagebreak

\@tempcnta=18\relax% default floats
\advance\@tempcnta by \morefloats@morefloats%
% \morefloats@morefloats is the number of additional
% floating tables to create.
\loop
  \ifnum\@tempcnta>0\relax%
  \begin{table}[t]\centering%
    \begin{tabular}{|l|}%
      \hline%
      A table, which will keep floating.\\%
      \hline
    \end{tabular}%
    \caption{A floating Table.}%
  \end{table}%
  \advance\@tempcnta by -1\relax%
\repeat

\makeatother

\end{document}
%</example>
%    \end{macrocode}
%
% \newpage
%
% \StopEventually{}
%
% \section{The implementation}
%
% We start off by checking that we are loading into \LaTeXe{} and
% announcing the name and version of this package.
%
%    \begin{macrocode}
%<*package>
%    \end{macrocode}
%
%    \begin{macrocode}
\NeedsTeXFormat{LaTeX2e}[2011/06/27]
%% The current format at the time of the release of this version of the
%% morefloats package was 2015/01/01, patch level 2.
\ProvidesPackage{morefloats}[2015/07/22 v1.0h
            Raise limit of unprocessed floats (HMM)]

%    \end{macrocode}
%
% \DescribeMacro{Options}
%    \begin{macrocode}
\RequirePackage{kvoptions}[2011/06/30]% v3.11
%% morefloats may work with earlier versions of LaTeX2e and that
%% package, but this was not tested.
%% Please consider updating your LaTeX and package
%% to the most recent version (if they are not already the most
%% recent version).

\SetupKeyvalOptions{family=morefloats,prefix=morefloats@}
\DeclareStringOption{maxfloats}%  \morefloats@maxfloats
\DeclareStringOption{morefloats}% \morefloats@morefloats

\ProcessKeyvalOptions*

%    \end{macrocode}
%
% The \xpackage{morefloats} package takes two options: |maxfloats| and |morefloats|,
% where |morefloats| gives the number of additional floats and |maxfloats| gives
% the maximum number of floats. |maxfloats=37| therefore means, that there are
% $18$ (default) floats and another $19$ additional floats. |morefloats=19| therefore
% has the same meaning. Version~1.0b used a fixed value of |maxfloats=36|.
% Therefore for backward compatibility this value will be taken as the default one.\\
% Now we check whether |maxfloats=...| or |morefloats=...| or both were used,
% and if one option was not used, we supply the according value.
% If no option was used at all, we use the default values.
% Too many requested floats produce error massages by \LaTeX ,
% which might not be easily traced back to this,
% therefore we issue a warning. If option |maxfloats| or |morefloats| is no number,
% the user will received the according error message by \LaTeX{} automatically.
%
%    \begin{macrocode}
\ifx\morefloats@maxfloats\@empty%
  \ifx\morefloats@morefloats\@empty% apply defaults:
    \gdef\morefloats@maxfloats{36}%
    \gdef\morefloats@morefloats{18}%
  \else%
    \ifnum\morefloats@morefloats>1569\relax%
      \PackageWarning{morefloats}{%
        \morefloats@morefloats\space more floats requested.\MessageBreak%
        LaTeX might run out of memory before this\MessageBreak%
        (in which case it will notify you)\MessageBreak%
       }%
    \else%
      \PackageInfo{morefloats}{%
        \morefloats@morefloats\space more floats requested.\MessageBreak%
        LaTeX might run out of memory before this\MessageBreak%
        (in which case it will notify you)\MessageBreak%
       }%
    \fi%
    \@tempcnta=\morefloats@morefloats\relax%
    \advance\@tempcnta by +18%
    \xdef\morefloats@maxfloats{\the\@tempcnta}%
  \fi%
\else%
  \ifx\morefloats@morefloats\@empty%
    \@tempcnta=\morefloats@maxfloats\relax%
    \advance\@tempcnta by -18%
    \xdef\morefloats@morefloats{\the\@tempcnta}%
    \ifnum\morefloats@morefloats<\z@\relax% i.e. \morefloats@maxfloats < 18
      \gdef\morefloats@morefloats{0}%
    \fi%
    \ifnum\morefloats@maxfloats>1587\relax%
      \PackageWarning{morefloats}{%
        \morefloats@maxfloats\space floats requested.\MessageBreak%
        LaTeX might run out of memory before this\MessageBreak%
        (in which case it will notify you)\MessageBreak%
       }%
    \fi%
  \fi%
\fi%

\@tempcnta=\morefloats@maxfloats\relax%
\xdef\morefloats@max{\the\@tempcnta}%

\ifnum\@tempcnta<18\relax%
  \PackageError{morefloats}{Option maxfloats is \the\@tempcnta<18}{%
    maxfloats must be a number equal to or larger than 18\MessageBreak%
    (or not used at all).\MessageBreak%
    Now setting maxfloats=18.\MessageBreak%
   }%
  \gdef\morefloats@max{18}%
\fi%

\@tempcnta=\morefloats@morefloats\relax%
\xdef\morefloats@more{\the\@tempcnta}%

\ifnum\@tempcnta<\z@\relax%
  \PackageError{morefloats}{Option morefloats is \the\@tempcnta<0}{%
    morefloats must be a number equal to or larger than 0\MessageBreak%
    (or not used at all).\MessageBreak%
    Now setting morefloats=0.\MessageBreak%
   }%
  \gdef\morefloats@more{0}%
\fi%

\@tempcnta=18\relax%
\advance\@tempcnta by \morefloats@more%
%    \end{macrocode}
%
% The value of |morefloats| should now be equal to the value of |morefloats@max|.
%
%    \begin{macrocode}
\advance\@tempcnta by -\morefloats@max%
%    \end{macrocode}
%
% Therefore |\@tempcnta| should now be equal to zero.
%
%    \begin{macrocode}
\xdef\morefloats@mx{\the\@tempcnta}%
\ifnum\morefloats@mx=\z@\relax%
  \@tempcnta=\morefloats@maxfloats\relax%
\else%
  \PackageError{morefloats}{%
    Clash between options maxfloats and morefloats}{%
    Option maxfloats must be empty\MessageBreak%
    or the sum of 18 and option value morefloats,\MessageBreak%
    but it is maxfloats=\morefloats@maxfloats\space and %
    morefloats=\morefloats@morefloats .\MessageBreak%
    }%
%    \end{macrocode}
%
% We choose the larger value to be used.
%
%    \begin{macrocode}
  \ifnum\@tempcnta<\z@% \morefloats@max > \morefloats@more
    \@tempcnta=\morefloats@maxfloats\relax%
  \else% \@tempcnta>0, \morefloats@max < \morefloats@more
    \@tempcnta=18\relax%
    \advance\@tempcnta by \morefloats@morefloats%
  \fi%
\fi%
\edef\morefloats@mx{\the\@tempcnta}%
%    \end{macrocode}
%
% Maybe we had to change |\morefloats@maxfloats| or |\morefloats@maxfloats|:
%
%    \begin{macrocode}
\xdef\morefloats@maxfloats{\the\@tempcnta}%
\advance\@tempcnta by -18\relax%
\xdef\morefloats@morefloats{\the\@tempcnta}%
\gdef\morefloats@test{1}%
\ifx\morefloats@morefloats\morefloats@test\relax%
  \PackageInfo{morefloats}{%
    Maximum number of possible floats asked for: \morefloats@maxfloats%
    \MessageBreak%
    (i.e. one more float)\@gobble%
   }%
\else%
  \PackageInfo{morefloats}{%
    Maximum number of possible floats asked for: \morefloats@maxfloats%
    \MessageBreak%
    (i.e. \morefloats@morefloats\space more floats).\MessageBreak%
    LaTeX might run out of memory before this\MessageBreak%
    (in which case it will notify you)%
    \@gobble%
   }%
\fi%


%    \end{macrocode}
%
% The \LaTeX{} 2015 \textquotedblleft release provides a new command in the format
% |\extrafloats| which does a similar job [as earlier versions of this package did],
% although as it doesn't use |\newinsert| (and as the 2015 format uses extended
% registers by default) you can allocate a lot more floats,\textquotedblright{} %
% \hbox{e.\,g. |\extrafloats{1234}|.} Loading the \xpackage{etex} package and
% \xpackage{morefloats} with the new format would
% \textquotedblleft over{-}write the new allocation mechanism and end up with
% fewer floats available.\textquotedblright{} Therefore here it is tested
% \textquotedblleft for the new format and switch[ed] to the new mechanism
% in that case, so that existing documents work as before but using the new allocation
% scheme underneath.\textquotedblright{} (all \textsc{David Carlisle}, 29. June 2015,
% who provided also main parts of the following code)
%
%    \begin{macrocode}
%% Test for new mechanism in LaTeX 2015:
\ifx\e@alloc\@undefined\relax%
  %% This is an old LaTeX format, \extrafloats is not available.
  \PackageWarning{morefloats}{%
    \fmtname\space <\fmtversion> %
    \ifx\patch@level\@undefined\relax%
    \else patch level \patch@level%
    \fi%
    \MessageBreak%
    found. At least\MessageBreak%
    LaTeX2e <2015/01/01> patch level 2\MessageBreak%
    is now available\MessageBreak%
    and can handle even more floats%
    \@gobble%
   }%
\else%
  %% This is new in LaTeX 2015, \extrafloats is available.
  \@ifpackageloaded{etex}%
  {%% etex package loaded:
   %% "it overwrites all the new allocation system
   %% so really \extrafloats shouldn't be expected to work"
   %% (D. Carlisle, 2015/07/16, who also provided the following
   %% \extrafloats redefinition).
   \gdef\extrafloats#1{%
     \ifnum#1>\z@\relax%
       \count@\numexpr\float@count-1\relax%
       \ch@ck0\count@\count\relax%
       \ch@ck1\count@\dimen\relax%
       \ch@ck2\count@\skip\relax%
       \ch@ck4\count@\box\relax%
       \e@alloc@chardef\float@count\count@%
       \expandafter\e@alloc@chardef\csname bx@\the\float@count\endcsname\float@count%
       \@cons\@freelist{\csname bx@\the\float@count\endcsname}%
       \expandafter%
       \extrafloats\expandafter{\numexpr#1-1\relax}%
     \fi%
   }%
  }{% etex package not loaded
   }%
  \extrafloats{\morefloats@morefloats}%
  % The part after the test is no longer needed and therefore not loaded:
  \expandafter\endinput%
\fi%
%% End of the test for LaTeX 2015 (or newer).
%% Not new format, otherwise the last \endinput would have been applied.

%% Test for e-TeX:
\RequirePackage{ifetex}[2011/12/15]% v1.2
\ifetex%
  %% then we can use code similar to the one from David Carlisle,
  %% https://tex.stackexchange.com/a/212483/6865
  \mathchardef\float@count=32767\relax%
  \gdef\extrafloats#1{%
    \ifnum#1>\z@\relax%
      \count@\numexpr\float@count-1\relax%
      \ch@ck0\count@\count\relax%
      \ch@ck1\count@\dimen\relax%
      \ch@ck2\count@\skip\relax%
      \ch@ck4\count@\box\relax%
      \mathchardef\float@count\count@\relax%
      \expandafter\mathchardef\csname bx@\the\float@count\endcsname\float@count%
      \@cons\@freelist{\csname bx@\the\float@count\endcsname}%
      \expandafter%
      \extrafloats\expandafter{\numexpr#1-1\relax}%
    \fi}%
  \extrafloats{\morefloats@morefloats}%
  \expandafter\endinput%
\fi%
%% End of the test for e-TeX.
%% Old format and not e-TeX,
%% otherwise the last \endinput would have been applied.


%    \end{macrocode}
%
% If we ever come to this place, \textquotedblleft everything\textquotedblright{} %
% failed and we need to do things the old fashioned way,
% which severely limits the maximum number of additionally available floats.
%
%    \begin{macrocode}
\PackageWarning{morefloats}{%
  e-TeX is not available here\MessageBreak%
  but it is available in almost all\MessageBreak%
  recent TeX distributions.\MessageBreak%
  Maybe consider updating to one of those%
  \@gobble%
 }%

%    \end{macrocode}
%
% \newpage
%
% \begin{landscape}
%
% |Float| uses |insert|, and each |insert| use a group of |count|, |dimen|, |skip|,
% and |box| each. When there are not enough available, no |\newinsert| can be created.
%
%    \begin{macrocode}
%% Code similar to the one from Heiko Oberdiek,
%% http://permalink.gmane.org/gmane.comp.tex.latex.latex3/2159
                           \@tempcnta=\the\count10 \relax \def\maxfloats@vln{count}    %
\ifnum \count11>\@tempcnta \@tempcnta=\the\count11 \relax \def\maxfloats@vln{dimen} \fi%
\ifnum \count12>\@tempcnta \@tempcnta=\the\count12 \relax \def\maxfloats@vln{skip}  \fi%
\ifnum \count14>\@tempcnta \@tempcnta=\the\count14 \relax \def\maxfloats@vln{box}   \fi%
%% end similar
\@tempcntb=234\relax%
\advance\@tempcntb by -\@tempcnta\relax%
\@tempcnta=\@tempcntb\relax%
\ifnum\morefloats@mx>\@tempcntb\relax%
  \PackageError{morefloats}{Too many floats requested}{%
    Maximum number of possible floats asked for: \morefloats@mx .\MessageBreak%
    There are only \the\@tempcnta\space \maxfloats@vln\space left,\MessageBreak%
    therefore only \the\@tempcntb\space floats will be possible.\MessageBreak%
    Load the morefloats package earlier and/or\MessageBreak%
    reduce the number of used \maxfloats@vln\space registers\MessageBreak%
    to have more floats available!\MessageBreak%
   }%
  \xdef\morefloats@mx{\the\@tempcntb}%
\fi%

%    \end{macrocode}
%
% The task at hand is to increase \LaTeX{}'s default limit of $18$~unprocessed
% floats in memory at once to |maxfloats|.
% An examination of \texttt{latex.tex} reveals that this is accomplished
% by allocating~(!) an insert register for each unprocessed float. A~quick
% check of (the obsolete, now \texttt{ltplain}, update to \LaTeX2e{}!)
% \texttt{lplain.lis} reveals that there is room, in fact, for up to
% $256$ unprocessed floats, but \TeX{}'s main memory could be exhausted
% well before that happened.\\
%
% \LaTeX2e{} uses a |\dimen| for each |\newinsert|, and the number of |\dimen|s
% is also restricted. Therefore only use the number of floats you need!
% To check the number of used registers, you could use the \xpackage{regstats}
% and/or \xpackage{regcount} packages (see subsection~\ref{ss:Downloads}).
%
% \bigskip
%
% \DescribeMacro{Allocating insert registers}
% \DescribeMacro{@freelist}
% \DescribeMacro{@elt}
% \DescribeMacro{newinsert}
% First we allocate the additional insert registers needed.\\
%
% That accomplished, the next step is to define the macro |\@freelist|,
% which is merely a~list of the box registers each preceded by |\@elt|.
% This approach allows processing of the list to be done far more efficiently.
% A similar approach is used by \textsc{Mittelbach \& Sch\"{o}pf}'s \texttt{doc.sty} to
% keep track of control sequences, which should not be indexed.\\
% First for the 18 default \LaTeX{} boxes.\\
% \noindent |\ifnum maxfloats <= 18|, \LaTeX{} already allocated the insert registers. |\fi|\\
%
%    \begin{macrocode}
\global\long\def\@freelist{\@elt\bx@A\@elt\bx@B\@elt\bx@C\@elt\bx@D\@elt\bx@E\@elt\bx@F\@elt\bx@G\@elt\bx@H\@elt%
\bx@I\@elt\bx@J\@elt\bx@K\@elt\bx@L\@elt\bx@M\@elt\bx@N\@elt\bx@O\@elt\bx@P\@elt\bx@Q\@elt\bx@R}

%    \end{macrocode}
%
% Now we need to add |\@elt\bx@...| depending on the number of |morefloats| wanted:\\
% (\textsc{Karl Berry} helped with two out of three |\expandafter|s, thanks!)
%
% \medskip
%
%    \begin{macrocode}
\ifnum \morefloats@mx> 18 \newinsert\bx@S  \expandafter\gdef\expandafter\@freelist\expandafter{\@freelist \@elt\bx@S}
\ifnum \morefloats@mx> 19 \newinsert\bx@T  \expandafter\gdef\expandafter\@freelist\expandafter{\@freelist \@elt\bx@T}
\ifnum \morefloats@mx> 20 \newinsert\bx@U  \expandafter\gdef\expandafter\@freelist\expandafter{\@freelist \@elt\bx@U}
\ifnum \morefloats@mx> 21 \newinsert\bx@V  \expandafter\gdef\expandafter\@freelist\expandafter{\@freelist \@elt\bx@V}
\ifnum \morefloats@mx> 22 \newinsert\bx@W  \expandafter\gdef\expandafter\@freelist\expandafter{\@freelist \@elt\bx@W}
\ifnum \morefloats@mx> 23 \newinsert\bx@X  \expandafter\gdef\expandafter\@freelist\expandafter{\@freelist \@elt\bx@X}
\ifnum \morefloats@mx> 24 \newinsert\bx@Y  \expandafter\gdef\expandafter\@freelist\expandafter{\@freelist \@elt\bx@Y}
\ifnum \morefloats@mx> 25 \newinsert\bx@Z  \expandafter\gdef\expandafter\@freelist\expandafter{\@freelist \@elt\bx@Z}
\ifnum \morefloats@mx> 26 \newinsert\bx@AA \expandafter\gdef\expandafter\@freelist\expandafter{\@freelist \@elt\bx@AA}
\ifnum \morefloats@mx> 27 \newinsert\bx@AB \expandafter\gdef\expandafter\@freelist\expandafter{\@freelist \@elt\bx@AB}
\ifnum \morefloats@mx> 28 \newinsert\bx@AC \expandafter\gdef\expandafter\@freelist\expandafter{\@freelist \@elt\bx@AC}
\ifnum \morefloats@mx> 29 \newinsert\bx@AD \expandafter\gdef\expandafter\@freelist\expandafter{\@freelist \@elt\bx@AD}
\ifnum \morefloats@mx> 30 \newinsert\bx@AE \expandafter\gdef\expandafter\@freelist\expandafter{\@freelist \@elt\bx@AE}
\ifnum \morefloats@mx> 31 \newinsert\bx@AF \expandafter\gdef\expandafter\@freelist\expandafter{\@freelist \@elt\bx@AF}
\ifnum \morefloats@mx> 32 \newinsert\bx@AG \expandafter\gdef\expandafter\@freelist\expandafter{\@freelist \@elt\bx@AG}
\ifnum \morefloats@mx> 33 \newinsert\bx@AH \expandafter\gdef\expandafter\@freelist\expandafter{\@freelist \@elt\bx@AH}
\ifnum \morefloats@mx> 34 \newinsert\bx@AI \expandafter\gdef\expandafter\@freelist\expandafter{\@freelist \@elt\bx@AI}
\ifnum \morefloats@mx> 35 \newinsert\bx@AJ \expandafter\gdef\expandafter\@freelist\expandafter{\@freelist \@elt\bx@AJ}
\ifnum \morefloats@mx> 36 \newinsert\bx@AK \expandafter\gdef\expandafter\@freelist\expandafter{\@freelist \@elt\bx@AK}
\ifnum \morefloats@mx> 37 \newinsert\bx@AL \expandafter\gdef\expandafter\@freelist\expandafter{\@freelist \@elt\bx@AL}
\ifnum \morefloats@mx> 38 \newinsert\bx@AM \expandafter\gdef\expandafter\@freelist\expandafter{\@freelist \@elt\bx@AM}
\ifnum \morefloats@mx> 39 \newinsert\bx@AN \expandafter\gdef\expandafter\@freelist\expandafter{\@freelist \@elt\bx@AN}
\ifnum \morefloats@mx> 40 \newinsert\bx@AO \expandafter\gdef\expandafter\@freelist\expandafter{\@freelist \@elt\bx@AO}
\ifnum \morefloats@mx> 41 \newinsert\bx@AP \expandafter\gdef\expandafter\@freelist\expandafter{\@freelist \@elt\bx@AP}
\ifnum \morefloats@mx> 42 \newinsert\bx@AQ \expandafter\gdef\expandafter\@freelist\expandafter{\@freelist \@elt\bx@AQ}
\ifnum \morefloats@mx> 43 \newinsert\bx@AR \expandafter\gdef\expandafter\@freelist\expandafter{\@freelist \@elt\bx@AR}
\ifnum \morefloats@mx> 44 \newinsert\bx@AS \expandafter\gdef\expandafter\@freelist\expandafter{\@freelist \@elt\bx@AS}
\ifnum \morefloats@mx> 45 \newinsert\bx@AT \expandafter\gdef\expandafter\@freelist\expandafter{\@freelist \@elt\bx@AT}
\ifnum \morefloats@mx> 46 \newinsert\bx@AU \expandafter\gdef\expandafter\@freelist\expandafter{\@freelist \@elt\bx@AU}
\ifnum \morefloats@mx> 47 \newinsert\bx@AV \expandafter\gdef\expandafter\@freelist\expandafter{\@freelist \@elt\bx@AV}
\ifnum \morefloats@mx> 48 \newinsert\bx@AW \expandafter\gdef\expandafter\@freelist\expandafter{\@freelist \@elt\bx@AW}
\ifnum \morefloats@mx> 49 \newinsert\bx@AX \expandafter\gdef\expandafter\@freelist\expandafter{\@freelist \@elt\bx@AX}
\ifnum \morefloats@mx> 50 \newinsert\bx@AY \expandafter\gdef\expandafter\@freelist\expandafter{\@freelist \@elt\bx@AY}
\ifnum \morefloats@mx> 51 \newinsert\bx@AZ \expandafter\gdef\expandafter\@freelist\expandafter{\@freelist \@elt\bx@AZ}
\ifnum \morefloats@mx> 52 \newinsert\bx@BA \expandafter\gdef\expandafter\@freelist\expandafter{\@freelist \@elt\bx@BA}
\ifnum \morefloats@mx> 53 \newinsert\bx@BB \expandafter\gdef\expandafter\@freelist\expandafter{\@freelist \@elt\bx@BB}
\ifnum \morefloats@mx> 54 \newinsert\bx@BC \expandafter\gdef\expandafter\@freelist\expandafter{\@freelist \@elt\bx@BC}
\ifnum \morefloats@mx> 55 \newinsert\bx@BD \expandafter\gdef\expandafter\@freelist\expandafter{\@freelist \@elt\bx@BD}
\ifnum \morefloats@mx> 56 \newinsert\bx@BE \expandafter\gdef\expandafter\@freelist\expandafter{\@freelist \@elt\bx@BE}
\ifnum \morefloats@mx> 57 \newinsert\bx@BF \expandafter\gdef\expandafter\@freelist\expandafter{\@freelist \@elt\bx@BF}
\ifnum \morefloats@mx> 58 \newinsert\bx@BG \expandafter\gdef\expandafter\@freelist\expandafter{\@freelist \@elt\bx@BG}
\ifnum \morefloats@mx> 59 \newinsert\bx@BH \expandafter\gdef\expandafter\@freelist\expandafter{\@freelist \@elt\bx@BH}
\ifnum \morefloats@mx> 60 \newinsert\bx@BI \expandafter\gdef\expandafter\@freelist\expandafter{\@freelist \@elt\bx@BI}
\ifnum \morefloats@mx> 61 \newinsert\bx@BJ \expandafter\gdef\expandafter\@freelist\expandafter{\@freelist \@elt\bx@BJ}
\ifnum \morefloats@mx> 62 \newinsert\bx@BK \expandafter\gdef\expandafter\@freelist\expandafter{\@freelist \@elt\bx@BK}
\ifnum \morefloats@mx> 63 \newinsert\bx@BL \expandafter\gdef\expandafter\@freelist\expandafter{\@freelist \@elt\bx@BL}
\ifnum \morefloats@mx> 64 \newinsert\bx@BM \expandafter\gdef\expandafter\@freelist\expandafter{\@freelist \@elt\bx@BM}
\ifnum \morefloats@mx> 65 \newinsert\bx@BN \expandafter\gdef\expandafter\@freelist\expandafter{\@freelist \@elt\bx@BN}
\ifnum \morefloats@mx> 66 \newinsert\bx@BO \expandafter\gdef\expandafter\@freelist\expandafter{\@freelist \@elt\bx@BO}
\ifnum \morefloats@mx> 67 \newinsert\bx@BP \expandafter\gdef\expandafter\@freelist\expandafter{\@freelist \@elt\bx@BP}
\ifnum \morefloats@mx> 68 \newinsert\bx@BQ \expandafter\gdef\expandafter\@freelist\expandafter{\@freelist \@elt\bx@BQ}
\ifnum \morefloats@mx> 69 \newinsert\bx@BR \expandafter\gdef\expandafter\@freelist\expandafter{\@freelist \@elt\bx@BR}
\ifnum \morefloats@mx> 70 \newinsert\bx@BS \expandafter\gdef\expandafter\@freelist\expandafter{\@freelist \@elt\bx@BS}
\ifnum \morefloats@mx> 71 \newinsert\bx@BT \expandafter\gdef\expandafter\@freelist\expandafter{\@freelist \@elt\bx@BT}
\ifnum \morefloats@mx> 72 \newinsert\bx@BU \expandafter\gdef\expandafter\@freelist\expandafter{\@freelist \@elt\bx@BU}
\ifnum \morefloats@mx> 73 \newinsert\bx@BV \expandafter\gdef\expandafter\@freelist\expandafter{\@freelist \@elt\bx@BV}
\ifnum \morefloats@mx> 74 \newinsert\bx@BW \expandafter\gdef\expandafter\@freelist\expandafter{\@freelist \@elt\bx@BW}
\ifnum \morefloats@mx> 75 \newinsert\bx@BX \expandafter\gdef\expandafter\@freelist\expandafter{\@freelist \@elt\bx@BX}
\ifnum \morefloats@mx> 76 \newinsert\bx@BY \expandafter\gdef\expandafter\@freelist\expandafter{\@freelist \@elt\bx@BY}
\ifnum \morefloats@mx> 77 \newinsert\bx@BZ \expandafter\gdef\expandafter\@freelist\expandafter{\@freelist \@elt\bx@BZ}
\ifnum \morefloats@mx> 78 \newinsert\bx@CA \expandafter\gdef\expandafter\@freelist\expandafter{\@freelist \@elt\bx@CA}
\ifnum \morefloats@mx> 79 \newinsert\bx@CB \expandafter\gdef\expandafter\@freelist\expandafter{\@freelist \@elt\bx@CB}
\ifnum \morefloats@mx> 80 \newinsert\bx@CC \expandafter\gdef\expandafter\@freelist\expandafter{\@freelist \@elt\bx@CC}
\ifnum \morefloats@mx> 81 \newinsert\bx@CD \expandafter\gdef\expandafter\@freelist\expandafter{\@freelist \@elt\bx@CD}
\ifnum \morefloats@mx> 82 \newinsert\bx@CE \expandafter\gdef\expandafter\@freelist\expandafter{\@freelist \@elt\bx@CE}
\ifnum \morefloats@mx> 83 \newinsert\bx@CF \expandafter\gdef\expandafter\@freelist\expandafter{\@freelist \@elt\bx@CF}
\ifnum \morefloats@mx> 84 \newinsert\bx@CG \expandafter\gdef\expandafter\@freelist\expandafter{\@freelist \@elt\bx@CG}
\ifnum \morefloats@mx> 85 \newinsert\bx@CH \expandafter\gdef\expandafter\@freelist\expandafter{\@freelist \@elt\bx@CH}
\ifnum \morefloats@mx> 86 \newinsert\bx@CI \expandafter\gdef\expandafter\@freelist\expandafter{\@freelist \@elt\bx@CI}
\ifnum \morefloats@mx> 87 \newinsert\bx@CJ \expandafter\gdef\expandafter\@freelist\expandafter{\@freelist \@elt\bx@CJ}
\ifnum \morefloats@mx> 88 \newinsert\bx@CK \expandafter\gdef\expandafter\@freelist\expandafter{\@freelist \@elt\bx@CK}
\ifnum \morefloats@mx> 89 \newinsert\bx@CL \expandafter\gdef\expandafter\@freelist\expandafter{\@freelist \@elt\bx@CL}
\ifnum \morefloats@mx> 90 \newinsert\bx@CM \expandafter\gdef\expandafter\@freelist\expandafter{\@freelist \@elt\bx@CM}
\ifnum \morefloats@mx> 91 \newinsert\bx@CN \expandafter\gdef\expandafter\@freelist\expandafter{\@freelist \@elt\bx@CN}
\ifnum \morefloats@mx> 92 \newinsert\bx@CO \expandafter\gdef\expandafter\@freelist\expandafter{\@freelist \@elt\bx@CO}
\ifnum \morefloats@mx> 93 \newinsert\bx@CP \expandafter\gdef\expandafter\@freelist\expandafter{\@freelist \@elt\bx@CP}
\ifnum \morefloats@mx> 94 \newinsert\bx@CQ \expandafter\gdef\expandafter\@freelist\expandafter{\@freelist \@elt\bx@CQ}
\ifnum \morefloats@mx> 95 \newinsert\bx@CR \expandafter\gdef\expandafter\@freelist\expandafter{\@freelist \@elt\bx@CR}
\ifnum \morefloats@mx> 96 \newinsert\bx@CS \expandafter\gdef\expandafter\@freelist\expandafter{\@freelist \@elt\bx@CS}
\ifnum \morefloats@mx> 97 \newinsert\bx@CT \expandafter\gdef\expandafter\@freelist\expandafter{\@freelist \@elt\bx@CT}
\ifnum \morefloats@mx> 98 \newinsert\bx@CU \expandafter\gdef\expandafter\@freelist\expandafter{\@freelist \@elt\bx@CU}
\ifnum \morefloats@mx> 99 \newinsert\bx@CV \expandafter\gdef\expandafter\@freelist\expandafter{\@freelist \@elt\bx@CV}
\ifnum \morefloats@mx>100 \newinsert\bx@CW \expandafter\gdef\expandafter\@freelist\expandafter{\@freelist \@elt\bx@CW}
\ifnum \morefloats@mx>101 \newinsert\bx@CX \expandafter\gdef\expandafter\@freelist\expandafter{\@freelist \@elt\bx@CX}
\ifnum \morefloats@mx>102 \newinsert\bx@CY \expandafter\gdef\expandafter\@freelist\expandafter{\@freelist \@elt\bx@CY}
\ifnum \morefloats@mx>103 \newinsert\bx@CZ \expandafter\gdef\expandafter\@freelist\expandafter{\@freelist \@elt\bx@CZ}
\ifnum \morefloats@mx>104 \newinsert\bx@DA \expandafter\gdef\expandafter\@freelist\expandafter{\@freelist \@elt\bx@DA}
\ifnum \morefloats@mx>105 \newinsert\bx@DB \expandafter\gdef\expandafter\@freelist\expandafter{\@freelist \@elt\bx@DB}
\ifnum \morefloats@mx>106 \newinsert\bx@DC \expandafter\gdef\expandafter\@freelist\expandafter{\@freelist \@elt\bx@DC}
\ifnum \morefloats@mx>107 \newinsert\bx@DD \expandafter\gdef\expandafter\@freelist\expandafter{\@freelist \@elt\bx@DD}
\ifnum \morefloats@mx>108 \newinsert\bx@DE \expandafter\gdef\expandafter\@freelist\expandafter{\@freelist \@elt\bx@DE}
\ifnum \morefloats@mx>109 \newinsert\bx@DF \expandafter\gdef\expandafter\@freelist\expandafter{\@freelist \@elt\bx@DF}
\ifnum \morefloats@mx>110 \newinsert\bx@DG \expandafter\gdef\expandafter\@freelist\expandafter{\@freelist \@elt\bx@DG}
\ifnum \morefloats@mx>111 \newinsert\bx@DH \expandafter\gdef\expandafter\@freelist\expandafter{\@freelist \@elt\bx@DH}
\ifnum \morefloats@mx>112 \newinsert\bx@DI \expandafter\gdef\expandafter\@freelist\expandafter{\@freelist \@elt\bx@DI}
\ifnum \morefloats@mx>113 \newinsert\bx@DJ \expandafter\gdef\expandafter\@freelist\expandafter{\@freelist \@elt\bx@DJ}
\ifnum \morefloats@mx>114 \newinsert\bx@DK \expandafter\gdef\expandafter\@freelist\expandafter{\@freelist \@elt\bx@DK}
\ifnum \morefloats@mx>115 \newinsert\bx@DL \expandafter\gdef\expandafter\@freelist\expandafter{\@freelist \@elt\bx@DL}
\ifnum \morefloats@mx>116 \newinsert\bx@DM \expandafter\gdef\expandafter\@freelist\expandafter{\@freelist \@elt\bx@DM}
\ifnum \morefloats@mx>117 \newinsert\bx@DN \expandafter\gdef\expandafter\@freelist\expandafter{\@freelist \@elt\bx@DN}
\ifnum \morefloats@mx>118 \newinsert\bx@DO \expandafter\gdef\expandafter\@freelist\expandafter{\@freelist \@elt\bx@DO}
\ifnum \morefloats@mx>119 \newinsert\bx@DP \expandafter\gdef\expandafter\@freelist\expandafter{\@freelist \@elt\bx@DP}
\ifnum \morefloats@mx>120 \newinsert\bx@DQ \expandafter\gdef\expandafter\@freelist\expandafter{\@freelist \@elt\bx@DQ}
\ifnum \morefloats@mx>121 \newinsert\bx@DR \expandafter\gdef\expandafter\@freelist\expandafter{\@freelist \@elt\bx@DR}
\ifnum \morefloats@mx>122 \newinsert\bx@DS \expandafter\gdef\expandafter\@freelist\expandafter{\@freelist \@elt\bx@DS}
\ifnum \morefloats@mx>123 \newinsert\bx@DT \expandafter\gdef\expandafter\@freelist\expandafter{\@freelist \@elt\bx@DT}
\ifnum \morefloats@mx>124 \newinsert\bx@DU \expandafter\gdef\expandafter\@freelist\expandafter{\@freelist \@elt\bx@DU}
\ifnum \morefloats@mx>125 \newinsert\bx@DV \expandafter\gdef\expandafter\@freelist\expandafter{\@freelist \@elt\bx@DV}
\ifnum \morefloats@mx>126 \newinsert\bx@DW \expandafter\gdef\expandafter\@freelist\expandafter{\@freelist \@elt\bx@DW}
\ifnum \morefloats@mx>127 \newinsert\bx@DX \expandafter\gdef\expandafter\@freelist\expandafter{\@freelist \@elt\bx@DX}
\ifnum \morefloats@mx>128 \newinsert\bx@DY \expandafter\gdef\expandafter\@freelist\expandafter{\@freelist \@elt\bx@DY}
\ifnum \morefloats@mx>129 \newinsert\bx@DZ \expandafter\gdef\expandafter\@freelist\expandafter{\@freelist \@elt\bx@DZ}
\ifnum \morefloats@mx>130 \newinsert\bx@EA \expandafter\gdef\expandafter\@freelist\expandafter{\@freelist \@elt\bx@EA}
\ifnum \morefloats@mx>131 \newinsert\bx@EB \expandafter\gdef\expandafter\@freelist\expandafter{\@freelist \@elt\bx@EB}
\ifnum \morefloats@mx>132 \newinsert\bx@EC \expandafter\gdef\expandafter\@freelist\expandafter{\@freelist \@elt\bx@EC}
\ifnum \morefloats@mx>133 \newinsert\bx@ED \expandafter\gdef\expandafter\@freelist\expandafter{\@freelist \@elt\bx@ED}
\ifnum \morefloats@mx>134 \newinsert\bx@EE \expandafter\gdef\expandafter\@freelist\expandafter{\@freelist \@elt\bx@EE}
\ifnum \morefloats@mx>135 \newinsert\bx@EF \expandafter\gdef\expandafter\@freelist\expandafter{\@freelist \@elt\bx@EF}
\ifnum \morefloats@mx>136 \newinsert\bx@EG \expandafter\gdef\expandafter\@freelist\expandafter{\@freelist \@elt\bx@EG}
\ifnum \morefloats@mx>137 \newinsert\bx@EH \expandafter\gdef\expandafter\@freelist\expandafter{\@freelist \@elt\bx@EH}
\ifnum \morefloats@mx>138 \newinsert\bx@EI \expandafter\gdef\expandafter\@freelist\expandafter{\@freelist \@elt\bx@EI}
\ifnum \morefloats@mx>139 \newinsert\bx@EJ \expandafter\gdef\expandafter\@freelist\expandafter{\@freelist \@elt\bx@EJ}
\ifnum \morefloats@mx>140 \newinsert\bx@EK \expandafter\gdef\expandafter\@freelist\expandafter{\@freelist \@elt\bx@EK}
\ifnum \morefloats@mx>141 \newinsert\bx@EL \expandafter\gdef\expandafter\@freelist\expandafter{\@freelist \@elt\bx@EL}
\ifnum \morefloats@mx>142 \newinsert\bx@EM \expandafter\gdef\expandafter\@freelist\expandafter{\@freelist \@elt\bx@EM}
\ifnum \morefloats@mx>143 \newinsert\bx@EN \expandafter\gdef\expandafter\@freelist\expandafter{\@freelist \@elt\bx@EN}
\ifnum \morefloats@mx>144 \newinsert\bx@EO \expandafter\gdef\expandafter\@freelist\expandafter{\@freelist \@elt\bx@EO}
\ifnum \morefloats@mx>145 \newinsert\bx@EP \expandafter\gdef\expandafter\@freelist\expandafter{\@freelist \@elt\bx@EP}
\ifnum \morefloats@mx>146 \newinsert\bx@EQ \expandafter\gdef\expandafter\@freelist\expandafter{\@freelist \@elt\bx@EQ}
\ifnum \morefloats@mx>147 \newinsert\bx@ER \expandafter\gdef\expandafter\@freelist\expandafter{\@freelist \@elt\bx@ER}
\ifnum \morefloats@mx>148 \newinsert\bx@ES \expandafter\gdef\expandafter\@freelist\expandafter{\@freelist \@elt\bx@ES}
\ifnum \morefloats@mx>149 \newinsert\bx@ET \expandafter\gdef\expandafter\@freelist\expandafter{\@freelist \@elt\bx@ET}
\ifnum \morefloats@mx>150 \newinsert\bx@EU \expandafter\gdef\expandafter\@freelist\expandafter{\@freelist \@elt\bx@EU}
\ifnum \morefloats@mx>151 \newinsert\bx@EV \expandafter\gdef\expandafter\@freelist\expandafter{\@freelist \@elt\bx@EV}
\ifnum \morefloats@mx>152 \newinsert\bx@EW \expandafter\gdef\expandafter\@freelist\expandafter{\@freelist \@elt\bx@EW}
\ifnum \morefloats@mx>153 \newinsert\bx@EX \expandafter\gdef\expandafter\@freelist\expandafter{\@freelist \@elt\bx@EX}
\ifnum \morefloats@mx>154 \newinsert\bx@EY \expandafter\gdef\expandafter\@freelist\expandafter{\@freelist \@elt\bx@EY}
\ifnum \morefloats@mx>155 \newinsert\bx@EZ \expandafter\gdef\expandafter\@freelist\expandafter{\@freelist \@elt\bx@EZ}
\ifnum \morefloats@mx>156 \newinsert\bx@FA \expandafter\gdef\expandafter\@freelist\expandafter{\@freelist \@elt\bx@FA}
\ifnum \morefloats@mx>157 \newinsert\bx@FB \expandafter\gdef\expandafter\@freelist\expandafter{\@freelist \@elt\bx@FB}
\ifnum \morefloats@mx>158 \newinsert\bx@FC \expandafter\gdef\expandafter\@freelist\expandafter{\@freelist \@elt\bx@FC}
\ifnum \morefloats@mx>159 \newinsert\bx@FD \expandafter\gdef\expandafter\@freelist\expandafter{\@freelist \@elt\bx@FD}
\ifnum \morefloats@mx>160 \newinsert\bx@FE \expandafter\gdef\expandafter\@freelist\expandafter{\@freelist \@elt\bx@FE}
\ifnum \morefloats@mx>161 \newinsert\bx@FF \expandafter\gdef\expandafter\@freelist\expandafter{\@freelist \@elt\bx@FF}
\ifnum \morefloats@mx>162 \newinsert\bx@FG \expandafter\gdef\expandafter\@freelist\expandafter{\@freelist \@elt\bx@FG}
\ifnum \morefloats@mx>163 \newinsert\bx@FH \expandafter\gdef\expandafter\@freelist\expandafter{\@freelist \@elt\bx@FH}
\ifnum \morefloats@mx>164 \newinsert\bx@FI \expandafter\gdef\expandafter\@freelist\expandafter{\@freelist \@elt\bx@FI}
\ifnum \morefloats@mx>165 \newinsert\bx@FJ \expandafter\gdef\expandafter\@freelist\expandafter{\@freelist \@elt\bx@FJ}
\ifnum \morefloats@mx>166 \newinsert\bx@FK \expandafter\gdef\expandafter\@freelist\expandafter{\@freelist \@elt\bx@FK}
\ifnum \morefloats@mx>167 \newinsert\bx@FL \expandafter\gdef\expandafter\@freelist\expandafter{\@freelist \@elt\bx@FL}
\ifnum \morefloats@mx>168 \newinsert\bx@FM \expandafter\gdef\expandafter\@freelist\expandafter{\@freelist \@elt\bx@FM}
\ifnum \morefloats@mx>169 \newinsert\bx@FN \expandafter\gdef\expandafter\@freelist\expandafter{\@freelist \@elt\bx@FN}
\ifnum \morefloats@mx>170 \newinsert\bx@FO \expandafter\gdef\expandafter\@freelist\expandafter{\@freelist \@elt\bx@FO}
\ifnum \morefloats@mx>171 \newinsert\bx@FP \expandafter\gdef\expandafter\@freelist\expandafter{\@freelist \@elt\bx@FP}
\ifnum \morefloats@mx>172 \newinsert\bx@FQ \expandafter\gdef\expandafter\@freelist\expandafter{\@freelist \@elt\bx@FQ}
\ifnum \morefloats@mx>173 \newinsert\bx@FR \expandafter\gdef\expandafter\@freelist\expandafter{\@freelist \@elt\bx@FR}
\ifnum \morefloats@mx>174 \newinsert\bx@FS \expandafter\gdef\expandafter\@freelist\expandafter{\@freelist \@elt\bx@FS}
\ifnum \morefloats@mx>175 \newinsert\bx@FT \expandafter\gdef\expandafter\@freelist\expandafter{\@freelist \@elt\bx@FT}
\ifnum \morefloats@mx>176 \newinsert\bx@FU \expandafter\gdef\expandafter\@freelist\expandafter{\@freelist \@elt\bx@FU}
\ifnum \morefloats@mx>177 \newinsert\bx@FV \expandafter\gdef\expandafter\@freelist\expandafter{\@freelist \@elt\bx@FV}
\ifnum \morefloats@mx>178 \newinsert\bx@FW \expandafter\gdef\expandafter\@freelist\expandafter{\@freelist \@elt\bx@FW}
\ifnum \morefloats@mx>179 \newinsert\bx@FX \expandafter\gdef\expandafter\@freelist\expandafter{\@freelist \@elt\bx@FX}
\ifnum \morefloats@mx>180 \newinsert\bx@FY \expandafter\gdef\expandafter\@freelist\expandafter{\@freelist \@elt\bx@FY}
\ifnum \morefloats@mx>181 \newinsert\bx@FZ \expandafter\gdef\expandafter\@freelist\expandafter{\@freelist \@elt\bx@FZ}
\ifnum \morefloats@mx>182 \newinsert\bx@GA \expandafter\gdef\expandafter\@freelist\expandafter{\@freelist \@elt\bx@GA}
\ifnum \morefloats@mx>183 \newinsert\bx@GB \expandafter\gdef\expandafter\@freelist\expandafter{\@freelist \@elt\bx@GB}
\ifnum \morefloats@mx>184 \newinsert\bx@GC \expandafter\gdef\expandafter\@freelist\expandafter{\@freelist \@elt\bx@GC}
\ifnum \morefloats@mx>185 \newinsert\bx@GD \expandafter\gdef\expandafter\@freelist\expandafter{\@freelist \@elt\bx@GD}
\ifnum \morefloats@mx>186 \newinsert\bx@GE \expandafter\gdef\expandafter\@freelist\expandafter{\@freelist \@elt\bx@GE}
\ifnum \morefloats@mx>187 \newinsert\bx@GF \expandafter\gdef\expandafter\@freelist\expandafter{\@freelist \@elt\bx@GF}
\ifnum \morefloats@mx>188 \newinsert\bx@GG \expandafter\gdef\expandafter\@freelist\expandafter{\@freelist \@elt\bx@GG}
\ifnum \morefloats@mx>189 \newinsert\bx@GH \expandafter\gdef\expandafter\@freelist\expandafter{\@freelist \@elt\bx@GH}
\ifnum \morefloats@mx>190 \newinsert\bx@GI \expandafter\gdef\expandafter\@freelist\expandafter{\@freelist \@elt\bx@GI}
\ifnum \morefloats@mx>191 \newinsert\bx@GJ \expandafter\gdef\expandafter\@freelist\expandafter{\@freelist \@elt\bx@GJ}
\ifnum \morefloats@mx>192 \newinsert\bx@GK \expandafter\gdef\expandafter\@freelist\expandafter{\@freelist \@elt\bx@GK}
\ifnum \morefloats@mx>193 \newinsert\bx@GL \expandafter\gdef\expandafter\@freelist\expandafter{\@freelist \@elt\bx@GL}
\ifnum \morefloats@mx>194 \newinsert\bx@GM \expandafter\gdef\expandafter\@freelist\expandafter{\@freelist \@elt\bx@GM}
\ifnum \morefloats@mx>195 \newinsert\bx@GN \expandafter\gdef\expandafter\@freelist\expandafter{\@freelist \@elt\bx@GN}
\ifnum \morefloats@mx>196 \newinsert\bx@GO \expandafter\gdef\expandafter\@freelist\expandafter{\@freelist \@elt\bx@GO}
\ifnum \morefloats@mx>197 \newinsert\bx@GP \expandafter\gdef\expandafter\@freelist\expandafter{\@freelist \@elt\bx@GP}
\ifnum \morefloats@mx>198 \newinsert\bx@GQ \expandafter\gdef\expandafter\@freelist\expandafter{\@freelist \@elt\bx@GQ}
\ifnum \morefloats@mx>199 \newinsert\bx@GR \expandafter\gdef\expandafter\@freelist\expandafter{\@freelist \@elt\bx@GR}
\ifnum \morefloats@mx>200 \newinsert\bx@GS \expandafter\gdef\expandafter\@freelist\expandafter{\@freelist \@elt\bx@GS}
\ifnum \morefloats@mx>201 \newinsert\bx@GT \expandafter\gdef\expandafter\@freelist\expandafter{\@freelist \@elt\bx@GT}
\ifnum \morefloats@mx>202 \newinsert\bx@GU \expandafter\gdef\expandafter\@freelist\expandafter{\@freelist \@elt\bx@GU}
\ifnum \morefloats@mx>203 \newinsert\bx@GV \expandafter\gdef\expandafter\@freelist\expandafter{\@freelist \@elt\bx@GV}
\ifnum \morefloats@mx>204 \newinsert\bx@GW \expandafter\gdef\expandafter\@freelist\expandafter{\@freelist \@elt\bx@GW}
\ifnum \morefloats@mx>205 \newinsert\bx@GX \expandafter\gdef\expandafter\@freelist\expandafter{\@freelist \@elt\bx@GX}
\ifnum \morefloats@mx>206 \newinsert\bx@GY \expandafter\gdef\expandafter\@freelist\expandafter{\@freelist \@elt\bx@GY}
\ifnum \morefloats@mx>207 \newinsert\bx@GZ \expandafter\gdef\expandafter\@freelist\expandafter{\@freelist \@elt\bx@GZ}
\ifnum \morefloats@mx>208 \newinsert\bx@HA \expandafter\gdef\expandafter\@freelist\expandafter{\@freelist \@elt\bx@HA}
\ifnum \morefloats@mx>209 \newinsert\bx@HB \expandafter\gdef\expandafter\@freelist\expandafter{\@freelist \@elt\bx@HB}
\ifnum \morefloats@mx>210 \newinsert\bx@HC \expandafter\gdef\expandafter\@freelist\expandafter{\@freelist \@elt\bx@HC}
\ifnum \morefloats@mx>211 \newinsert\bx@HD \expandafter\gdef\expandafter\@freelist\expandafter{\@freelist \@elt\bx@HD}
\ifnum \morefloats@mx>212 \newinsert\bx@HE \expandafter\gdef\expandafter\@freelist\expandafter{\@freelist \@elt\bx@HE}
\ifnum \morefloats@mx>213 \newinsert\bx@HF \expandafter\gdef\expandafter\@freelist\expandafter{\@freelist \@elt\bx@HF}
\ifnum \morefloats@mx>214 \newinsert\bx@HG \expandafter\gdef\expandafter\@freelist\expandafter{\@freelist \@elt\bx@HG}
\ifnum \morefloats@mx>215 \newinsert\bx@HH \expandafter\gdef\expandafter\@freelist\expandafter{\@freelist \@elt\bx@HH}
\ifnum \morefloats@mx>216 \newinsert\bx@HI \expandafter\gdef\expandafter\@freelist\expandafter{\@freelist \@elt\bx@HI}
\ifnum \morefloats@mx>217 \newinsert\bx@HJ \expandafter\gdef\expandafter\@freelist\expandafter{\@freelist \@elt\bx@HJ}
\ifnum \morefloats@mx>218 \newinsert\bx@HK \expandafter\gdef\expandafter\@freelist\expandafter{\@freelist \@elt\bx@HK}
\ifnum \morefloats@mx>219 \newinsert\bx@HL \expandafter\gdef\expandafter\@freelist\expandafter{\@freelist \@elt\bx@HL}
\ifnum \morefloats@mx>220 \newinsert\bx@HM \expandafter\gdef\expandafter\@freelist\expandafter{\@freelist \@elt\bx@HM}
\ifnum \morefloats@mx>221 \newinsert\bx@HN \expandafter\gdef\expandafter\@freelist\expandafter{\@freelist \@elt\bx@HN}
\ifnum \morefloats@mx>222 \newinsert\bx@HO \expandafter\gdef\expandafter\@freelist\expandafter{\@freelist \@elt\bx@HO}
\ifnum \morefloats@mx>223 \newinsert\bx@HP \expandafter\gdef\expandafter\@freelist\expandafter{\@freelist \@elt\bx@HP}
\ifnum \morefloats@mx>224 \newinsert\bx@HQ \expandafter\gdef\expandafter\@freelist\expandafter{\@freelist \@elt\bx@HQ}
\ifnum \morefloats@mx>225 \newinsert\bx@HR \expandafter\gdef\expandafter\@freelist\expandafter{\@freelist \@elt\bx@HR}
\ifnum \morefloats@mx>226 \newinsert\bx@HS \expandafter\gdef\expandafter\@freelist\expandafter{\@freelist \@elt\bx@HS}
\ifnum \morefloats@mx>227 \newinsert\bx@HT \expandafter\gdef\expandafter\@freelist\expandafter{\@freelist \@elt\bx@HT}
\ifnum \morefloats@mx>228 \newinsert\bx@HU \expandafter\gdef\expandafter\@freelist\expandafter{\@freelist \@elt\bx@HU}
\ifnum \morefloats@mx>229 \newinsert\bx@HV \expandafter\gdef\expandafter\@freelist\expandafter{\@freelist \@elt\bx@HV}
\ifnum \morefloats@mx>230 \newinsert\bx@HW \expandafter\gdef\expandafter\@freelist\expandafter{\@freelist \@elt\bx@HW}
\ifnum \morefloats@mx>231 \newinsert\bx@HX \expandafter\gdef\expandafter\@freelist\expandafter{\@freelist \@elt\bx@HX}
\ifnum \morefloats@mx>232 \newinsert\bx@HY \expandafter\gdef\expandafter\@freelist\expandafter{\@freelist \@elt\bx@HY}
\ifnum \morefloats@mx>233 \newinsert\bx@HZ \expandafter\gdef\expandafter\@freelist\expandafter{\@freelist \@elt\bx@HZ}
\ifnum \morefloats@mx>234 \newinsert\bx@IA \expandafter\gdef\expandafter\@freelist\expandafter{\@freelist \@elt\bx@IA}
\ifnum \morefloats@mx>235 \newinsert\bx@IB \expandafter\gdef\expandafter\@freelist\expandafter{\@freelist \@elt\bx@IB}
\ifnum \morefloats@mx>236 \newinsert\bx@IC \expandafter\gdef\expandafter\@freelist\expandafter{\@freelist \@elt\bx@IC}
\ifnum \morefloats@mx>237 \newinsert\bx@ID \expandafter\gdef\expandafter\@freelist\expandafter{\@freelist \@elt\bx@ID}
\ifnum \morefloats@mx>238 \newinsert\bx@IE \expandafter\gdef\expandafter\@freelist\expandafter{\@freelist \@elt\bx@IE}
\ifnum \morefloats@mx>239 \newinsert\bx@IF \expandafter\gdef\expandafter\@freelist\expandafter{\@freelist \@elt\bx@IF}
\ifnum \morefloats@mx>240 \newinsert\bx@IG \expandafter\gdef\expandafter\@freelist\expandafter{\@freelist \@elt\bx@IG}
\ifnum \morefloats@mx>241 \newinsert\bx@IH \expandafter\gdef\expandafter\@freelist\expandafter{\@freelist \@elt\bx@IH}
\ifnum \morefloats@mx>242 \newinsert\bx@II \expandafter\gdef\expandafter\@freelist\expandafter{\@freelist \@elt\bx@II}
\ifnum \morefloats@mx>243 \newinsert\bx@IJ \expandafter\gdef\expandafter\@freelist\expandafter{\@freelist \@elt\bx@IJ}
\ifnum \morefloats@mx>244 \newinsert\bx@IK \expandafter\gdef\expandafter\@freelist\expandafter{\@freelist \@elt\bx@IK}
\ifnum \morefloats@mx>245 \newinsert\bx@IL \expandafter\gdef\expandafter\@freelist\expandafter{\@freelist \@elt\bx@IL}
\ifnum \morefloats@mx>246 \newinsert\bx@IM \expandafter\gdef\expandafter\@freelist\expandafter{\@freelist \@elt\bx@IM}
\ifnum \morefloats@mx>247 \newinsert\bx@IN \expandafter\gdef\expandafter\@freelist\expandafter{\@freelist \@elt\bx@IN}
\ifnum \morefloats@mx>248 \newinsert\bx@IO \expandafter\gdef\expandafter\@freelist\expandafter{\@freelist \@elt\bx@IO}
\ifnum \morefloats@mx>249 \newinsert\bx@IP \expandafter\gdef\expandafter\@freelist\expandafter{\@freelist \@elt\bx@IP}
\ifnum \morefloats@mx>250 \newinsert\bx@IQ \expandafter\gdef\expandafter\@freelist\expandafter{\@freelist \@elt\bx@IQ}
\ifnum \morefloats@mx>251 \newinsert\bx@IR \expandafter\gdef\expandafter\@freelist\expandafter{\@freelist \@elt\bx@IR}
\ifnum \morefloats@mx>252 \newinsert\bx@IS \expandafter\gdef\expandafter\@freelist\expandafter{\@freelist \@elt\bx@IS}
\ifnum \morefloats@mx>253 \newinsert\bx@IT \expandafter\gdef\expandafter\@freelist\expandafter{\@freelist \@elt\bx@IT}
\ifnum \morefloats@mx>254 \newinsert\bx@IU \expandafter\gdef\expandafter\@freelist\expandafter{\@freelist \@elt\bx@IU}
\ifnum \morefloats@mx>255 \newinsert\bx@IV \expandafter\gdef\expandafter\@freelist\expandafter{\@freelist \@elt\bx@IV}
%    \end{macrocode}
%
% \newpage
%
%    \begin{macrocode}
\ifnum \morefloats@mx>256\relax%
  \PackageError{morefloats}{Too many floats called for}{%
    You requested more than 256 floats.\MessageBreak%
    (\morefloats@mx\space to be precise.)\MessageBreak%
    LaTeX before 2015 could not process\MessageBreak%
    more than 256 floats, therefore the morefloats\MessageBreak%
    package only provides 256 floats.\MessageBreak%
    If you need more floats,\MessageBreak%
    update to a current (>=2015) LaTeX distribution.\MessageBreak%
    I expected LaTeX (prior 2015) to run out of dimensions\MessageBreak%
    or memory long before reaching 256 floats anyway.\MessageBreak%
   }%
\fi \fi \fi \fi \fi \fi \fi \fi \fi \fi \fi \fi \fi \fi \fi \fi \fi \fi
\fi \fi \fi \fi \fi \fi \fi \fi \fi \fi \fi \fi \fi \fi \fi \fi \fi \fi
\fi \fi \fi \fi \fi \fi \fi \fi \fi \fi \fi \fi \fi \fi \fi \fi \fi \fi
\fi \fi \fi \fi \fi \fi \fi \fi \fi \fi \fi \fi \fi \fi \fi \fi \fi \fi
\fi \fi \fi \fi \fi \fi \fi \fi \fi \fi \fi \fi \fi \fi \fi \fi \fi \fi
\fi \fi \fi \fi \fi \fi \fi \fi \fi \fi \fi \fi \fi \fi \fi \fi \fi \fi
\fi \fi \fi \fi \fi \fi \fi \fi \fi \fi \fi \fi \fi \fi \fi \fi \fi \fi
\fi \fi \fi \fi \fi \fi \fi \fi \fi \fi \fi \fi \fi \fi \fi \fi \fi \fi
\fi \fi \fi \fi \fi \fi \fi \fi \fi \fi \fi \fi \fi \fi \fi \fi \fi \fi
\fi \fi \fi \fi \fi \fi \fi \fi \fi \fi \fi \fi \fi \fi \fi \fi \fi \fi
\fi \fi \fi \fi \fi \fi \fi \fi \fi \fi \fi \fi \fi \fi \fi \fi \fi \fi
\fi \fi \fi \fi \fi \fi \fi \fi \fi \fi \fi \fi \fi \fi \fi \fi \fi \fi
\fi \fi \fi \fi \fi \fi \fi \fi \fi \fi \fi \fi \fi \fi \fi \fi \fi \fi
\fi \fi \fi \fi \fi

%    \end{macrocode}
%
%    \begin{macrocode}
%</package>
%    \end{macrocode}
%
% \end{landscape}
% \newpage
%
% \section{Installation}
%
% \subsection{Downloads\label{ss:Downloads}}
%
% Everything is available at \url{https://www.ctan.org},
% but may need additional packages themselves.\\
%
% \DescribeMacro{morefloats.dtx}
% For unpacking the |morefloats.dtx| file and constructing the documentation it is required:
% \begin{description}
% \item[-] \TeX Format \LaTeXe{}: \url{https://www.CTAN.org}
%
% \item[-] document class \xclass{ltxdoc}, 2015/03/26, v2.0w,
%   \url{https://www.ctan.org/pkg/ltxdoc}
%
% \item[-] package \xpackage{fontenc}, 2005/09/27, v1.99g,
%   \url{https://ctan.org/pkg/fontenc}
%
% \item[-] package \xpackage{pdflscape}, 2008/08/11, v0.10,
%   \url{https://ctan.org/pkg/pdflscape}
%
% \item[-] package \xpackage{holtxdoc}, 2012/03/21, v0.24,
%   \url{https://ctan.org/pkg/holtxdoc}
%
% \item[-] package \xpackage{hypdoc}, 2011/08/19, v1.11,
%   \url{https://ctan.org/pkg/hypdoc}
% \end{description}
%
% \DescribeMacro{morefloats.sty}
% The \texttt{morefloats.sty} for \LaTeXe{} \hbox{(i.\,e. each} document using
% the \xpackage{morefloats} package) requires:
% \begin{description}
% \item[-] \TeX Format \LaTeXe{}, \url{https://www.CTAN.org/}
%
% \item[-] package \xpackage{kvoptions}, 2011/06/30, v3.11,
%   \url{https://ctan.org/pkg/kvoptions}
%
% \item[-] package \xpackage{ifetex}, 2011/12/15, v1.2,
%   \url{https://ctan.org/pkg/ifetex}, is used in some cases
% \end{description}
%
% \DescribeMacro{regstats}
% \DescribeMacro{regcount}
% To check the number of used registers it was mentioned:
% \begin{description}
% \item[-] package \xpackage{regstats}, \url{https://ctan.org/pkg/regstats}
% \item[-] package \xpackage{regcount}, \url{https://ctan.org/pkg/regcount}
% \end{description}
%
% \DescribeMacro{Oberdiek}
% \DescribeMacro{holtxdoc}
% \DescribeMacro{hypdoc}
% All packages of \textsc{Heiko Oberdiek}'s bundle `oberdiek'
% (especially \xpackage{holtxdoc}, \xpackage{hypdoc}, and \xpackage{kvoptions})
% are also available in a TDS compliant ZIP archive:\\
% \url{http://mirror.ctan.org/install/macros/latex/contrib/oberdiek.tds.zip}.\\
% It is probably best to download and use this, because the packages in there
% are quite probably both recent and compatible among themselves.\\
%
% \DescribeMacro{hyperref}
% \noindent \xpackage{hyperref} is not included in that bundle and needs to be
% downloaded separately,\\
% \url{http://mirror.ctan.org/install/macros/latex/contrib/hyperref.tds.zip}.\\
%
% \DescribeMacro{M\"{u}nch}
% A hyperlinked list of my (other) packages can be found at
% \url{https://www.ctan.org/author/muench-hm}.\\
%
% \subsection{Package, unpacking TDS}
% \paragraph{Package.} This package is available on \url{https://www.CTAN.org}.
% \begin{description}
% \item[\url{http://mirror.ctan.org/macros/latex/contrib/morefloats/morefloats.dtx}]\hspace*{0.1cm}
%       The source file.
% \item[\url{http://mirror.ctan.org/macros/latex/contrib/morefloats/morefloats.pdf}]\hspace*{0.1cm}
%       The documentation.
% \item[\url{http://mirror.ctan.org/macros/latex/contrib/morefloats/README}]\hspace*{0.1cm}\\
%       \hspace*{1em}The README file.
% \end{description}
%
% \noindent There is also a |morefloats.tds.zip| available:
% \begin{description}
% \item[\url{http://mirror.ctan.org/install/macros/latex/contrib/morefloats.tds.zip}]\hspace*{0.1cm}
%       Everything in TDS compliant, compiled format.
% \end{description}
% which additionally contains\\
% \begin{tabular}{ll}
% morefloats.ins & The installation file.\\
% morefloats.drv & The driver to generate the documentation.\\
% morefloats.sty & The \xext{sty}le file.\\
% morefloats-example.tex & The example file.\\
% morefloats-example.pdf & The compiled example file.
% \end{tabular}
%
% \bigskip
%
% \noindent For required other packages, please see the preceding subsection.
%
% \paragraph{Unpacking.} The  \xfile{.dtx} file is a self-extracting
% \docstrip{} archive. The files are extracted by running the
% \xfile{.dtx} through \plainTeX{}:
% \begin{quote}
%   \verb|tex morefloats.dtx|
% \end{quote}
%
% About generating the documentation see paragraph~\ref{GenDoc} below.\\
%
% \paragraph{TDS.} Now the different files must be moved into
% the different directories in your installation TDS tree
% (also known as \xfile{texmf} tree):
% \begin{quote}
% \def\t{^^A
% \begin{tabular}{@{}>{\ttfamily}l@{ $\rightarrow$ }>{\ttfamily}l@{}}
%   morefloats.sty & tex/latex/morefloats/morefloats.sty\\
%   morefloats.pdf & doc/latex/morefloats/morefloats.pdf\\
%   morefloats-example.tex & doc/latex/morefloats/morefloats-example.tex\\
%   morefloats-example.pdf & doc/latex/morefloats/morefloats-example.pdf\\
%   morefloats.dtx & source/latex/morefloats/morefloats.dtx\\
% \end{tabular}^^A
% }^^A
% \sbox0{\t}^^A
% \ifdim\wd0>\linewidth
%   \begingroup
%     \advance\linewidth by\leftmargin
%     \advance\linewidth by\rightmargin
%   \edef\x{\endgroup
%     \def\noexpand\lw{\the\linewidth}^^A
%   }\x
%   \def\lwbox{^^A
%     \leavevmode
%     \hbox to \linewidth{^^A
%       \kern-\leftmargin\relax
%       \hss
%       \usebox0
%       \hss
%       \kern-\rightmargin\relax
%     }^^A
%   }^^A
%   \ifdim\wd0>\lw
%     \sbox0{\small\t}^^A
%     \ifdim\wd0>\linewidth
%       \ifdim\wd0>\lw
%         \sbox0{\footnotesize\t}^^A
%         \ifdim\wd0>\linewidth
%           \ifdim\wd0>\lw
%             \sbox0{\scriptsize\t}^^A
%             \ifdim\wd0>\linewidth
%               \ifdim\wd0>\lw
%                 \sbox0{\tiny\t}^^A
%                 \ifdim\wd0>\linewidth
%                   \lwbox
%                 \else
%                   \usebox0
%                 \fi
%               \else
%                 \lwbox
%               \fi
%             \else
%               \usebox0
%             \fi
%           \else
%             \lwbox
%           \fi
%         \else
%           \usebox0
%         \fi
%       \else
%         \lwbox
%       \fi
%     \else
%       \usebox0
%     \fi
%   \else
%     \lwbox
%   \fi
% \else
%   \usebox0
% \fi
% \end{quote}
% If you have a \xfile{docstrip.cfg} that configures and enables \docstrip's
% TDS installing feature, then some files can already be in the right
% place, see the documentation of \docstrip{}.
%
% \subsection{Refresh file name databases}
%
% If your \TeX~distribution (\TeX{} Live, \mikTeX, \teTeX, \dots) relies on
% file name databases, you must refresh these. For example, \teTeX{} users run
% \verb|texhash| or \verb|mktexlsr|.
%
% \subsection{Some details for the interested}
%
% \paragraph{Unpacking with \LaTeX{}.}
% The \xfile{.dtx} chooses its action depending on the format:
% \begin{description}
% \item[\plainTeX:] Run \docstrip{} and extract the files.
% \item[\LaTeX:] Generate the documentation.
% \end{description}
% If you insist on using \LaTeX{} for \docstrip{} (really,
% \docstrip{} does not need \LaTeX ), then inform the autodetect routine
% about your intention:
% \begin{quote}
%   \verb|latex \let\install=y% \iffalse meta-comment
%
% File: morefloats.dtx
% Version: 2015/07/22 v1.0h
%
% Copyright (C) 2010 - 2015 by
%    H.-Martin M"unch <Martin dot Muench at Uni-Bonn dot de>
% Portions of code copyrighted by other people as marked.
%
% LaTeX 2015 provides the extrafloats command.
% Don Hosek, Quixote, 1990/07/27 (Thanks!)
% invented the main code for handling more floats
% before extrafloats was available.
% Maintenance has been taken over in September 2010
% by H.-Martin M\"{u}nch.
% David Carlisle pointed the maintainer to the new
% extrafloats command (Thanks!).
%
% This work may be distributed and/or modified under the
% conditions of the LaTeX Project Public License, either
% version 1.3c of this license or (at your option) any later
% version. This version of this license is in
%    http://www.latex-project.org/lppl/lppl-1-3c.txt
% and the latest version of this license is in
%    http://www.latex-project.org/lppl.txt
% and version 1.3c or later is part of all distributions of
% LaTeX version 2005/12/01 or later.
%
% This work has the LPPL maintenance status "maintained".
%
% The Current Maintainer of this work is H.-Martin Muench.
%
% This work consists of the main source file morefloats.dtx,
% the README, and the derived files
%    morefloats.sty, morefloats.pdf,
%    morefloats.ins, morefloats.drv,
%    morefloats-example.tex, morefloats-example.pdf.
%
% 'morefloats' is available on CTAN:
% https://www.ctan.org/pkg/morefloats
%
% Also a TDS.ZIP file is provided that contains all the files
% already sorted in a TDS tree:
% http://mirror.ctan.org/install/macros/latex/contrib/morefloats.tds.zip
%
%<*ignore>
\begingroup
  \catcode123=1 %
  \catcode125=2 %
  \def\x{LaTeX2e}%
\expandafter\endgroup
\ifcase 0\ifx\install y1\fi\expandafter
         \ifx\csname processbatchFile\endcsname\relax\else1\fi
         \ifx\fmtname\x\else 1\fi\relax
\else\csname fi\endcsname
%</ignore>
%<*install>
\input docstrip.tex
\Msg{*******************************************************************************}
\Msg{* Installation                                                                *}
\Msg{* Package: morefloats 2015/07/22 v1.0h Raise limit of unprocessed floats (HMM)*}
\Msg{*******************************************************************************}

\keepsilent
\askforoverwritefalse

\let\MetaPrefix\relax
\preamble

This is a generated file.

Project: morefloats
Version: 2015/07/22 v1.0h

Copyright (C) 2010 - 2015 by
    H.-Martin M"unch <Martin dot Muench at Uni-Bonn dot de>
Portions of code copyrighted by other people as marked.

The usual disclaimer applies:
If it doesn't work right that's your problem.
(Nevertheless, send an e-mail to the maintainer
 when you find an error in this package.)

This work may be distributed and/or modified under the
conditions of the LaTeX Project Public License, either
version 1.3c of this license or (at your option) any later
version. This version of this license is in
   http://www.latex-project.org/lppl/lppl-1-3c.txt
and the latest version of this license is in
   http://www.latex-project.org/lppl.txt
and version 1.3c or later is part of all distributions of
LaTeX version 2005/12/01 or later.

This work has the LPPL maintenance status "maintained".

The Current Maintainer of this work is H.-Martin Muench.

LaTeX 2015 provides the extrafloats command.
Don Hosek, Quixote, 1990/07/27 (Thanks!)
invented the main code for handling more floats
before extrafloats was available.
Maintenance has been taken over in September 2010
by H.-Martin Muench.
David Carlisle pointed the maintainer to the new
extrafloats command (Thanks!).

This work consists of the main source file morefloats.dtx,
the README, and the derived files
   morefloats.sty, morefloats.pdf,
   morefloats.ins, morefloats.drv,
   morefloats-example.tex, morefloats-example.pdf.

In memoriam
 Claudia Simone Barth + 1996/01/30
 Tommy Muench + 2014/01/02
 Hans-Klaus Muench + 2014/08/24

\endpreamble
\let\MetaPrefix\DoubleperCent

\generate{%
  \file{morefloats.ins}{\from{morefloats.dtx}{install}}%
  \file{morefloats.drv}{\from{morefloats.dtx}{driver}}%
  \usedir{tex/latex/morefloats}%
  \file{morefloats.sty}{\from{morefloats.dtx}{package}}%
  \usedir{doc/latex/morefloats}%
  \file{morefloats-example.tex}{\from{morefloats.dtx}{example}}%
}

\catcode32=13\relax% active space
\let =\space%
\Msg{************************************************************************}
\Msg{*}
\Msg{* To finish the installation you have to move the following}
\Msg{* file into a directory searched by TeX:}
\Msg{*}
\Msg{*  morefloats.sty}
\Msg{*}
\Msg{* To produce the documentation run the file `morefloats.drv'}
\Msg{* through (pdf)LaTeX, e.g.}
\Msg{*  pdflatex morefloats.drv}
\Msg{*  makeindex -s gind.ist morefloats.idx}
\Msg{*  pdflatex morefloats.drv}
\Msg{*  makeindex -s gind.ist morefloats.idx}
\Msg{*  pdflatex morefloats.drv}
\Msg{*}
\Msg{* At least three runs are necessary e.g. to get the}
\Msg{*  references right!}
\Msg{*}
\Msg{* Happy TeXing!}
\Msg{*}
\Msg{************************************************************************}

\endbatchfile
%</install>
%<*ignore>
\fi
%</ignore>
%
% \section{The documentation driver file}
%
% The next bit of code contains the documentation driver file for
% \TeX , i.\,e., the file that will produce the documentation you
% are currently reading. It will be extracted from this file by the
% \texttt{docstrip} programme. That is, run \LaTeX{} on \texttt{docstrip}
% and specify the \texttt{driver} option when \texttt{docstrip}
% asks for options.
%
%    \begin{macrocode}
%<*driver>
\NeedsTeXFormat{LaTeX2e}[2015/01/01]
\ProvidesFile{morefloats.drv}%
  [2015/07/22 v1.0h Raise limit of unprocessed floats (HMM)]
\documentclass{ltxdoc}[2015/03/26]%   v2.0w
\usepackage[T1]{fontenc}[2005/09/27]% v1.99g
\usepackage{pdflscape}[2008/08/11]%   v0.10
\usepackage{holtxdoc}[2012/03/21]%    v0.24
%% morefloats should work with earlier versions of LaTeX2e and
%% may work with earlier versions of the class and those packages,
%% but this was not tested.
%% Please consider updating your LaTeX, class, and packages
%% to the most recent version (if they are not already the most
%% recent version).
\hypersetup{%
 pdfsubject={LaTeX2e package for increasing the limit of unprocessed floats (HMM)},%
 pdfkeywords={LaTeX, morefloats, floats, H.-Martin Muench},%
 pdfencoding=auto,%
 pdflang={en},%
 breaklinks=true,%
 linktoc=all,%
 pdfstartview=FitH,%
 pdfpagelayout=OneColumn,%
 bookmarksnumbered=true,%
 bookmarksopen=true,%
 bookmarksopenlevel=2,%
 pdfmenubar=true,%
 pdftoolbar=true,%
 pdfwindowui=true,%
 pdfnewwindow=true%
}
\CodelineIndex
\hyphenation{docu-ment}
\gdef\unit#1{\mathord{\thinspace\mathrm{#1}}}%
\begin{document}
  \DocInput{morefloats.dtx}%
\end{document}
%</driver>
%    \end{macrocode}
%
% \fi
%
% \CheckSum{3565}
%
% \CharacterTable
%  {Upper-case    \A\B\C\D\E\F\G\H\I\J\K\L\M\N\O\P\Q\R\S\T\U\V\W\X\Y\Z
%   Lower-case    \a\b\c\d\e\f\g\h\i\j\k\l\m\n\o\p\q\r\s\t\u\v\w\x\y\z
%   Digits        \0\1\2\3\4\5\6\7\8\9
%   Exclamation   \!     Double quote  \"     Hash (number) \#
%   Dollar        \$     Percent       \%     Ampersand     \&
%   Acute accent  \'     Left paren    \(     Right paren   \)
%   Asterisk      \*     Plus          \+     Comma         \,
%   Minus         \-     Point         \.     Solidus       \/
%   Colon         \:     Semicolon     \;     Less than     \<
%   Equals        \=     Greater than  \>     Question mark \?
%   Commercial at \@     Left bracket  \[     Backslash     \\
%   Right bracket \]     Circumflex    \^     Underscore    \_
%   Grave accent  \`     Left brace    \{     Vertical bar  \|
%   Right brace   \}     Tilde         \~}
%
% \GetFileInfo{morefloats.drv}
%
% \begingroup
%   \def\x{\#,\$,\^,\_,\~,\ ,\&,\{,\},\%}%
%   \makeatletter
%   \@onelevel@sanitize\x
% \expandafter\endgroup
% \expandafter\DoNotIndex\expandafter{\x}
% \expandafter\DoNotIndex\expandafter{\string\ }
% \begingroup
%   \makeatletter
%     \lccode`9=32\relax
%     \lowercase{%^^A
%       \edef\x{\noexpand\DoNotIndex{\@backslashchar9}}%^^A
%     }%^^A
%   \expandafter\endgroup\x
% \DoNotIndex{\\,\,}
% \DoNotIndex{\def,\edef,\gdef, \xdef}
% \DoNotIndex{\ifnum, \ifx}
% \DoNotIndex{\begin, \end, \LaTeX, \LateXe}
% \DoNotIndex{\bigskip, \caption, \centering, \hline, \MessageBreak}
% \DoNotIndex{\documentclass, \markboth, \mathrm, \mathord}
% \DoNotIndex{\NeedsTeXFormat, \usepackage, \ProvidesPackage, \RequirePackage}
% \DoNotIndex{\newline, \newpage, \pagebreak}
% \DoNotIndex{\section, \subsection, \space, \thinspace}
% \DoNotIndex{\textsf, \texttt}
% \DoNotIndex{\the, \@tempcnta,\@tempcntb}
% \DoNotIndex{\@elt,\@freelist, \newinsert}
% \DoNotIndex{\bx@A,  \bx@B,  \bx@C,  \bx@D,  \bx@E,  \bx@F,  \bx@G,  \bx@H,  \bx@I,  \bx@J,  \bx@K,  \bx@L,  \bx@M,  \bx@N,  \bx@O,  \bx@P,  \bx@Q,  \bx@R,  \bx@S,  \bx@T,  \bx@U,  \bx@V,  \bx@W,  \bx@X,  \bx@Y,  \bx@Z}
% \DoNotIndex{\bx@AA, \bx@AB, \bx@AC, \bx@AD, \bx@AE, \bx@AF, \bx@AG, \bx@AH, \bx@AI, \bx@AJ, \bx@AK, \bx@AL, \bx@AM, \bx@AN, \bx@AO, \bx@AP, \bx@AQ, \bx@AR, \bx@AS, \bx@AT, \bx@AU, \bx@AV, \bx@AW, \bx@AX, \bx@AY, \bx@AZ}
% \DoNotIndex{\bx@BA, \bx@BB, \bx@BC, \bx@BD, \bx@BE, \bx@BF, \bx@BG, \bx@BH, \bx@BI, \bx@BJ, \bx@BK, \bx@BL, \bx@BM, \bx@BN, \bx@BO, \bx@BP, \bx@BQ, \bx@BR, \bx@BS, \bx@BT, \bx@BU, \bx@BV, \bx@BW, \bx@BX, \bx@BY, \bx@BZ}
% \DoNotIndex{\bx@CA, \bx@CB, \bx@CC, \bx@CD, \bx@CE, \bx@CF, \bx@CG, \bx@CH, \bx@CI, \bx@CJ, \bx@CK, \bx@CL, \bx@CM, \bx@CN, \bx@CO, \bx@CP, \bx@CQ, \bx@CR, \bx@CS, \bx@CT, \bx@CU, \bx@CV, \bx@CW, \bx@CX, \bx@CY, \bx@CZ}
% \DoNotIndex{\bx@DA, \bx@DB, \bx@DC, \bx@DD, \bx@DE, \bx@DF, \bx@DG, \bx@DH, \bx@DI, \bx@DJ, \bx@DK, \bx@DL, \bx@DM, \bx@DN, \bx@DO, \bx@DP, \bx@DQ, \bx@DR, \bx@DS, \bx@DT, \bx@DU, \bx@DV, \bx@DW, \bx@DX, \bx@DY, \bx@DZ}
% \DoNotIndex{\bx@EA, \bx@EB, \bx@EC, \bx@ED, \bx@EE, \bx@EF, \bx@EG, \bx@EH, \bx@EI, \bx@EJ, \bx@EK, \bx@EL, \bx@EM, \bx@EN, \bx@EO, \bx@EP, \bx@EQ, \bx@ER, \bx@ES, \bx@ET, \bx@EU, \bx@EV, \bx@EW, \bx@EX, \bx@EY, \bx@EZ}
% \DoNotIndex{\bx@FA, \bx@FB, \bx@FC, \bx@FD, \bx@FE, \bx@FF, \bx@FG, \bx@FH, \bx@FI, \bx@FJ, \bx@FK, \bx@FL, \bx@FM, \bx@FN, \bx@FO, \bx@FP, \bx@FQ, \bx@FR, \bx@FS, \bx@FT, \bx@FU, \bx@FV, \bx@FW, \bx@FX, \bx@FY, \bx@FZ}
% \DoNotIndex{\bx@GA, \bx@GB, \bx@GC, \bx@GD, \bx@GE, \bx@GF, \bx@GG, \bx@GH, \bx@GI, \bx@GJ, \bx@GK, \bx@GL, \bx@GM, \bx@GN, \bx@GO, \bx@GP, \bx@GQ, \bx@GR, \bx@GS, \bx@GT, \bx@GU, \bx@GV, \bx@GW, \bx@GX, \bx@GY, \bx@GZ}
% \DoNotIndex{\bx@HA, \bx@HB, \bx@HC, \bx@HD, \bx@HE, \bx@HF, \bx@HG, \bx@HH, \bx@HI, \bx@HJ, \bx@HK, \bx@HL, \bx@HM, \bx@HN, \bx@HO, \bx@HP, \bx@HQ, \bx@HR, \bx@HS, \bx@HT, \bx@HU, \bx@HV, \bx@HW, \bx@HX, \bx@HY, \bx@HZ}
% \DoNotIndex{\bx@IA, \bx@IB, \bx@IC, \bx@ID, \bx@IE, \bx@IF, \bx@IG, \bx@IH, \bx@II, \bx@IJ, \bx@IK, \bx@IL, \bx@IM, \bx@IN, \bx@IO, \bx@IP, \bx@IQ, \bx@IR, \bx@IS, \bx@IT, \bx@IU, \bx@IV, \bx@IW, \bx@IX, \bx@IY, \bx@IZ}
% \DoNotIndex{\bx@JA, \bx@JB, \bx@JC, \bx@JD, \bx@JE, \bx@JF, \bx@JG, \bx@JH, \bx@JI, \bx@JJ, \bx@JK, \bx@JL, \bx@JM, \bx@JN, \bx@JO, \bx@JP, \bx@JQ, \bx@JR, \bx@JS, \bx@JT, \bx@JU, \bx@JV, \bx@JW, \bx@JX, \bx@JY, \bx@JZ}
% \DoNotIndex{\morefloats@mx}
%
% \title{The \xpackage{morefloats} package}
% \date{2015/07/22 v1.0h}
% \author{H.-Martin M\"{u}nch (current maintainer;\\
%  invented by Don Hosek, Quixote)\\
%  \xemail{Martin.Muench at Uni-Bonn.de}}
%
% \maketitle
%
% \begin{abstract}
% The default limit of unprocessed floats, $18$,
% can be increased with this \xpackage{morefloats} package.
% Otherwise, |\clear(double)page|, |h(!)|, |H|~from the \xpackage{float} package,
% or |\FloatBarrier| from the \xpackage{picins} package might help.
% \end{abstract}
%
% \bigskip
%
% \noindent Note: \LaTeX{} 2015 provides the |\extrafloats| command.
% \textsc{Don Hosek}, Quixote, 1990/07/27 (Thanks!)
% invented the main code for handling more floats
% before |\extrafloats| was available.
% \textsc{David Carlisle} pointed the maintainer to the new
% |\extrafloats| (Thanks!).
% The current maintainer is \textsc{H.-Martin M\"{u}nch}.\\
%
% \bigskip
%
% \noindent Disclaimer for web links: The author is not responsible for any contents
% referred to in this work unless he has full knowledge of illegal contents.
% If any damage occurs by the use of information presented there, only the
% author of the respective pages might be liable, not the one who has referred
% to these pages.
%
% \bigskip
%
% \noindent {\color{green} Save per page about $200\unit{ml}$ water,
% $2\unit{g}$ CO$_{2}$ and $2\unit{g}$ wood:\\
% Therefore please print only if this is really necessary.}
%
% \newpage
%
% \tableofcontents
%
% \newpage
%
% \section{Introduction\label{sec:Introduction}}
%
% The default limit of unprocessed floats, $18$,
% can be increased with this \xpackage{morefloats} package.\\
% \textquotedblleft{}Of course one immediately begins to wonder:
% \guillemotright{}Why eighteen?!\guillemotleft{} And it turns out that $18$
% one{-}line tables with $10$~point Computer Modern using \xclass{article.cls}
% produces almost exactly one page worth of material.\textquotedblright{}\\
% (user \url{https://tex.stackexchange.com/users/1495/kahen} as comment to\\
% \url{https://tex.stackexchange.com/a/35596/6865} on 2011/11/21)\\
% As alternatives (see also section \ref{sec:alternatives} below)
% |\clear(double)page|, |h(!)|, |H|~from the
% \href{https://www.ctan.org/pkg/float}{\xpackage{float}} package,
% or |\FloatBarrier| from the %
% \href{https://www.ctan.org/pkg/picins}{\xpackage{picins}} package might help.
% If the floats cannot be placed anywhere at all, extending the number of floats
% will just delay the arrival of the corresponding error.
%
% \section{Usage}
%
% \subsection{General usage:}
% Load the package placing
% \begin{quote}
%   |\usepackage[<|\textit{options}|>]{morefloats}|
% \end{quote}
% \noindent in the preamble of your \LaTeXe{} source file (the earlier the better).\\
% \noindent The \xpackage{morefloats} package takes two options: |maxfloats| and
% |morefloats|, where |morefloats| gives the number of additional floats and
% |maxfloats| gives the maximum number of floats. |maxfloats=25| therefore means,
% that there are $18$ (default) floats and $7$ additional floats.
% |morefloats=7| therefore has the same meaning. It is only necessary to give
% one of these two options. At the time being, it is not possible to reduce
% the number of floats (for example to save boxes). If you have code
% accomplishing that, please send it to the package maintainer, thanks.\\
% Version 1.0b used a fixed value of |maxfloats=36|. Therefore for backward
% compatibility this value is taken as the default one.\\
% Example:
% \begin{quote}
%   |\usepackage[maxfloats=25]{morefloats}|
% \end{quote}
% or
% \begin{quote}
%   |\usepackage[morefloats=7]{morefloats}|
% \end{quote}
% or
% \begin{quote}
%   |\usepackage[maxfloats=25,morefloats=7]{morefloats}|
% \end{quote}
%
% \subsection{Situation for \LaTeX{} before 2015:}
% |Float| uses |insert|, and each |insert| uses a group of |count|, |dimen|,
% |skip|, and |box| each. When there are not enough available, no |\newinsert|
% can be created. The
% \href{https://www.ctan.org/pkg/etex-pkg}{\xpackage{etex}} package
% provides access at an extended range of those registers,
% but does not use those for |\newinsert|. Therefore the inserts must be
% reserved first, which forces the use of the extended register range
% for other new |count|, |dimen|, |skip|, and |box|:
% To have more floats available, use |\usepackage{etex}\reserveinserts{...}|
% right after |\documentclass[...]{...}|, where the argument of |\reserveinserts|
% should be at least the maximum number of floats. Add another $10$
% if the \href{https://www.ctan.org/pkg/bigfoot}{\xpackage{bigfoot}} or the
% \href{https://www.ctan.org/pkg/manyfoot}{\xpackage{manyfoot}} package
% is used, but |\reserveinserts| can be about $234$ at most for older
% \LaTeX{} formats.
%
% \subsection{Situation for \LaTeX{} since 2015:}
% Now |\reserveinserts| can be about $2\,147\,483\,647$,
% but |\insert255{}| even then produces an error.
% The \LaTeX{} 2015 \textquotedblleft release provides a new command in the format
% |\extrafloats|\textquotedblright ; \textquotedblleft as it doesn't use
% |\newinsert| (and as the 2015 format uses extended registers by default)
% you can allocate a lot more floats\textquotedblright{} %
% (both \textsc{David Carlisle}, 29. June 2015), \hbox{e.\,g. |\extrafloats{1234}|.}
%
% \section{Alternatives (kind of)\label{sec:alternatives}}
%
% The very old \xpackage{morefloats} with a fixed number of |maxfloats=36| {}%
% \hbox{(i.\,e. $18$ |morefloats|)} has been archived at
% \href{http://mirror.ctan.org/obsolete/macros/latex/contrib/misc/morefloats.sty}{%
%  http://mirror.ctan.org/obsolete/macros/latex/contrib/}\newline%
% \href{http://mirror.ctan.org/obsolete/macros/latex/contrib/misc/morefloats.sty}{%
%  misc/morefloats.sty}.
%
% \bigskip
%
% If you really want to increase the number of (possible) floats,
% this is the right package. On the other hand, if you ran into trouble of
% \texttt{Too many unprocessed floats}, but would also accept less floats,
% there are some other possibilities:
% \begin{description}
%   \item[-] The command |\clearpage| forces \LaTeX{} to output any floating objects
%     that occurred before this command (and go to the next page).
%     |\cleardoublepage| does the same but ensures that the next page with
%     output is one with odd page number.
%   \item[-] Using different float specifiers: |t|~top, |b|~bottom, |p|~page
%     of floats.
%   \item[-] Suggesting \LaTeX{} to put the object where it was placed:
%     |h| (= here) float specifier.
%   \item[-] Telling \LaTeX{} to please put the object where it was placed:
%     |h!| (= here!) float specifier.
%   \item[-] Forcing \LaTeX{} to put the object where it was placed and shut up:
%     The \xpackage{float} package provides the \textquotedblleft style
%     option here, giving floating environments a [H] option which means
%     `PUT IT HERE' (as opposed to the standard [h] option which means
%     `You may put it here if you like')\textquotedblright{} (\xpackage{float}
%     package documentation v1.3d as of 2001/11/08).
%     Changing e.\,g. |\begin{figure}[tbp]...| to |\begin{figure}[H]...|
%     forces the figure to be placed HERE instead of floating away.\\
%     The \xpackage{float} package is available at \url{https://www.ctan.org/pkg/float}.
%   \item[-] The \xpackage{placeins} package provides the command |\FloatBarrier|.
%     Floats occurring before the |\FloatBarrier| are not allowed to float
%     to a later place, and floats occurring after the |\FloatBarrier| are not
%     allowed to float to an earlier place than the |\FloatBarrier|. (There
%     can be more than one |\FloatBarrier| in a document.) -- %
%     The same package also provides an option to automatically add |\FloatBarrier|s to
%     section headings. It is further possible to make
%     |\FloatBarrier|s less strict (see that package's documentation).\\
%     The \xpackage{placeins} package is available at \url{https://www.ctan.org/pkg/placeins}.
%   \item[-] Sometimes also increasing the maximum number (|\maxdeadcycles|)
%     of calls of |\output| without a |\shipout| can help,
%     for example |\maxdeadcycles=123\relax|.
% \end{description}
%
% \newpage
%
% \noindent See also the following entries in the
% \texttt{UK~List of TeX Frequently Asked Questions on the Web}:
% \begin{description}
%   \item[-] \url{http://www.tex.ac.uk/cgi-bin/texfaq2html?label=floats}
%   \item[-] \url{http://www.tex.ac.uk/cgi-bin/texfaq2html?label=tmupfl}
%   \item[-] \url{http://www.tex.ac.uk/cgi-bin/texfaq2html?label=figurehere}
% \end{description}
% and the \textbf{excellent article on \textquotedblleft How to influence the position
% of float environments like figure and table in \hbox{\LaTeX ?\textquotedblright } by
% \textsc{Frank Mittelbach}} at \url{https://tex.stackexchange.com/a/39020/6865}{}!\\
%
% \bigskip
%
% \noindent (You programmed or found another alternative,
%  which is available at CTAN?\\
%  OK, send an e-mail to me with the name, location at CTAN,
%  and a short notice, and I will probably include it in
%  the list above.)
%
% \bigskip
%
% \section{Example}
%
%    \begin{macrocode}
%<*example>
\documentclass[british]{article}[2014/09/29]%      v1.4h
%%%%%%%%%%%%%%%%%%%%%%%%%%%%%%%%%%%%%%%%%%%%%%%%%%%%%%%%%%%%%%%%%%%%%
\usepackage[maxfloats=25]{morefloats}[2015/07/22]% v1.0h
%\maxdeadcycles=200\relax%
%% \maxdeadcycles is the maximum number of calls of \output
%% without a \shipout.
\gdef\unit#1{\mathord{\thinspace\mathrm{#1}}}%
\listfiles
\begin{document}

\makeatletter

\section*{Example for morefloats}
\markboth{Example for morefloats}{Example for morefloats}

This example demonstrates the use of package\newline
\textsf{morefloats}, v1.0h as of 2015/07/22 (HMM).\newline
The package takes options (here:
\verb|maxfloats=|\texttt{\morefloats@maxfloats} is used).\newline
For more details please see the documentation!\newline

To reproduce the\newline
\LaTeX{} \texttt{ Error: Too many unprocessed floats},\newline
comment out the \verb|\usepackage...| in the preamble
(line~3)\newline
(by placing a \% before it).\newline

\bigskip

Save per page about $200\unit{ml}$~water, $2\unit{g}$~CO$_{2}$
and $2\unit{g}$~wood:\newline
Therefore please print only if this is really necessary.\newline
I do NOT think, that it is necessary to print THIS file, really!

\bigskip

There follow \morefloats@maxfloats{} floating tables.

\pagebreak

\@tempcnta=18\relax% default floats
\advance\@tempcnta by \morefloats@morefloats%
% \morefloats@morefloats is the number of additional
% floating tables to create.
\loop
  \ifnum\@tempcnta>0\relax%
  \begin{table}[t]\centering%
    \begin{tabular}{|l|}%
      \hline%
      A table, which will keep floating.\\%
      \hline
    \end{tabular}%
    \caption{A floating Table.}%
  \end{table}%
  \advance\@tempcnta by -1\relax%
\repeat

\makeatother

\end{document}
%</example>
%    \end{macrocode}
%
% \newpage
%
% \StopEventually{}
%
% \section{The implementation}
%
% We start off by checking that we are loading into \LaTeXe{} and
% announcing the name and version of this package.
%
%    \begin{macrocode}
%<*package>
%    \end{macrocode}
%
%    \begin{macrocode}
\NeedsTeXFormat{LaTeX2e}[2011/06/27]
%% The current format at the time of the release of this version of the
%% morefloats package was 2015/01/01, patch level 2.
\ProvidesPackage{morefloats}[2015/07/22 v1.0h
            Raise limit of unprocessed floats (HMM)]

%    \end{macrocode}
%
% \DescribeMacro{Options}
%    \begin{macrocode}
\RequirePackage{kvoptions}[2011/06/30]% v3.11
%% morefloats may work with earlier versions of LaTeX2e and that
%% package, but this was not tested.
%% Please consider updating your LaTeX and package
%% to the most recent version (if they are not already the most
%% recent version).

\SetupKeyvalOptions{family=morefloats,prefix=morefloats@}
\DeclareStringOption{maxfloats}%  \morefloats@maxfloats
\DeclareStringOption{morefloats}% \morefloats@morefloats

\ProcessKeyvalOptions*

%    \end{macrocode}
%
% The \xpackage{morefloats} package takes two options: |maxfloats| and |morefloats|,
% where |morefloats| gives the number of additional floats and |maxfloats| gives
% the maximum number of floats. |maxfloats=37| therefore means, that there are
% $18$ (default) floats and another $19$ additional floats. |morefloats=19| therefore
% has the same meaning. Version~1.0b used a fixed value of |maxfloats=36|.
% Therefore for backward compatibility this value will be taken as the default one.\\
% Now we check whether |maxfloats=...| or |morefloats=...| or both were used,
% and if one option was not used, we supply the according value.
% If no option was used at all, we use the default values.
% Too many requested floats produce error massages by \LaTeX ,
% which might not be easily traced back to this,
% therefore we issue a warning. If option |maxfloats| or |morefloats| is no number,
% the user will received the according error message by \LaTeX{} automatically.
%
%    \begin{macrocode}
\ifx\morefloats@maxfloats\@empty%
  \ifx\morefloats@morefloats\@empty% apply defaults:
    \gdef\morefloats@maxfloats{36}%
    \gdef\morefloats@morefloats{18}%
  \else%
    \ifnum\morefloats@morefloats>1569\relax%
      \PackageWarning{morefloats}{%
        \morefloats@morefloats\space more floats requested.\MessageBreak%
        LaTeX might run out of memory before this\MessageBreak%
        (in which case it will notify you)\MessageBreak%
       }%
    \else%
      \PackageInfo{morefloats}{%
        \morefloats@morefloats\space more floats requested.\MessageBreak%
        LaTeX might run out of memory before this\MessageBreak%
        (in which case it will notify you)\MessageBreak%
       }%
    \fi%
    \@tempcnta=\morefloats@morefloats\relax%
    \advance\@tempcnta by +18%
    \xdef\morefloats@maxfloats{\the\@tempcnta}%
  \fi%
\else%
  \ifx\morefloats@morefloats\@empty%
    \@tempcnta=\morefloats@maxfloats\relax%
    \advance\@tempcnta by -18%
    \xdef\morefloats@morefloats{\the\@tempcnta}%
    \ifnum\morefloats@morefloats<\z@\relax% i.e. \morefloats@maxfloats < 18
      \gdef\morefloats@morefloats{0}%
    \fi%
    \ifnum\morefloats@maxfloats>1587\relax%
      \PackageWarning{morefloats}{%
        \morefloats@maxfloats\space floats requested.\MessageBreak%
        LaTeX might run out of memory before this\MessageBreak%
        (in which case it will notify you)\MessageBreak%
       }%
    \fi%
  \fi%
\fi%

\@tempcnta=\morefloats@maxfloats\relax%
\xdef\morefloats@max{\the\@tempcnta}%

\ifnum\@tempcnta<18\relax%
  \PackageError{morefloats}{Option maxfloats is \the\@tempcnta<18}{%
    maxfloats must be a number equal to or larger than 18\MessageBreak%
    (or not used at all).\MessageBreak%
    Now setting maxfloats=18.\MessageBreak%
   }%
  \gdef\morefloats@max{18}%
\fi%

\@tempcnta=\morefloats@morefloats\relax%
\xdef\morefloats@more{\the\@tempcnta}%

\ifnum\@tempcnta<\z@\relax%
  \PackageError{morefloats}{Option morefloats is \the\@tempcnta<0}{%
    morefloats must be a number equal to or larger than 0\MessageBreak%
    (or not used at all).\MessageBreak%
    Now setting morefloats=0.\MessageBreak%
   }%
  \gdef\morefloats@more{0}%
\fi%

\@tempcnta=18\relax%
\advance\@tempcnta by \morefloats@more%
%    \end{macrocode}
%
% The value of |morefloats| should now be equal to the value of |morefloats@max|.
%
%    \begin{macrocode}
\advance\@tempcnta by -\morefloats@max%
%    \end{macrocode}
%
% Therefore |\@tempcnta| should now be equal to zero.
%
%    \begin{macrocode}
\xdef\morefloats@mx{\the\@tempcnta}%
\ifnum\morefloats@mx=\z@\relax%
  \@tempcnta=\morefloats@maxfloats\relax%
\else%
  \PackageError{morefloats}{%
    Clash between options maxfloats and morefloats}{%
    Option maxfloats must be empty\MessageBreak%
    or the sum of 18 and option value morefloats,\MessageBreak%
    but it is maxfloats=\morefloats@maxfloats\space and %
    morefloats=\morefloats@morefloats .\MessageBreak%
    }%
%    \end{macrocode}
%
% We choose the larger value to be used.
%
%    \begin{macrocode}
  \ifnum\@tempcnta<\z@% \morefloats@max > \morefloats@more
    \@tempcnta=\morefloats@maxfloats\relax%
  \else% \@tempcnta>0, \morefloats@max < \morefloats@more
    \@tempcnta=18\relax%
    \advance\@tempcnta by \morefloats@morefloats%
  \fi%
\fi%
\edef\morefloats@mx{\the\@tempcnta}%
%    \end{macrocode}
%
% Maybe we had to change |\morefloats@maxfloats| or |\morefloats@maxfloats|:
%
%    \begin{macrocode}
\xdef\morefloats@maxfloats{\the\@tempcnta}%
\advance\@tempcnta by -18\relax%
\xdef\morefloats@morefloats{\the\@tempcnta}%
\gdef\morefloats@test{1}%
\ifx\morefloats@morefloats\morefloats@test\relax%
  \PackageInfo{morefloats}{%
    Maximum number of possible floats asked for: \morefloats@maxfloats%
    \MessageBreak%
    (i.e. one more float)\@gobble%
   }%
\else%
  \PackageInfo{morefloats}{%
    Maximum number of possible floats asked for: \morefloats@maxfloats%
    \MessageBreak%
    (i.e. \morefloats@morefloats\space more floats).\MessageBreak%
    LaTeX might run out of memory before this\MessageBreak%
    (in which case it will notify you)%
    \@gobble%
   }%
\fi%


%    \end{macrocode}
%
% The \LaTeX{} 2015 \textquotedblleft release provides a new command in the format
% |\extrafloats| which does a similar job [as earlier versions of this package did],
% although as it doesn't use |\newinsert| (and as the 2015 format uses extended
% registers by default) you can allocate a lot more floats,\textquotedblright{} %
% \hbox{e.\,g. |\extrafloats{1234}|.} Loading the \xpackage{etex} package and
% \xpackage{morefloats} with the new format would
% \textquotedblleft over{-}write the new allocation mechanism and end up with
% fewer floats available.\textquotedblright{} Therefore here it is tested
% \textquotedblleft for the new format and switch[ed] to the new mechanism
% in that case, so that existing documents work as before but using the new allocation
% scheme underneath.\textquotedblright{} (all \textsc{David Carlisle}, 29. June 2015,
% who provided also main parts of the following code)
%
%    \begin{macrocode}
%% Test for new mechanism in LaTeX 2015:
\ifx\e@alloc\@undefined\relax%
  %% This is an old LaTeX format, \extrafloats is not available.
  \PackageWarning{morefloats}{%
    \fmtname\space <\fmtversion> %
    \ifx\patch@level\@undefined\relax%
    \else patch level \patch@level%
    \fi%
    \MessageBreak%
    found. At least\MessageBreak%
    LaTeX2e <2015/01/01> patch level 2\MessageBreak%
    is now available\MessageBreak%
    and can handle even more floats%
    \@gobble%
   }%
\else%
  %% This is new in LaTeX 2015, \extrafloats is available.
  \@ifpackageloaded{etex}%
  {%% etex package loaded:
   %% "it overwrites all the new allocation system
   %% so really \extrafloats shouldn't be expected to work"
   %% (D. Carlisle, 2015/07/16, who also provided the following
   %% \extrafloats redefinition).
   \gdef\extrafloats#1{%
     \ifnum#1>\z@\relax%
       \count@\numexpr\float@count-1\relax%
       \ch@ck0\count@\count\relax%
       \ch@ck1\count@\dimen\relax%
       \ch@ck2\count@\skip\relax%
       \ch@ck4\count@\box\relax%
       \e@alloc@chardef\float@count\count@%
       \expandafter\e@alloc@chardef\csname bx@\the\float@count\endcsname\float@count%
       \@cons\@freelist{\csname bx@\the\float@count\endcsname}%
       \expandafter%
       \extrafloats\expandafter{\numexpr#1-1\relax}%
     \fi%
   }%
  }{% etex package not loaded
   }%
  \extrafloats{\morefloats@morefloats}%
  % The part after the test is no longer needed and therefore not loaded:
  \expandafter\endinput%
\fi%
%% End of the test for LaTeX 2015 (or newer).
%% Not new format, otherwise the last \endinput would have been applied.

%% Test for e-TeX:
\RequirePackage{ifetex}[2011/12/15]% v1.2
\ifetex%
  %% then we can use code similar to the one from David Carlisle,
  %% https://tex.stackexchange.com/a/212483/6865
  \mathchardef\float@count=32767\relax%
  \gdef\extrafloats#1{%
    \ifnum#1>\z@\relax%
      \count@\numexpr\float@count-1\relax%
      \ch@ck0\count@\count\relax%
      \ch@ck1\count@\dimen\relax%
      \ch@ck2\count@\skip\relax%
      \ch@ck4\count@\box\relax%
      \mathchardef\float@count\count@\relax%
      \expandafter\mathchardef\csname bx@\the\float@count\endcsname\float@count%
      \@cons\@freelist{\csname bx@\the\float@count\endcsname}%
      \expandafter%
      \extrafloats\expandafter{\numexpr#1-1\relax}%
    \fi}%
  \extrafloats{\morefloats@morefloats}%
  \expandafter\endinput%
\fi%
%% End of the test for e-TeX.
%% Old format and not e-TeX,
%% otherwise the last \endinput would have been applied.


%    \end{macrocode}
%
% If we ever come to this place, \textquotedblleft everything\textquotedblright{} %
% failed and we need to do things the old fashioned way,
% which severely limits the maximum number of additionally available floats.
%
%    \begin{macrocode}
\PackageWarning{morefloats}{%
  e-TeX is not available here\MessageBreak%
  but it is available in almost all\MessageBreak%
  recent TeX distributions.\MessageBreak%
  Maybe consider updating to one of those%
  \@gobble%
 }%

%    \end{macrocode}
%
% \newpage
%
% \begin{landscape}
%
% |Float| uses |insert|, and each |insert| use a group of |count|, |dimen|, |skip|,
% and |box| each. When there are not enough available, no |\newinsert| can be created.
%
%    \begin{macrocode}
%% Code similar to the one from Heiko Oberdiek,
%% http://permalink.gmane.org/gmane.comp.tex.latex.latex3/2159
                           \@tempcnta=\the\count10 \relax \def\maxfloats@vln{count}    %
\ifnum \count11>\@tempcnta \@tempcnta=\the\count11 \relax \def\maxfloats@vln{dimen} \fi%
\ifnum \count12>\@tempcnta \@tempcnta=\the\count12 \relax \def\maxfloats@vln{skip}  \fi%
\ifnum \count14>\@tempcnta \@tempcnta=\the\count14 \relax \def\maxfloats@vln{box}   \fi%
%% end similar
\@tempcntb=234\relax%
\advance\@tempcntb by -\@tempcnta\relax%
\@tempcnta=\@tempcntb\relax%
\ifnum\morefloats@mx>\@tempcntb\relax%
  \PackageError{morefloats}{Too many floats requested}{%
    Maximum number of possible floats asked for: \morefloats@mx .\MessageBreak%
    There are only \the\@tempcnta\space \maxfloats@vln\space left,\MessageBreak%
    therefore only \the\@tempcntb\space floats will be possible.\MessageBreak%
    Load the morefloats package earlier and/or\MessageBreak%
    reduce the number of used \maxfloats@vln\space registers\MessageBreak%
    to have more floats available!\MessageBreak%
   }%
  \xdef\morefloats@mx{\the\@tempcntb}%
\fi%

%    \end{macrocode}
%
% The task at hand is to increase \LaTeX{}'s default limit of $18$~unprocessed
% floats in memory at once to |maxfloats|.
% An examination of \texttt{latex.tex} reveals that this is accomplished
% by allocating~(!) an insert register for each unprocessed float. A~quick
% check of (the obsolete, now \texttt{ltplain}, update to \LaTeX2e{}!)
% \texttt{lplain.lis} reveals that there is room, in fact, for up to
% $256$ unprocessed floats, but \TeX{}'s main memory could be exhausted
% well before that happened.\\
%
% \LaTeX2e{} uses a |\dimen| for each |\newinsert|, and the number of |\dimen|s
% is also restricted. Therefore only use the number of floats you need!
% To check the number of used registers, you could use the \xpackage{regstats}
% and/or \xpackage{regcount} packages (see subsection~\ref{ss:Downloads}).
%
% \bigskip
%
% \DescribeMacro{Allocating insert registers}
% \DescribeMacro{@freelist}
% \DescribeMacro{@elt}
% \DescribeMacro{newinsert}
% First we allocate the additional insert registers needed.\\
%
% That accomplished, the next step is to define the macro |\@freelist|,
% which is merely a~list of the box registers each preceded by |\@elt|.
% This approach allows processing of the list to be done far more efficiently.
% A similar approach is used by \textsc{Mittelbach \& Sch\"{o}pf}'s \texttt{doc.sty} to
% keep track of control sequences, which should not be indexed.\\
% First for the 18 default \LaTeX{} boxes.\\
% \noindent |\ifnum maxfloats <= 18|, \LaTeX{} already allocated the insert registers. |\fi|\\
%
%    \begin{macrocode}
\global\long\def\@freelist{\@elt\bx@A\@elt\bx@B\@elt\bx@C\@elt\bx@D\@elt\bx@E\@elt\bx@F\@elt\bx@G\@elt\bx@H\@elt%
\bx@I\@elt\bx@J\@elt\bx@K\@elt\bx@L\@elt\bx@M\@elt\bx@N\@elt\bx@O\@elt\bx@P\@elt\bx@Q\@elt\bx@R}

%    \end{macrocode}
%
% Now we need to add |\@elt\bx@...| depending on the number of |morefloats| wanted:\\
% (\textsc{Karl Berry} helped with two out of three |\expandafter|s, thanks!)
%
% \medskip
%
%    \begin{macrocode}
\ifnum \morefloats@mx> 18 \newinsert\bx@S  \expandafter\gdef\expandafter\@freelist\expandafter{\@freelist \@elt\bx@S}
\ifnum \morefloats@mx> 19 \newinsert\bx@T  \expandafter\gdef\expandafter\@freelist\expandafter{\@freelist \@elt\bx@T}
\ifnum \morefloats@mx> 20 \newinsert\bx@U  \expandafter\gdef\expandafter\@freelist\expandafter{\@freelist \@elt\bx@U}
\ifnum \morefloats@mx> 21 \newinsert\bx@V  \expandafter\gdef\expandafter\@freelist\expandafter{\@freelist \@elt\bx@V}
\ifnum \morefloats@mx> 22 \newinsert\bx@W  \expandafter\gdef\expandafter\@freelist\expandafter{\@freelist \@elt\bx@W}
\ifnum \morefloats@mx> 23 \newinsert\bx@X  \expandafter\gdef\expandafter\@freelist\expandafter{\@freelist \@elt\bx@X}
\ifnum \morefloats@mx> 24 \newinsert\bx@Y  \expandafter\gdef\expandafter\@freelist\expandafter{\@freelist \@elt\bx@Y}
\ifnum \morefloats@mx> 25 \newinsert\bx@Z  \expandafter\gdef\expandafter\@freelist\expandafter{\@freelist \@elt\bx@Z}
\ifnum \morefloats@mx> 26 \newinsert\bx@AA \expandafter\gdef\expandafter\@freelist\expandafter{\@freelist \@elt\bx@AA}
\ifnum \morefloats@mx> 27 \newinsert\bx@AB \expandafter\gdef\expandafter\@freelist\expandafter{\@freelist \@elt\bx@AB}
\ifnum \morefloats@mx> 28 \newinsert\bx@AC \expandafter\gdef\expandafter\@freelist\expandafter{\@freelist \@elt\bx@AC}
\ifnum \morefloats@mx> 29 \newinsert\bx@AD \expandafter\gdef\expandafter\@freelist\expandafter{\@freelist \@elt\bx@AD}
\ifnum \morefloats@mx> 30 \newinsert\bx@AE \expandafter\gdef\expandafter\@freelist\expandafter{\@freelist \@elt\bx@AE}
\ifnum \morefloats@mx> 31 \newinsert\bx@AF \expandafter\gdef\expandafter\@freelist\expandafter{\@freelist \@elt\bx@AF}
\ifnum \morefloats@mx> 32 \newinsert\bx@AG \expandafter\gdef\expandafter\@freelist\expandafter{\@freelist \@elt\bx@AG}
\ifnum \morefloats@mx> 33 \newinsert\bx@AH \expandafter\gdef\expandafter\@freelist\expandafter{\@freelist \@elt\bx@AH}
\ifnum \morefloats@mx> 34 \newinsert\bx@AI \expandafter\gdef\expandafter\@freelist\expandafter{\@freelist \@elt\bx@AI}
\ifnum \morefloats@mx> 35 \newinsert\bx@AJ \expandafter\gdef\expandafter\@freelist\expandafter{\@freelist \@elt\bx@AJ}
\ifnum \morefloats@mx> 36 \newinsert\bx@AK \expandafter\gdef\expandafter\@freelist\expandafter{\@freelist \@elt\bx@AK}
\ifnum \morefloats@mx> 37 \newinsert\bx@AL \expandafter\gdef\expandafter\@freelist\expandafter{\@freelist \@elt\bx@AL}
\ifnum \morefloats@mx> 38 \newinsert\bx@AM \expandafter\gdef\expandafter\@freelist\expandafter{\@freelist \@elt\bx@AM}
\ifnum \morefloats@mx> 39 \newinsert\bx@AN \expandafter\gdef\expandafter\@freelist\expandafter{\@freelist \@elt\bx@AN}
\ifnum \morefloats@mx> 40 \newinsert\bx@AO \expandafter\gdef\expandafter\@freelist\expandafter{\@freelist \@elt\bx@AO}
\ifnum \morefloats@mx> 41 \newinsert\bx@AP \expandafter\gdef\expandafter\@freelist\expandafter{\@freelist \@elt\bx@AP}
\ifnum \morefloats@mx> 42 \newinsert\bx@AQ \expandafter\gdef\expandafter\@freelist\expandafter{\@freelist \@elt\bx@AQ}
\ifnum \morefloats@mx> 43 \newinsert\bx@AR \expandafter\gdef\expandafter\@freelist\expandafter{\@freelist \@elt\bx@AR}
\ifnum \morefloats@mx> 44 \newinsert\bx@AS \expandafter\gdef\expandafter\@freelist\expandafter{\@freelist \@elt\bx@AS}
\ifnum \morefloats@mx> 45 \newinsert\bx@AT \expandafter\gdef\expandafter\@freelist\expandafter{\@freelist \@elt\bx@AT}
\ifnum \morefloats@mx> 46 \newinsert\bx@AU \expandafter\gdef\expandafter\@freelist\expandafter{\@freelist \@elt\bx@AU}
\ifnum \morefloats@mx> 47 \newinsert\bx@AV \expandafter\gdef\expandafter\@freelist\expandafter{\@freelist \@elt\bx@AV}
\ifnum \morefloats@mx> 48 \newinsert\bx@AW \expandafter\gdef\expandafter\@freelist\expandafter{\@freelist \@elt\bx@AW}
\ifnum \morefloats@mx> 49 \newinsert\bx@AX \expandafter\gdef\expandafter\@freelist\expandafter{\@freelist \@elt\bx@AX}
\ifnum \morefloats@mx> 50 \newinsert\bx@AY \expandafter\gdef\expandafter\@freelist\expandafter{\@freelist \@elt\bx@AY}
\ifnum \morefloats@mx> 51 \newinsert\bx@AZ \expandafter\gdef\expandafter\@freelist\expandafter{\@freelist \@elt\bx@AZ}
\ifnum \morefloats@mx> 52 \newinsert\bx@BA \expandafter\gdef\expandafter\@freelist\expandafter{\@freelist \@elt\bx@BA}
\ifnum \morefloats@mx> 53 \newinsert\bx@BB \expandafter\gdef\expandafter\@freelist\expandafter{\@freelist \@elt\bx@BB}
\ifnum \morefloats@mx> 54 \newinsert\bx@BC \expandafter\gdef\expandafter\@freelist\expandafter{\@freelist \@elt\bx@BC}
\ifnum \morefloats@mx> 55 \newinsert\bx@BD \expandafter\gdef\expandafter\@freelist\expandafter{\@freelist \@elt\bx@BD}
\ifnum \morefloats@mx> 56 \newinsert\bx@BE \expandafter\gdef\expandafter\@freelist\expandafter{\@freelist \@elt\bx@BE}
\ifnum \morefloats@mx> 57 \newinsert\bx@BF \expandafter\gdef\expandafter\@freelist\expandafter{\@freelist \@elt\bx@BF}
\ifnum \morefloats@mx> 58 \newinsert\bx@BG \expandafter\gdef\expandafter\@freelist\expandafter{\@freelist \@elt\bx@BG}
\ifnum \morefloats@mx> 59 \newinsert\bx@BH \expandafter\gdef\expandafter\@freelist\expandafter{\@freelist \@elt\bx@BH}
\ifnum \morefloats@mx> 60 \newinsert\bx@BI \expandafter\gdef\expandafter\@freelist\expandafter{\@freelist \@elt\bx@BI}
\ifnum \morefloats@mx> 61 \newinsert\bx@BJ \expandafter\gdef\expandafter\@freelist\expandafter{\@freelist \@elt\bx@BJ}
\ifnum \morefloats@mx> 62 \newinsert\bx@BK \expandafter\gdef\expandafter\@freelist\expandafter{\@freelist \@elt\bx@BK}
\ifnum \morefloats@mx> 63 \newinsert\bx@BL \expandafter\gdef\expandafter\@freelist\expandafter{\@freelist \@elt\bx@BL}
\ifnum \morefloats@mx> 64 \newinsert\bx@BM \expandafter\gdef\expandafter\@freelist\expandafter{\@freelist \@elt\bx@BM}
\ifnum \morefloats@mx> 65 \newinsert\bx@BN \expandafter\gdef\expandafter\@freelist\expandafter{\@freelist \@elt\bx@BN}
\ifnum \morefloats@mx> 66 \newinsert\bx@BO \expandafter\gdef\expandafter\@freelist\expandafter{\@freelist \@elt\bx@BO}
\ifnum \morefloats@mx> 67 \newinsert\bx@BP \expandafter\gdef\expandafter\@freelist\expandafter{\@freelist \@elt\bx@BP}
\ifnum \morefloats@mx> 68 \newinsert\bx@BQ \expandafter\gdef\expandafter\@freelist\expandafter{\@freelist \@elt\bx@BQ}
\ifnum \morefloats@mx> 69 \newinsert\bx@BR \expandafter\gdef\expandafter\@freelist\expandafter{\@freelist \@elt\bx@BR}
\ifnum \morefloats@mx> 70 \newinsert\bx@BS \expandafter\gdef\expandafter\@freelist\expandafter{\@freelist \@elt\bx@BS}
\ifnum \morefloats@mx> 71 \newinsert\bx@BT \expandafter\gdef\expandafter\@freelist\expandafter{\@freelist \@elt\bx@BT}
\ifnum \morefloats@mx> 72 \newinsert\bx@BU \expandafter\gdef\expandafter\@freelist\expandafter{\@freelist \@elt\bx@BU}
\ifnum \morefloats@mx> 73 \newinsert\bx@BV \expandafter\gdef\expandafter\@freelist\expandafter{\@freelist \@elt\bx@BV}
\ifnum \morefloats@mx> 74 \newinsert\bx@BW \expandafter\gdef\expandafter\@freelist\expandafter{\@freelist \@elt\bx@BW}
\ifnum \morefloats@mx> 75 \newinsert\bx@BX \expandafter\gdef\expandafter\@freelist\expandafter{\@freelist \@elt\bx@BX}
\ifnum \morefloats@mx> 76 \newinsert\bx@BY \expandafter\gdef\expandafter\@freelist\expandafter{\@freelist \@elt\bx@BY}
\ifnum \morefloats@mx> 77 \newinsert\bx@BZ \expandafter\gdef\expandafter\@freelist\expandafter{\@freelist \@elt\bx@BZ}
\ifnum \morefloats@mx> 78 \newinsert\bx@CA \expandafter\gdef\expandafter\@freelist\expandafter{\@freelist \@elt\bx@CA}
\ifnum \morefloats@mx> 79 \newinsert\bx@CB \expandafter\gdef\expandafter\@freelist\expandafter{\@freelist \@elt\bx@CB}
\ifnum \morefloats@mx> 80 \newinsert\bx@CC \expandafter\gdef\expandafter\@freelist\expandafter{\@freelist \@elt\bx@CC}
\ifnum \morefloats@mx> 81 \newinsert\bx@CD \expandafter\gdef\expandafter\@freelist\expandafter{\@freelist \@elt\bx@CD}
\ifnum \morefloats@mx> 82 \newinsert\bx@CE \expandafter\gdef\expandafter\@freelist\expandafter{\@freelist \@elt\bx@CE}
\ifnum \morefloats@mx> 83 \newinsert\bx@CF \expandafter\gdef\expandafter\@freelist\expandafter{\@freelist \@elt\bx@CF}
\ifnum \morefloats@mx> 84 \newinsert\bx@CG \expandafter\gdef\expandafter\@freelist\expandafter{\@freelist \@elt\bx@CG}
\ifnum \morefloats@mx> 85 \newinsert\bx@CH \expandafter\gdef\expandafter\@freelist\expandafter{\@freelist \@elt\bx@CH}
\ifnum \morefloats@mx> 86 \newinsert\bx@CI \expandafter\gdef\expandafter\@freelist\expandafter{\@freelist \@elt\bx@CI}
\ifnum \morefloats@mx> 87 \newinsert\bx@CJ \expandafter\gdef\expandafter\@freelist\expandafter{\@freelist \@elt\bx@CJ}
\ifnum \morefloats@mx> 88 \newinsert\bx@CK \expandafter\gdef\expandafter\@freelist\expandafter{\@freelist \@elt\bx@CK}
\ifnum \morefloats@mx> 89 \newinsert\bx@CL \expandafter\gdef\expandafter\@freelist\expandafter{\@freelist \@elt\bx@CL}
\ifnum \morefloats@mx> 90 \newinsert\bx@CM \expandafter\gdef\expandafter\@freelist\expandafter{\@freelist \@elt\bx@CM}
\ifnum \morefloats@mx> 91 \newinsert\bx@CN \expandafter\gdef\expandafter\@freelist\expandafter{\@freelist \@elt\bx@CN}
\ifnum \morefloats@mx> 92 \newinsert\bx@CO \expandafter\gdef\expandafter\@freelist\expandafter{\@freelist \@elt\bx@CO}
\ifnum \morefloats@mx> 93 \newinsert\bx@CP \expandafter\gdef\expandafter\@freelist\expandafter{\@freelist \@elt\bx@CP}
\ifnum \morefloats@mx> 94 \newinsert\bx@CQ \expandafter\gdef\expandafter\@freelist\expandafter{\@freelist \@elt\bx@CQ}
\ifnum \morefloats@mx> 95 \newinsert\bx@CR \expandafter\gdef\expandafter\@freelist\expandafter{\@freelist \@elt\bx@CR}
\ifnum \morefloats@mx> 96 \newinsert\bx@CS \expandafter\gdef\expandafter\@freelist\expandafter{\@freelist \@elt\bx@CS}
\ifnum \morefloats@mx> 97 \newinsert\bx@CT \expandafter\gdef\expandafter\@freelist\expandafter{\@freelist \@elt\bx@CT}
\ifnum \morefloats@mx> 98 \newinsert\bx@CU \expandafter\gdef\expandafter\@freelist\expandafter{\@freelist \@elt\bx@CU}
\ifnum \morefloats@mx> 99 \newinsert\bx@CV \expandafter\gdef\expandafter\@freelist\expandafter{\@freelist \@elt\bx@CV}
\ifnum \morefloats@mx>100 \newinsert\bx@CW \expandafter\gdef\expandafter\@freelist\expandafter{\@freelist \@elt\bx@CW}
\ifnum \morefloats@mx>101 \newinsert\bx@CX \expandafter\gdef\expandafter\@freelist\expandafter{\@freelist \@elt\bx@CX}
\ifnum \morefloats@mx>102 \newinsert\bx@CY \expandafter\gdef\expandafter\@freelist\expandafter{\@freelist \@elt\bx@CY}
\ifnum \morefloats@mx>103 \newinsert\bx@CZ \expandafter\gdef\expandafter\@freelist\expandafter{\@freelist \@elt\bx@CZ}
\ifnum \morefloats@mx>104 \newinsert\bx@DA \expandafter\gdef\expandafter\@freelist\expandafter{\@freelist \@elt\bx@DA}
\ifnum \morefloats@mx>105 \newinsert\bx@DB \expandafter\gdef\expandafter\@freelist\expandafter{\@freelist \@elt\bx@DB}
\ifnum \morefloats@mx>106 \newinsert\bx@DC \expandafter\gdef\expandafter\@freelist\expandafter{\@freelist \@elt\bx@DC}
\ifnum \morefloats@mx>107 \newinsert\bx@DD \expandafter\gdef\expandafter\@freelist\expandafter{\@freelist \@elt\bx@DD}
\ifnum \morefloats@mx>108 \newinsert\bx@DE \expandafter\gdef\expandafter\@freelist\expandafter{\@freelist \@elt\bx@DE}
\ifnum \morefloats@mx>109 \newinsert\bx@DF \expandafter\gdef\expandafter\@freelist\expandafter{\@freelist \@elt\bx@DF}
\ifnum \morefloats@mx>110 \newinsert\bx@DG \expandafter\gdef\expandafter\@freelist\expandafter{\@freelist \@elt\bx@DG}
\ifnum \morefloats@mx>111 \newinsert\bx@DH \expandafter\gdef\expandafter\@freelist\expandafter{\@freelist \@elt\bx@DH}
\ifnum \morefloats@mx>112 \newinsert\bx@DI \expandafter\gdef\expandafter\@freelist\expandafter{\@freelist \@elt\bx@DI}
\ifnum \morefloats@mx>113 \newinsert\bx@DJ \expandafter\gdef\expandafter\@freelist\expandafter{\@freelist \@elt\bx@DJ}
\ifnum \morefloats@mx>114 \newinsert\bx@DK \expandafter\gdef\expandafter\@freelist\expandafter{\@freelist \@elt\bx@DK}
\ifnum \morefloats@mx>115 \newinsert\bx@DL \expandafter\gdef\expandafter\@freelist\expandafter{\@freelist \@elt\bx@DL}
\ifnum \morefloats@mx>116 \newinsert\bx@DM \expandafter\gdef\expandafter\@freelist\expandafter{\@freelist \@elt\bx@DM}
\ifnum \morefloats@mx>117 \newinsert\bx@DN \expandafter\gdef\expandafter\@freelist\expandafter{\@freelist \@elt\bx@DN}
\ifnum \morefloats@mx>118 \newinsert\bx@DO \expandafter\gdef\expandafter\@freelist\expandafter{\@freelist \@elt\bx@DO}
\ifnum \morefloats@mx>119 \newinsert\bx@DP \expandafter\gdef\expandafter\@freelist\expandafter{\@freelist \@elt\bx@DP}
\ifnum \morefloats@mx>120 \newinsert\bx@DQ \expandafter\gdef\expandafter\@freelist\expandafter{\@freelist \@elt\bx@DQ}
\ifnum \morefloats@mx>121 \newinsert\bx@DR \expandafter\gdef\expandafter\@freelist\expandafter{\@freelist \@elt\bx@DR}
\ifnum \morefloats@mx>122 \newinsert\bx@DS \expandafter\gdef\expandafter\@freelist\expandafter{\@freelist \@elt\bx@DS}
\ifnum \morefloats@mx>123 \newinsert\bx@DT \expandafter\gdef\expandafter\@freelist\expandafter{\@freelist \@elt\bx@DT}
\ifnum \morefloats@mx>124 \newinsert\bx@DU \expandafter\gdef\expandafter\@freelist\expandafter{\@freelist \@elt\bx@DU}
\ifnum \morefloats@mx>125 \newinsert\bx@DV \expandafter\gdef\expandafter\@freelist\expandafter{\@freelist \@elt\bx@DV}
\ifnum \morefloats@mx>126 \newinsert\bx@DW \expandafter\gdef\expandafter\@freelist\expandafter{\@freelist \@elt\bx@DW}
\ifnum \morefloats@mx>127 \newinsert\bx@DX \expandafter\gdef\expandafter\@freelist\expandafter{\@freelist \@elt\bx@DX}
\ifnum \morefloats@mx>128 \newinsert\bx@DY \expandafter\gdef\expandafter\@freelist\expandafter{\@freelist \@elt\bx@DY}
\ifnum \morefloats@mx>129 \newinsert\bx@DZ \expandafter\gdef\expandafter\@freelist\expandafter{\@freelist \@elt\bx@DZ}
\ifnum \morefloats@mx>130 \newinsert\bx@EA \expandafter\gdef\expandafter\@freelist\expandafter{\@freelist \@elt\bx@EA}
\ifnum \morefloats@mx>131 \newinsert\bx@EB \expandafter\gdef\expandafter\@freelist\expandafter{\@freelist \@elt\bx@EB}
\ifnum \morefloats@mx>132 \newinsert\bx@EC \expandafter\gdef\expandafter\@freelist\expandafter{\@freelist \@elt\bx@EC}
\ifnum \morefloats@mx>133 \newinsert\bx@ED \expandafter\gdef\expandafter\@freelist\expandafter{\@freelist \@elt\bx@ED}
\ifnum \morefloats@mx>134 \newinsert\bx@EE \expandafter\gdef\expandafter\@freelist\expandafter{\@freelist \@elt\bx@EE}
\ifnum \morefloats@mx>135 \newinsert\bx@EF \expandafter\gdef\expandafter\@freelist\expandafter{\@freelist \@elt\bx@EF}
\ifnum \morefloats@mx>136 \newinsert\bx@EG \expandafter\gdef\expandafter\@freelist\expandafter{\@freelist \@elt\bx@EG}
\ifnum \morefloats@mx>137 \newinsert\bx@EH \expandafter\gdef\expandafter\@freelist\expandafter{\@freelist \@elt\bx@EH}
\ifnum \morefloats@mx>138 \newinsert\bx@EI \expandafter\gdef\expandafter\@freelist\expandafter{\@freelist \@elt\bx@EI}
\ifnum \morefloats@mx>139 \newinsert\bx@EJ \expandafter\gdef\expandafter\@freelist\expandafter{\@freelist \@elt\bx@EJ}
\ifnum \morefloats@mx>140 \newinsert\bx@EK \expandafter\gdef\expandafter\@freelist\expandafter{\@freelist \@elt\bx@EK}
\ifnum \morefloats@mx>141 \newinsert\bx@EL \expandafter\gdef\expandafter\@freelist\expandafter{\@freelist \@elt\bx@EL}
\ifnum \morefloats@mx>142 \newinsert\bx@EM \expandafter\gdef\expandafter\@freelist\expandafter{\@freelist \@elt\bx@EM}
\ifnum \morefloats@mx>143 \newinsert\bx@EN \expandafter\gdef\expandafter\@freelist\expandafter{\@freelist \@elt\bx@EN}
\ifnum \morefloats@mx>144 \newinsert\bx@EO \expandafter\gdef\expandafter\@freelist\expandafter{\@freelist \@elt\bx@EO}
\ifnum \morefloats@mx>145 \newinsert\bx@EP \expandafter\gdef\expandafter\@freelist\expandafter{\@freelist \@elt\bx@EP}
\ifnum \morefloats@mx>146 \newinsert\bx@EQ \expandafter\gdef\expandafter\@freelist\expandafter{\@freelist \@elt\bx@EQ}
\ifnum \morefloats@mx>147 \newinsert\bx@ER \expandafter\gdef\expandafter\@freelist\expandafter{\@freelist \@elt\bx@ER}
\ifnum \morefloats@mx>148 \newinsert\bx@ES \expandafter\gdef\expandafter\@freelist\expandafter{\@freelist \@elt\bx@ES}
\ifnum \morefloats@mx>149 \newinsert\bx@ET \expandafter\gdef\expandafter\@freelist\expandafter{\@freelist \@elt\bx@ET}
\ifnum \morefloats@mx>150 \newinsert\bx@EU \expandafter\gdef\expandafter\@freelist\expandafter{\@freelist \@elt\bx@EU}
\ifnum \morefloats@mx>151 \newinsert\bx@EV \expandafter\gdef\expandafter\@freelist\expandafter{\@freelist \@elt\bx@EV}
\ifnum \morefloats@mx>152 \newinsert\bx@EW \expandafter\gdef\expandafter\@freelist\expandafter{\@freelist \@elt\bx@EW}
\ifnum \morefloats@mx>153 \newinsert\bx@EX \expandafter\gdef\expandafter\@freelist\expandafter{\@freelist \@elt\bx@EX}
\ifnum \morefloats@mx>154 \newinsert\bx@EY \expandafter\gdef\expandafter\@freelist\expandafter{\@freelist \@elt\bx@EY}
\ifnum \morefloats@mx>155 \newinsert\bx@EZ \expandafter\gdef\expandafter\@freelist\expandafter{\@freelist \@elt\bx@EZ}
\ifnum \morefloats@mx>156 \newinsert\bx@FA \expandafter\gdef\expandafter\@freelist\expandafter{\@freelist \@elt\bx@FA}
\ifnum \morefloats@mx>157 \newinsert\bx@FB \expandafter\gdef\expandafter\@freelist\expandafter{\@freelist \@elt\bx@FB}
\ifnum \morefloats@mx>158 \newinsert\bx@FC \expandafter\gdef\expandafter\@freelist\expandafter{\@freelist \@elt\bx@FC}
\ifnum \morefloats@mx>159 \newinsert\bx@FD \expandafter\gdef\expandafter\@freelist\expandafter{\@freelist \@elt\bx@FD}
\ifnum \morefloats@mx>160 \newinsert\bx@FE \expandafter\gdef\expandafter\@freelist\expandafter{\@freelist \@elt\bx@FE}
\ifnum \morefloats@mx>161 \newinsert\bx@FF \expandafter\gdef\expandafter\@freelist\expandafter{\@freelist \@elt\bx@FF}
\ifnum \morefloats@mx>162 \newinsert\bx@FG \expandafter\gdef\expandafter\@freelist\expandafter{\@freelist \@elt\bx@FG}
\ifnum \morefloats@mx>163 \newinsert\bx@FH \expandafter\gdef\expandafter\@freelist\expandafter{\@freelist \@elt\bx@FH}
\ifnum \morefloats@mx>164 \newinsert\bx@FI \expandafter\gdef\expandafter\@freelist\expandafter{\@freelist \@elt\bx@FI}
\ifnum \morefloats@mx>165 \newinsert\bx@FJ \expandafter\gdef\expandafter\@freelist\expandafter{\@freelist \@elt\bx@FJ}
\ifnum \morefloats@mx>166 \newinsert\bx@FK \expandafter\gdef\expandafter\@freelist\expandafter{\@freelist \@elt\bx@FK}
\ifnum \morefloats@mx>167 \newinsert\bx@FL \expandafter\gdef\expandafter\@freelist\expandafter{\@freelist \@elt\bx@FL}
\ifnum \morefloats@mx>168 \newinsert\bx@FM \expandafter\gdef\expandafter\@freelist\expandafter{\@freelist \@elt\bx@FM}
\ifnum \morefloats@mx>169 \newinsert\bx@FN \expandafter\gdef\expandafter\@freelist\expandafter{\@freelist \@elt\bx@FN}
\ifnum \morefloats@mx>170 \newinsert\bx@FO \expandafter\gdef\expandafter\@freelist\expandafter{\@freelist \@elt\bx@FO}
\ifnum \morefloats@mx>171 \newinsert\bx@FP \expandafter\gdef\expandafter\@freelist\expandafter{\@freelist \@elt\bx@FP}
\ifnum \morefloats@mx>172 \newinsert\bx@FQ \expandafter\gdef\expandafter\@freelist\expandafter{\@freelist \@elt\bx@FQ}
\ifnum \morefloats@mx>173 \newinsert\bx@FR \expandafter\gdef\expandafter\@freelist\expandafter{\@freelist \@elt\bx@FR}
\ifnum \morefloats@mx>174 \newinsert\bx@FS \expandafter\gdef\expandafter\@freelist\expandafter{\@freelist \@elt\bx@FS}
\ifnum \morefloats@mx>175 \newinsert\bx@FT \expandafter\gdef\expandafter\@freelist\expandafter{\@freelist \@elt\bx@FT}
\ifnum \morefloats@mx>176 \newinsert\bx@FU \expandafter\gdef\expandafter\@freelist\expandafter{\@freelist \@elt\bx@FU}
\ifnum \morefloats@mx>177 \newinsert\bx@FV \expandafter\gdef\expandafter\@freelist\expandafter{\@freelist \@elt\bx@FV}
\ifnum \morefloats@mx>178 \newinsert\bx@FW \expandafter\gdef\expandafter\@freelist\expandafter{\@freelist \@elt\bx@FW}
\ifnum \morefloats@mx>179 \newinsert\bx@FX \expandafter\gdef\expandafter\@freelist\expandafter{\@freelist \@elt\bx@FX}
\ifnum \morefloats@mx>180 \newinsert\bx@FY \expandafter\gdef\expandafter\@freelist\expandafter{\@freelist \@elt\bx@FY}
\ifnum \morefloats@mx>181 \newinsert\bx@FZ \expandafter\gdef\expandafter\@freelist\expandafter{\@freelist \@elt\bx@FZ}
\ifnum \morefloats@mx>182 \newinsert\bx@GA \expandafter\gdef\expandafter\@freelist\expandafter{\@freelist \@elt\bx@GA}
\ifnum \morefloats@mx>183 \newinsert\bx@GB \expandafter\gdef\expandafter\@freelist\expandafter{\@freelist \@elt\bx@GB}
\ifnum \morefloats@mx>184 \newinsert\bx@GC \expandafter\gdef\expandafter\@freelist\expandafter{\@freelist \@elt\bx@GC}
\ifnum \morefloats@mx>185 \newinsert\bx@GD \expandafter\gdef\expandafter\@freelist\expandafter{\@freelist \@elt\bx@GD}
\ifnum \morefloats@mx>186 \newinsert\bx@GE \expandafter\gdef\expandafter\@freelist\expandafter{\@freelist \@elt\bx@GE}
\ifnum \morefloats@mx>187 \newinsert\bx@GF \expandafter\gdef\expandafter\@freelist\expandafter{\@freelist \@elt\bx@GF}
\ifnum \morefloats@mx>188 \newinsert\bx@GG \expandafter\gdef\expandafter\@freelist\expandafter{\@freelist \@elt\bx@GG}
\ifnum \morefloats@mx>189 \newinsert\bx@GH \expandafter\gdef\expandafter\@freelist\expandafter{\@freelist \@elt\bx@GH}
\ifnum \morefloats@mx>190 \newinsert\bx@GI \expandafter\gdef\expandafter\@freelist\expandafter{\@freelist \@elt\bx@GI}
\ifnum \morefloats@mx>191 \newinsert\bx@GJ \expandafter\gdef\expandafter\@freelist\expandafter{\@freelist \@elt\bx@GJ}
\ifnum \morefloats@mx>192 \newinsert\bx@GK \expandafter\gdef\expandafter\@freelist\expandafter{\@freelist \@elt\bx@GK}
\ifnum \morefloats@mx>193 \newinsert\bx@GL \expandafter\gdef\expandafter\@freelist\expandafter{\@freelist \@elt\bx@GL}
\ifnum \morefloats@mx>194 \newinsert\bx@GM \expandafter\gdef\expandafter\@freelist\expandafter{\@freelist \@elt\bx@GM}
\ifnum \morefloats@mx>195 \newinsert\bx@GN \expandafter\gdef\expandafter\@freelist\expandafter{\@freelist \@elt\bx@GN}
\ifnum \morefloats@mx>196 \newinsert\bx@GO \expandafter\gdef\expandafter\@freelist\expandafter{\@freelist \@elt\bx@GO}
\ifnum \morefloats@mx>197 \newinsert\bx@GP \expandafter\gdef\expandafter\@freelist\expandafter{\@freelist \@elt\bx@GP}
\ifnum \morefloats@mx>198 \newinsert\bx@GQ \expandafter\gdef\expandafter\@freelist\expandafter{\@freelist \@elt\bx@GQ}
\ifnum \morefloats@mx>199 \newinsert\bx@GR \expandafter\gdef\expandafter\@freelist\expandafter{\@freelist \@elt\bx@GR}
\ifnum \morefloats@mx>200 \newinsert\bx@GS \expandafter\gdef\expandafter\@freelist\expandafter{\@freelist \@elt\bx@GS}
\ifnum \morefloats@mx>201 \newinsert\bx@GT \expandafter\gdef\expandafter\@freelist\expandafter{\@freelist \@elt\bx@GT}
\ifnum \morefloats@mx>202 \newinsert\bx@GU \expandafter\gdef\expandafter\@freelist\expandafter{\@freelist \@elt\bx@GU}
\ifnum \morefloats@mx>203 \newinsert\bx@GV \expandafter\gdef\expandafter\@freelist\expandafter{\@freelist \@elt\bx@GV}
\ifnum \morefloats@mx>204 \newinsert\bx@GW \expandafter\gdef\expandafter\@freelist\expandafter{\@freelist \@elt\bx@GW}
\ifnum \morefloats@mx>205 \newinsert\bx@GX \expandafter\gdef\expandafter\@freelist\expandafter{\@freelist \@elt\bx@GX}
\ifnum \morefloats@mx>206 \newinsert\bx@GY \expandafter\gdef\expandafter\@freelist\expandafter{\@freelist \@elt\bx@GY}
\ifnum \morefloats@mx>207 \newinsert\bx@GZ \expandafter\gdef\expandafter\@freelist\expandafter{\@freelist \@elt\bx@GZ}
\ifnum \morefloats@mx>208 \newinsert\bx@HA \expandafter\gdef\expandafter\@freelist\expandafter{\@freelist \@elt\bx@HA}
\ifnum \morefloats@mx>209 \newinsert\bx@HB \expandafter\gdef\expandafter\@freelist\expandafter{\@freelist \@elt\bx@HB}
\ifnum \morefloats@mx>210 \newinsert\bx@HC \expandafter\gdef\expandafter\@freelist\expandafter{\@freelist \@elt\bx@HC}
\ifnum \morefloats@mx>211 \newinsert\bx@HD \expandafter\gdef\expandafter\@freelist\expandafter{\@freelist \@elt\bx@HD}
\ifnum \morefloats@mx>212 \newinsert\bx@HE \expandafter\gdef\expandafter\@freelist\expandafter{\@freelist \@elt\bx@HE}
\ifnum \morefloats@mx>213 \newinsert\bx@HF \expandafter\gdef\expandafter\@freelist\expandafter{\@freelist \@elt\bx@HF}
\ifnum \morefloats@mx>214 \newinsert\bx@HG \expandafter\gdef\expandafter\@freelist\expandafter{\@freelist \@elt\bx@HG}
\ifnum \morefloats@mx>215 \newinsert\bx@HH \expandafter\gdef\expandafter\@freelist\expandafter{\@freelist \@elt\bx@HH}
\ifnum \morefloats@mx>216 \newinsert\bx@HI \expandafter\gdef\expandafter\@freelist\expandafter{\@freelist \@elt\bx@HI}
\ifnum \morefloats@mx>217 \newinsert\bx@HJ \expandafter\gdef\expandafter\@freelist\expandafter{\@freelist \@elt\bx@HJ}
\ifnum \morefloats@mx>218 \newinsert\bx@HK \expandafter\gdef\expandafter\@freelist\expandafter{\@freelist \@elt\bx@HK}
\ifnum \morefloats@mx>219 \newinsert\bx@HL \expandafter\gdef\expandafter\@freelist\expandafter{\@freelist \@elt\bx@HL}
\ifnum \morefloats@mx>220 \newinsert\bx@HM \expandafter\gdef\expandafter\@freelist\expandafter{\@freelist \@elt\bx@HM}
\ifnum \morefloats@mx>221 \newinsert\bx@HN \expandafter\gdef\expandafter\@freelist\expandafter{\@freelist \@elt\bx@HN}
\ifnum \morefloats@mx>222 \newinsert\bx@HO \expandafter\gdef\expandafter\@freelist\expandafter{\@freelist \@elt\bx@HO}
\ifnum \morefloats@mx>223 \newinsert\bx@HP \expandafter\gdef\expandafter\@freelist\expandafter{\@freelist \@elt\bx@HP}
\ifnum \morefloats@mx>224 \newinsert\bx@HQ \expandafter\gdef\expandafter\@freelist\expandafter{\@freelist \@elt\bx@HQ}
\ifnum \morefloats@mx>225 \newinsert\bx@HR \expandafter\gdef\expandafter\@freelist\expandafter{\@freelist \@elt\bx@HR}
\ifnum \morefloats@mx>226 \newinsert\bx@HS \expandafter\gdef\expandafter\@freelist\expandafter{\@freelist \@elt\bx@HS}
\ifnum \morefloats@mx>227 \newinsert\bx@HT \expandafter\gdef\expandafter\@freelist\expandafter{\@freelist \@elt\bx@HT}
\ifnum \morefloats@mx>228 \newinsert\bx@HU \expandafter\gdef\expandafter\@freelist\expandafter{\@freelist \@elt\bx@HU}
\ifnum \morefloats@mx>229 \newinsert\bx@HV \expandafter\gdef\expandafter\@freelist\expandafter{\@freelist \@elt\bx@HV}
\ifnum \morefloats@mx>230 \newinsert\bx@HW \expandafter\gdef\expandafter\@freelist\expandafter{\@freelist \@elt\bx@HW}
\ifnum \morefloats@mx>231 \newinsert\bx@HX \expandafter\gdef\expandafter\@freelist\expandafter{\@freelist \@elt\bx@HX}
\ifnum \morefloats@mx>232 \newinsert\bx@HY \expandafter\gdef\expandafter\@freelist\expandafter{\@freelist \@elt\bx@HY}
\ifnum \morefloats@mx>233 \newinsert\bx@HZ \expandafter\gdef\expandafter\@freelist\expandafter{\@freelist \@elt\bx@HZ}
\ifnum \morefloats@mx>234 \newinsert\bx@IA \expandafter\gdef\expandafter\@freelist\expandafter{\@freelist \@elt\bx@IA}
\ifnum \morefloats@mx>235 \newinsert\bx@IB \expandafter\gdef\expandafter\@freelist\expandafter{\@freelist \@elt\bx@IB}
\ifnum \morefloats@mx>236 \newinsert\bx@IC \expandafter\gdef\expandafter\@freelist\expandafter{\@freelist \@elt\bx@IC}
\ifnum \morefloats@mx>237 \newinsert\bx@ID \expandafter\gdef\expandafter\@freelist\expandafter{\@freelist \@elt\bx@ID}
\ifnum \morefloats@mx>238 \newinsert\bx@IE \expandafter\gdef\expandafter\@freelist\expandafter{\@freelist \@elt\bx@IE}
\ifnum \morefloats@mx>239 \newinsert\bx@IF \expandafter\gdef\expandafter\@freelist\expandafter{\@freelist \@elt\bx@IF}
\ifnum \morefloats@mx>240 \newinsert\bx@IG \expandafter\gdef\expandafter\@freelist\expandafter{\@freelist \@elt\bx@IG}
\ifnum \morefloats@mx>241 \newinsert\bx@IH \expandafter\gdef\expandafter\@freelist\expandafter{\@freelist \@elt\bx@IH}
\ifnum \morefloats@mx>242 \newinsert\bx@II \expandafter\gdef\expandafter\@freelist\expandafter{\@freelist \@elt\bx@II}
\ifnum \morefloats@mx>243 \newinsert\bx@IJ \expandafter\gdef\expandafter\@freelist\expandafter{\@freelist \@elt\bx@IJ}
\ifnum \morefloats@mx>244 \newinsert\bx@IK \expandafter\gdef\expandafter\@freelist\expandafter{\@freelist \@elt\bx@IK}
\ifnum \morefloats@mx>245 \newinsert\bx@IL \expandafter\gdef\expandafter\@freelist\expandafter{\@freelist \@elt\bx@IL}
\ifnum \morefloats@mx>246 \newinsert\bx@IM \expandafter\gdef\expandafter\@freelist\expandafter{\@freelist \@elt\bx@IM}
\ifnum \morefloats@mx>247 \newinsert\bx@IN \expandafter\gdef\expandafter\@freelist\expandafter{\@freelist \@elt\bx@IN}
\ifnum \morefloats@mx>248 \newinsert\bx@IO \expandafter\gdef\expandafter\@freelist\expandafter{\@freelist \@elt\bx@IO}
\ifnum \morefloats@mx>249 \newinsert\bx@IP \expandafter\gdef\expandafter\@freelist\expandafter{\@freelist \@elt\bx@IP}
\ifnum \morefloats@mx>250 \newinsert\bx@IQ \expandafter\gdef\expandafter\@freelist\expandafter{\@freelist \@elt\bx@IQ}
\ifnum \morefloats@mx>251 \newinsert\bx@IR \expandafter\gdef\expandafter\@freelist\expandafter{\@freelist \@elt\bx@IR}
\ifnum \morefloats@mx>252 \newinsert\bx@IS \expandafter\gdef\expandafter\@freelist\expandafter{\@freelist \@elt\bx@IS}
\ifnum \morefloats@mx>253 \newinsert\bx@IT \expandafter\gdef\expandafter\@freelist\expandafter{\@freelist \@elt\bx@IT}
\ifnum \morefloats@mx>254 \newinsert\bx@IU \expandafter\gdef\expandafter\@freelist\expandafter{\@freelist \@elt\bx@IU}
\ifnum \morefloats@mx>255 \newinsert\bx@IV \expandafter\gdef\expandafter\@freelist\expandafter{\@freelist \@elt\bx@IV}
%    \end{macrocode}
%
% \newpage
%
%    \begin{macrocode}
\ifnum \morefloats@mx>256\relax%
  \PackageError{morefloats}{Too many floats called for}{%
    You requested more than 256 floats.\MessageBreak%
    (\morefloats@mx\space to be precise.)\MessageBreak%
    LaTeX before 2015 could not process\MessageBreak%
    more than 256 floats, therefore the morefloats\MessageBreak%
    package only provides 256 floats.\MessageBreak%
    If you need more floats,\MessageBreak%
    update to a current (>=2015) LaTeX distribution.\MessageBreak%
    I expected LaTeX (prior 2015) to run out of dimensions\MessageBreak%
    or memory long before reaching 256 floats anyway.\MessageBreak%
   }%
\fi \fi \fi \fi \fi \fi \fi \fi \fi \fi \fi \fi \fi \fi \fi \fi \fi \fi
\fi \fi \fi \fi \fi \fi \fi \fi \fi \fi \fi \fi \fi \fi \fi \fi \fi \fi
\fi \fi \fi \fi \fi \fi \fi \fi \fi \fi \fi \fi \fi \fi \fi \fi \fi \fi
\fi \fi \fi \fi \fi \fi \fi \fi \fi \fi \fi \fi \fi \fi \fi \fi \fi \fi
\fi \fi \fi \fi \fi \fi \fi \fi \fi \fi \fi \fi \fi \fi \fi \fi \fi \fi
\fi \fi \fi \fi \fi \fi \fi \fi \fi \fi \fi \fi \fi \fi \fi \fi \fi \fi
\fi \fi \fi \fi \fi \fi \fi \fi \fi \fi \fi \fi \fi \fi \fi \fi \fi \fi
\fi \fi \fi \fi \fi \fi \fi \fi \fi \fi \fi \fi \fi \fi \fi \fi \fi \fi
\fi \fi \fi \fi \fi \fi \fi \fi \fi \fi \fi \fi \fi \fi \fi \fi \fi \fi
\fi \fi \fi \fi \fi \fi \fi \fi \fi \fi \fi \fi \fi \fi \fi \fi \fi \fi
\fi \fi \fi \fi \fi \fi \fi \fi \fi \fi \fi \fi \fi \fi \fi \fi \fi \fi
\fi \fi \fi \fi \fi \fi \fi \fi \fi \fi \fi \fi \fi \fi \fi \fi \fi \fi
\fi \fi \fi \fi \fi \fi \fi \fi \fi \fi \fi \fi \fi \fi \fi \fi \fi \fi
\fi \fi \fi \fi \fi

%    \end{macrocode}
%
%    \begin{macrocode}
%</package>
%    \end{macrocode}
%
% \end{landscape}
% \newpage
%
% \section{Installation}
%
% \subsection{Downloads\label{ss:Downloads}}
%
% Everything is available at \url{https://www.ctan.org},
% but may need additional packages themselves.\\
%
% \DescribeMacro{morefloats.dtx}
% For unpacking the |morefloats.dtx| file and constructing the documentation it is required:
% \begin{description}
% \item[-] \TeX Format \LaTeXe{}: \url{https://www.CTAN.org}
%
% \item[-] document class \xclass{ltxdoc}, 2015/03/26, v2.0w,
%   \url{https://www.ctan.org/pkg/ltxdoc}
%
% \item[-] package \xpackage{fontenc}, 2005/09/27, v1.99g,
%   \url{https://ctan.org/pkg/fontenc}
%
% \item[-] package \xpackage{pdflscape}, 2008/08/11, v0.10,
%   \url{https://ctan.org/pkg/pdflscape}
%
% \item[-] package \xpackage{holtxdoc}, 2012/03/21, v0.24,
%   \url{https://ctan.org/pkg/holtxdoc}
%
% \item[-] package \xpackage{hypdoc}, 2011/08/19, v1.11,
%   \url{https://ctan.org/pkg/hypdoc}
% \end{description}
%
% \DescribeMacro{morefloats.sty}
% The \texttt{morefloats.sty} for \LaTeXe{} \hbox{(i.\,e. each} document using
% the \xpackage{morefloats} package) requires:
% \begin{description}
% \item[-] \TeX Format \LaTeXe{}, \url{https://www.CTAN.org/}
%
% \item[-] package \xpackage{kvoptions}, 2011/06/30, v3.11,
%   \url{https://ctan.org/pkg/kvoptions}
%
% \item[-] package \xpackage{ifetex}, 2011/12/15, v1.2,
%   \url{https://ctan.org/pkg/ifetex}, is used in some cases
% \end{description}
%
% \DescribeMacro{regstats}
% \DescribeMacro{regcount}
% To check the number of used registers it was mentioned:
% \begin{description}
% \item[-] package \xpackage{regstats}, \url{https://ctan.org/pkg/regstats}
% \item[-] package \xpackage{regcount}, \url{https://ctan.org/pkg/regcount}
% \end{description}
%
% \DescribeMacro{Oberdiek}
% \DescribeMacro{holtxdoc}
% \DescribeMacro{hypdoc}
% All packages of \textsc{Heiko Oberdiek}'s bundle `oberdiek'
% (especially \xpackage{holtxdoc}, \xpackage{hypdoc}, and \xpackage{kvoptions})
% are also available in a TDS compliant ZIP archive:\\
% \url{http://mirror.ctan.org/install/macros/latex/contrib/oberdiek.tds.zip}.\\
% It is probably best to download and use this, because the packages in there
% are quite probably both recent and compatible among themselves.\\
%
% \DescribeMacro{hyperref}
% \noindent \xpackage{hyperref} is not included in that bundle and needs to be
% downloaded separately,\\
% \url{http://mirror.ctan.org/install/macros/latex/contrib/hyperref.tds.zip}.\\
%
% \DescribeMacro{M\"{u}nch}
% A hyperlinked list of my (other) packages can be found at
% \url{https://www.ctan.org/author/muench-hm}.\\
%
% \subsection{Package, unpacking TDS}
% \paragraph{Package.} This package is available on \url{https://www.CTAN.org}.
% \begin{description}
% \item[\url{http://mirror.ctan.org/macros/latex/contrib/morefloats/morefloats.dtx}]\hspace*{0.1cm}
%       The source file.
% \item[\url{http://mirror.ctan.org/macros/latex/contrib/morefloats/morefloats.pdf}]\hspace*{0.1cm}
%       The documentation.
% \item[\url{http://mirror.ctan.org/macros/latex/contrib/morefloats/README}]\hspace*{0.1cm}\\
%       \hspace*{1em}The README file.
% \end{description}
%
% \noindent There is also a |morefloats.tds.zip| available:
% \begin{description}
% \item[\url{http://mirror.ctan.org/install/macros/latex/contrib/morefloats.tds.zip}]\hspace*{0.1cm}
%       Everything in TDS compliant, compiled format.
% \end{description}
% which additionally contains\\
% \begin{tabular}{ll}
% morefloats.ins & The installation file.\\
% morefloats.drv & The driver to generate the documentation.\\
% morefloats.sty & The \xext{sty}le file.\\
% morefloats-example.tex & The example file.\\
% morefloats-example.pdf & The compiled example file.
% \end{tabular}
%
% \bigskip
%
% \noindent For required other packages, please see the preceding subsection.
%
% \paragraph{Unpacking.} The  \xfile{.dtx} file is a self-extracting
% \docstrip{} archive. The files are extracted by running the
% \xfile{.dtx} through \plainTeX{}:
% \begin{quote}
%   \verb|tex morefloats.dtx|
% \end{quote}
%
% About generating the documentation see paragraph~\ref{GenDoc} below.\\
%
% \paragraph{TDS.} Now the different files must be moved into
% the different directories in your installation TDS tree
% (also known as \xfile{texmf} tree):
% \begin{quote}
% \def\t{^^A
% \begin{tabular}{@{}>{\ttfamily}l@{ $\rightarrow$ }>{\ttfamily}l@{}}
%   morefloats.sty & tex/latex/morefloats/morefloats.sty\\
%   morefloats.pdf & doc/latex/morefloats/morefloats.pdf\\
%   morefloats-example.tex & doc/latex/morefloats/morefloats-example.tex\\
%   morefloats-example.pdf & doc/latex/morefloats/morefloats-example.pdf\\
%   morefloats.dtx & source/latex/morefloats/morefloats.dtx\\
% \end{tabular}^^A
% }^^A
% \sbox0{\t}^^A
% \ifdim\wd0>\linewidth
%   \begingroup
%     \advance\linewidth by\leftmargin
%     \advance\linewidth by\rightmargin
%   \edef\x{\endgroup
%     \def\noexpand\lw{\the\linewidth}^^A
%   }\x
%   \def\lwbox{^^A
%     \leavevmode
%     \hbox to \linewidth{^^A
%       \kern-\leftmargin\relax
%       \hss
%       \usebox0
%       \hss
%       \kern-\rightmargin\relax
%     }^^A
%   }^^A
%   \ifdim\wd0>\lw
%     \sbox0{\small\t}^^A
%     \ifdim\wd0>\linewidth
%       \ifdim\wd0>\lw
%         \sbox0{\footnotesize\t}^^A
%         \ifdim\wd0>\linewidth
%           \ifdim\wd0>\lw
%             \sbox0{\scriptsize\t}^^A
%             \ifdim\wd0>\linewidth
%               \ifdim\wd0>\lw
%                 \sbox0{\tiny\t}^^A
%                 \ifdim\wd0>\linewidth
%                   \lwbox
%                 \else
%                   \usebox0
%                 \fi
%               \else
%                 \lwbox
%               \fi
%             \else
%               \usebox0
%             \fi
%           \else
%             \lwbox
%           \fi
%         \else
%           \usebox0
%         \fi
%       \else
%         \lwbox
%       \fi
%     \else
%       \usebox0
%     \fi
%   \else
%     \lwbox
%   \fi
% \else
%   \usebox0
% \fi
% \end{quote}
% If you have a \xfile{docstrip.cfg} that configures and enables \docstrip's
% TDS installing feature, then some files can already be in the right
% place, see the documentation of \docstrip{}.
%
% \subsection{Refresh file name databases}
%
% If your \TeX~distribution (\TeX{} Live, \mikTeX, \teTeX, \dots) relies on
% file name databases, you must refresh these. For example, \teTeX{} users run
% \verb|texhash| or \verb|mktexlsr|.
%
% \subsection{Some details for the interested}
%
% \paragraph{Unpacking with \LaTeX{}.}
% The \xfile{.dtx} chooses its action depending on the format:
% \begin{description}
% \item[\plainTeX:] Run \docstrip{} and extract the files.
% \item[\LaTeX:] Generate the documentation.
% \end{description}
% If you insist on using \LaTeX{} for \docstrip{} (really,
% \docstrip{} does not need \LaTeX ), then inform the autodetect routine
% about your intention:
% \begin{quote}
%   \verb|latex \let\install=y\input{morefloats.dtx}|
% \end{quote}
% Do not forget to quote the argument according to the demands
% of your shell.
%
% \paragraph{Generating the documentation.\label{GenDoc}}
% You can use both the \xfile{.dtx} or the \xfile{.drv} to generate
% the documentation. The process can be configured by a
% configuration file \xfile{ltxdoc.cfg}. For instance, put the following
% line into this file, if you want to have A4 as paper format:
% \begin{quote}
%   \verb|\PassOptionsToClass{a4paper}{article}|
% \end{quote}
%
% \noindent An example follows how to generate the
% documentation with \pdfLaTeX :
%
% \begin{quote}
%\begin{verbatim}
%pdflatex morefloats.dtx
%makeindex -s gind.ist morefloats.idx
%pdflatex morefloats.dtx
%makeindex -s gind.ist morefloats.idx
%pdflatex morefloats.dtx
%\end{verbatim}
% \end{quote}
%
% \subsection{Compiling the example}
%
% The example file, \textsf{morefloats-example.tex}, can be compiled via\\
% |(pdf)latex morefloats-example.tex|.
%
% \section{Acknowledgements}
%
% \LaTeX{} 2015 provides the |\extrafloats| command.
% \textsc{Don Hosek}, Quixote, 1990/07/27 (Thanks!)
% invented the main code for handling more floats
% before |\extrafloats| was available.
% \textsc{David Carlisle} pointed the maintainer to the new |\extrafloats|
% and provided the code for |\extrafloats| in case |\extrafloats| is not yet
% available at the used system (Thanks!).
% The current maintainer is \textsc{H.-Martin M\"{u}nch}.\\
% I would like to thank additionally \textsc{Karl Berry} for helping with taking
% over the maintainership of this package and two missing |\expandafter|s,
% \textsc{Heiko Oberdiek} for providing a~lot~(!) of useful packages (from
% which I also got everything I know about creating a file in \xfile{dtx}
% format, ok, say it: copying), everybody of the CTAN team for managing
% CTAN, and the \Newsgroup{comp.text.tex} and \Newsgroup{de.comp.text.tex}
% newsgroups and everybody at \url{http://tex.stackexchange.com/}
% for their help in all things \hbox{\TeX{}.}
%
% \newpage
%
% \phantomsection
% \begin{History}\label{History}
%
%   Some old versions have been archived at
%   \url{http://ctanhg.scharrer-online.de/pkg/morefloats.html}.
%
%   \begin{Version}{1990/07/27 v1.0a}
%     \item Created by \textsc{Don Hosek}.
%   \end{Version}
%   \begin{Version}{2008/11/14 v1.0b}
%     \item \textsc{Clea F. Rees} added a license line.
%   \end{Version}
%   \begin{Version}{2010/09/20 v1.0c}
%     \item \xfile{.dtx} created by \textsc{H.-Martin M\"{u}nch}.
%     \item Included more documentation and alternatives.
%     \item Included options to allow the user to flexible choose the number
%             of floats from $18$ up to $256$ instead of fixed $36$.
%     \item Included an example file.
%     \item Created a \texttt{README} file.
%   \end{Version}
%   \begin{Version}{2011/02/01 v1.0d}
%     \item References to\\
%             \url{http://www.tex.ac.uk/cgi-bin/texfaq2html?label=figurehere} and\\
%             \url{http://mirror.ctan.org/obsolete/macros/latex/contrib/misc/morefloats.sty}
%             added.
%     \item Now using the \xpackage{lscape} package from the \xpackage{graphics}
%             bundle to print some pages of the documentation in landscape instead
%             of portrait mode, because they were way too wide. (\textit{Since v1.0e
%             replaced by \xpackage{pdflscape} package.})
%     \item Updated the version of the \xpackage{hyperref} package.
%             (\textit{Since version~1.0e the \xpackage{morefloats} package uses
%              a fixed version of the \xpackage{holtxdoc} package, which calls for
%              the right version of the \xpackage{hyperref} package, therefore
%              it is no longer necessary to give the recent version of the
%              \xpackage{hyperref} package here.})
%   \end{Version}
%   \begin{Version}{2011/07/10 v1.0e}
%     \item There is a new version of the used \xpackage{kvoptions} package.
%     \item Now using the \xpackage{pdflscape} package instead of the
%             \xpackage{lscape} package in the documentation.
%     \item The \xpackage{holtxdoc} package was fixed,
%             therefore the warning in \xfile{drv} could be removed.~-- Adapted
%             the style of this documentation to new \textsc{Oberdiek} \xfile{dtx}
%             style.
%   \end{Version}
%   \begin{Version}{2012/01/28 v1.0f}
%     \item Bug fix: wrong path given in the documentation, fixed.
%     \item Replaced |\global\edef| by |\xdef|.
%     \item No longer uses a counter for itself but temporary ones. (For the floats
%             of course inserts and therefore counts are still used.)
%     \item The number of available inserts is checked before the allocation.
%     \item Maximum number of floats/inserts is $256$, not $266$; corrected.
%     \item Quite some additional changes in the \xfile{dtx} and README files.
%   \end{Version}
%   \begin{Version}{2015/07/16 v1.0g}
%     \item Implemented the new |\extrafloats| from \LaTeX{} 2015 allowing
%            several hundreds of additional floats.
%     \item Update of documentation, README, and \xfile{dtx} internals.
%   \end{Version}
%   \begin{Version}{2015/07/22 v1.0h}
%     \item Handling of more floats depending on new{/}old \LaTeX{} format,
%            availability of \eTeX{} in the used distribution,
%            and loading of the \xpackage{etex} package
%            (before \xpackage{morefloats}{/}after \xpackage{morefloats}{/}not
%            at all) should now ensure that the maximum number
%            for available floats can be allocated.
%     \item The example file now uses a flexible number of floats.
%   \end{Version}
% \end{History}
%
% \bigskip
%
% When you find a mistake or have a suggestion for an improvement of this package,
% please send an e-mail to the maintainer, thanks! (Please see BUG REPORTS in the README.)
%
% \newpage
%
% \PrintIndex
%
% \Finale
\endinput|
% \end{quote}
% Do not forget to quote the argument according to the demands
% of your shell.
%
% \paragraph{Generating the documentation.\label{GenDoc}}
% You can use both the \xfile{.dtx} or the \xfile{.drv} to generate
% the documentation. The process can be configured by a
% configuration file \xfile{ltxdoc.cfg}. For instance, put the following
% line into this file, if you want to have A4 as paper format:
% \begin{quote}
%   \verb|\PassOptionsToClass{a4paper}{article}|
% \end{quote}
%
% \noindent An example follows how to generate the
% documentation with \pdfLaTeX :
%
% \begin{quote}
%\begin{verbatim}
%pdflatex morefloats.dtx
%makeindex -s gind.ist morefloats.idx
%pdflatex morefloats.dtx
%makeindex -s gind.ist morefloats.idx
%pdflatex morefloats.dtx
%\end{verbatim}
% \end{quote}
%
% \subsection{Compiling the example}
%
% The example file, \textsf{morefloats-example.tex}, can be compiled via\\
% |(pdf)latex morefloats-example.tex|.
%
% \section{Acknowledgements}
%
% \LaTeX{} 2015 provides the |\extrafloats| command.
% \textsc{Don Hosek}, Quixote, 1990/07/27 (Thanks!)
% invented the main code for handling more floats
% before |\extrafloats| was available.
% \textsc{David Carlisle} pointed the maintainer to the new |\extrafloats|
% and provided the code for |\extrafloats| in case |\extrafloats| is not yet
% available at the used system (Thanks!).
% The current maintainer is \textsc{H.-Martin M\"{u}nch}.\\
% I would like to thank additionally \textsc{Karl Berry} for helping with taking
% over the maintainership of this package and two missing |\expandafter|s,
% \textsc{Heiko Oberdiek} for providing a~lot~(!) of useful packages (from
% which I also got everything I know about creating a file in \xfile{dtx}
% format, ok, say it: copying), everybody of the CTAN team for managing
% CTAN, and the \Newsgroup{comp.text.tex} and \Newsgroup{de.comp.text.tex}
% newsgroups and everybody at \url{http://tex.stackexchange.com/}
% for their help in all things \hbox{\TeX{}.}
%
% \newpage
%
% \phantomsection
% \begin{History}\label{History}
%
%   Some old versions have been archived at
%   \url{http://ctanhg.scharrer-online.de/pkg/morefloats.html}.
%
%   \begin{Version}{1990/07/27 v1.0a}
%     \item Created by \textsc{Don Hosek}.
%   \end{Version}
%   \begin{Version}{2008/11/14 v1.0b}
%     \item \textsc{Clea F. Rees} added a license line.
%   \end{Version}
%   \begin{Version}{2010/09/20 v1.0c}
%     \item \xfile{.dtx} created by \textsc{H.-Martin M\"{u}nch}.
%     \item Included more documentation and alternatives.
%     \item Included options to allow the user to flexible choose the number
%             of floats from $18$ up to $256$ instead of fixed $36$.
%     \item Included an example file.
%     \item Created a \texttt{README} file.
%   \end{Version}
%   \begin{Version}{2011/02/01 v1.0d}
%     \item References to\\
%             \url{http://www.tex.ac.uk/cgi-bin/texfaq2html?label=figurehere} and\\
%             \url{http://mirror.ctan.org/obsolete/macros/latex/contrib/misc/morefloats.sty}
%             added.
%     \item Now using the \xpackage{lscape} package from the \xpackage{graphics}
%             bundle to print some pages of the documentation in landscape instead
%             of portrait mode, because they were way too wide. (\textit{Since v1.0e
%             replaced by \xpackage{pdflscape} package.})
%     \item Updated the version of the \xpackage{hyperref} package.
%             (\textit{Since version~1.0e the \xpackage{morefloats} package uses
%              a fixed version of the \xpackage{holtxdoc} package, which calls for
%              the right version of the \xpackage{hyperref} package, therefore
%              it is no longer necessary to give the recent version of the
%              \xpackage{hyperref} package here.})
%   \end{Version}
%   \begin{Version}{2011/07/10 v1.0e}
%     \item There is a new version of the used \xpackage{kvoptions} package.
%     \item Now using the \xpackage{pdflscape} package instead of the
%             \xpackage{lscape} package in the documentation.
%     \item The \xpackage{holtxdoc} package was fixed,
%             therefore the warning in \xfile{drv} could be removed.~-- Adapted
%             the style of this documentation to new \textsc{Oberdiek} \xfile{dtx}
%             style.
%   \end{Version}
%   \begin{Version}{2012/01/28 v1.0f}
%     \item Bug fix: wrong path given in the documentation, fixed.
%     \item Replaced |\global\edef| by |\xdef|.
%     \item No longer uses a counter for itself but temporary ones. (For the floats
%             of course inserts and therefore counts are still used.)
%     \item The number of available inserts is checked before the allocation.
%     \item Maximum number of floats/inserts is $256$, not $266$; corrected.
%     \item Quite some additional changes in the \xfile{dtx} and README files.
%   \end{Version}
%   \begin{Version}{2015/07/16 v1.0g}
%     \item Implemented the new |\extrafloats| from \LaTeX{} 2015 allowing
%            several hundreds of additional floats.
%     \item Update of documentation, README, and \xfile{dtx} internals.
%   \end{Version}
%   \begin{Version}{2015/07/22 v1.0h}
%     \item Handling of more floats depending on new{/}old \LaTeX{} format,
%            availability of \eTeX{} in the used distribution,
%            and loading of the \xpackage{etex} package
%            (before \xpackage{morefloats}{/}after \xpackage{morefloats}{/}not
%            at all) should now ensure that the maximum number
%            for available floats can be allocated.
%     \item The example file now uses a flexible number of floats.
%   \end{Version}
% \end{History}
%
% \bigskip
%
% When you find a mistake or have a suggestion for an improvement of this package,
% please send an e-mail to the maintainer, thanks! (Please see BUG REPORTS in the README.)
%
% \newpage
%
% \PrintIndex
%
% \Finale
\endinput|
% \end{quote}
% Do not forget to quote the argument according to the demands
% of your shell.
%
% \paragraph{Generating the documentation.\label{GenDoc}}
% You can use both the \xfile{.dtx} or the \xfile{.drv} to generate
% the documentation. The process can be configured by a
% configuration file \xfile{ltxdoc.cfg}. For instance, put the following
% line into this file, if you want to have A4 as paper format:
% \begin{quote}
%   \verb|\PassOptionsToClass{a4paper}{article}|
% \end{quote}
%
% \noindent An example follows how to generate the
% documentation with \pdfLaTeX :
%
% \begin{quote}
%\begin{verbatim}
%pdflatex morefloats.dtx
%makeindex -s gind.ist morefloats.idx
%pdflatex morefloats.dtx
%makeindex -s gind.ist morefloats.idx
%pdflatex morefloats.dtx
%\end{verbatim}
% \end{quote}
%
% \subsection{Compiling the example}
%
% The example file, \textsf{morefloats-example.tex}, can be compiled via\\
% |(pdf)latex morefloats-example.tex|.
%
% \section{Acknowledgements}
%
% \LaTeX{} 2015 provides the |\extrafloats| command.
% \textsc{Don Hosek}, Quixote, 1990/07/27 (Thanks!)
% invented the main code for handling more floats
% before |\extrafloats| was available.
% \textsc{David Carlisle} pointed the maintainer to the new |\extrafloats|
% and provided the code for |\extrafloats| in case |\extrafloats| is not yet
% available at the used system (Thanks!).
% The current maintainer is \textsc{H.-Martin M\"{u}nch}.\\
% I would like to thank additionally \textsc{Karl Berry} for helping with taking
% over the maintainership of this package and two missing |\expandafter|s,
% \textsc{Heiko Oberdiek} for providing a~lot~(!) of useful packages (from
% which I also got everything I know about creating a file in \xfile{dtx}
% format, ok, say it: copying), everybody of the CTAN team for managing
% CTAN, and the \Newsgroup{comp.text.tex} and \Newsgroup{de.comp.text.tex}
% newsgroups and everybody at \url{http://tex.stackexchange.com/}
% for their help in all things \hbox{\TeX{}.}
%
% \newpage
%
% \phantomsection
% \begin{History}\label{History}
%
%   Some old versions have been archived at
%   \url{http://ctanhg.scharrer-online.de/pkg/morefloats.html}.
%
%   \begin{Version}{1990/07/27 v1.0a}
%     \item Created by \textsc{Don Hosek}.
%   \end{Version}
%   \begin{Version}{2008/11/14 v1.0b}
%     \item \textsc{Clea F. Rees} added a license line.
%   \end{Version}
%   \begin{Version}{2010/09/20 v1.0c}
%     \item \xfile{.dtx} created by \textsc{H.-Martin M\"{u}nch}.
%     \item Included more documentation and alternatives.
%     \item Included options to allow the user to flexible choose the number
%             of floats from $18$ up to $256$ instead of fixed $36$.
%     \item Included an example file.
%     \item Created a \texttt{README} file.
%   \end{Version}
%   \begin{Version}{2011/02/01 v1.0d}
%     \item References to\\
%             \url{http://www.tex.ac.uk/cgi-bin/texfaq2html?label=figurehere} and\\
%             \url{http://mirror.ctan.org/obsolete/macros/latex/contrib/misc/morefloats.sty}
%             added.
%     \item Now using the \xpackage{lscape} package from the \xpackage{graphics}
%             bundle to print some pages of the documentation in landscape instead
%             of portrait mode, because they were way too wide. (\textit{Since v1.0e
%             replaced by \xpackage{pdflscape} package.})
%     \item Updated the version of the \xpackage{hyperref} package.
%             (\textit{Since version~1.0e the \xpackage{morefloats} package uses
%              a fixed version of the \xpackage{holtxdoc} package, which calls for
%              the right version of the \xpackage{hyperref} package, therefore
%              it is no longer necessary to give the recent version of the
%              \xpackage{hyperref} package here.})
%   \end{Version}
%   \begin{Version}{2011/07/10 v1.0e}
%     \item There is a new version of the used \xpackage{kvoptions} package.
%     \item Now using the \xpackage{pdflscape} package instead of the
%             \xpackage{lscape} package in the documentation.
%     \item The \xpackage{holtxdoc} package was fixed,
%             therefore the warning in \xfile{drv} could be removed.~-- Adapted
%             the style of this documentation to new \textsc{Oberdiek} \xfile{dtx}
%             style.
%   \end{Version}
%   \begin{Version}{2012/01/28 v1.0f}
%     \item Bug fix: wrong path given in the documentation, fixed.
%     \item Replaced |\global\edef| by |\xdef|.
%     \item No longer uses a counter for itself but temporary ones. (For the floats
%             of course inserts and therefore counts are still used.)
%     \item The number of available inserts is checked before the allocation.
%     \item Maximum number of floats/inserts is $256$, not $266$; corrected.
%     \item Quite some additional changes in the \xfile{dtx} and README files.
%   \end{Version}
%   \begin{Version}{2015/07/16 v1.0g}
%     \item Implemented the new |\extrafloats| from \LaTeX{} 2015 allowing
%            several hundreds of additional floats.
%     \item Update of documentation, README, and \xfile{dtx} internals.
%   \end{Version}
%   \begin{Version}{2015/07/22 v1.0h}
%     \item Handling of more floats depending on new{/}old \LaTeX{} format,
%            availability of \eTeX{} in the used distribution,
%            and loading of the \xpackage{etex} package
%            (before \xpackage{morefloats}{/}after \xpackage{morefloats}{/}not
%            at all) should now ensure that the maximum number
%            for available floats can be allocated.
%     \item The example file now uses a flexible number of floats.
%   \end{Version}
% \end{History}
%
% \bigskip
%
% When you find a mistake or have a suggestion for an improvement of this package,
% please send an e-mail to the maintainer, thanks! (Please see BUG REPORTS in the README.)
%
% \newpage
%
% \PrintIndex
%
% \Finale
\endinput|
% \end{quote}
% Do not forget to quote the argument according to the demands
% of your shell.
%
% \paragraph{Generating the documentation.\label{GenDoc}}
% You can use both the \xfile{.dtx} or the \xfile{.drv} to generate
% the documentation. The process can be configured by a
% configuration file \xfile{ltxdoc.cfg}. For instance, put the following
% line into this file, if you want to have A4 as paper format:
% \begin{quote}
%   \verb|\PassOptionsToClass{a4paper}{article}|
% \end{quote}
%
% \noindent An example follows how to generate the
% documentation with \pdfLaTeX :
%
% \begin{quote}
%\begin{verbatim}
%pdflatex morefloats.dtx
%makeindex -s gind.ist morefloats.idx
%pdflatex morefloats.dtx
%makeindex -s gind.ist morefloats.idx
%pdflatex morefloats.dtx
%\end{verbatim}
% \end{quote}
%
% \subsection{Compiling the example}
%
% The example file, \textsf{morefloats-example.tex}, can be compiled via\\
% |(pdf)latex morefloats-example.tex|.
%
% \section{Acknowledgements}
%
% \LaTeX{} 2015 provides the |\extrafloats| command.
% \textsc{Don Hosek}, Quixote, 1990/07/27 (Thanks!)
% invented the main code for handling more floats
% before |\extrafloats| was available.
% \textsc{David Carlisle} pointed the maintainer to the new |\extrafloats|
% and provided the code for |\extrafloats| in case |\extrafloats| is not yet
% available at the used system (Thanks!).
% The current maintainer is \textsc{H.-Martin M\"{u}nch}.\\
% I would like to thank additionally \textsc{Karl Berry} for helping with taking
% over the maintainership of this package and two missing |\expandafter|s,
% \textsc{Heiko Oberdiek} for providing a~lot~(!) of useful packages (from
% which I also got everything I know about creating a file in \xfile{dtx}
% format, ok, say it: copying), everybody of the CTAN team for managing
% CTAN, and the \Newsgroup{comp.text.tex} and \Newsgroup{de.comp.text.tex}
% newsgroups and everybody at \url{http://tex.stackexchange.com/}
% for their help in all things \hbox{\TeX{}.}
%
% \newpage
%
% \phantomsection
% \begin{History}\label{History}
%
%   Some old versions have been archived at
%   \url{http://ctanhg.scharrer-online.de/pkg/morefloats.html}.
%
%   \begin{Version}{1990/07/27 v1.0a}
%     \item Created by \textsc{Don Hosek}.
%   \end{Version}
%   \begin{Version}{2008/11/14 v1.0b}
%     \item \textsc{Clea F. Rees} added a license line.
%   \end{Version}
%   \begin{Version}{2010/09/20 v1.0c}
%     \item \xfile{.dtx} created by \textsc{H.-Martin M\"{u}nch}.
%     \item Included more documentation and alternatives.
%     \item Included options to allow the user to flexible choose the number
%             of floats from $18$ up to $256$ instead of fixed $36$.
%     \item Included an example file.
%     \item Created a \texttt{README} file.
%   \end{Version}
%   \begin{Version}{2011/02/01 v1.0d}
%     \item References to\\
%             \url{http://www.tex.ac.uk/cgi-bin/texfaq2html?label=figurehere} and\\
%             \url{http://mirror.ctan.org/obsolete/macros/latex/contrib/misc/morefloats.sty}
%             added.
%     \item Now using the \xpackage{lscape} package from the \xpackage{graphics}
%             bundle to print some pages of the documentation in landscape instead
%             of portrait mode, because they were way too wide. (\textit{Since v1.0e
%             replaced by \xpackage{pdflscape} package.})
%     \item Updated the version of the \xpackage{hyperref} package.
%             (\textit{Since version~1.0e the \xpackage{morefloats} package uses
%              a fixed version of the \xpackage{holtxdoc} package, which calls for
%              the right version of the \xpackage{hyperref} package, therefore
%              it is no longer necessary to give the recent version of the
%              \xpackage{hyperref} package here.})
%   \end{Version}
%   \begin{Version}{2011/07/10 v1.0e}
%     \item There is a new version of the used \xpackage{kvoptions} package.
%     \item Now using the \xpackage{pdflscape} package instead of the
%             \xpackage{lscape} package in the documentation.
%     \item The \xpackage{holtxdoc} package was fixed,
%             therefore the warning in \xfile{drv} could be removed.~-- Adapted
%             the style of this documentation to new \textsc{Oberdiek} \xfile{dtx}
%             style.
%   \end{Version}
%   \begin{Version}{2012/01/28 v1.0f}
%     \item Bug fix: wrong path given in the documentation, fixed.
%     \item Replaced |\global\edef| by |\xdef|.
%     \item No longer uses a counter for itself but temporary ones. (For the floats
%             of course inserts and therefore counts are still used.)
%     \item The number of available inserts is checked before the allocation.
%     \item Maximum number of floats/inserts is $256$, not $266$; corrected.
%     \item Quite some additional changes in the \xfile{dtx} and README files.
%   \end{Version}
%   \begin{Version}{2015/07/16 v1.0g}
%     \item Implemented the new |\extrafloats| from \LaTeX{} 2015 allowing
%            several hundreds of additional floats.
%     \item Update of documentation, README, and \xfile{dtx} internals.
%   \end{Version}
%   \begin{Version}{2015/07/22 v1.0h}
%     \item Handling of more floats depending on new{/}old \LaTeX{} format,
%            availability of \eTeX{} in the used distribution,
%            and loading of the \xpackage{etex} package
%            (before \xpackage{morefloats}{/}after \xpackage{morefloats}{/}not
%            at all) should now ensure that the maximum number
%            for available floats can be allocated.
%     \item The example file now uses a flexible number of floats.
%   \end{Version}
% \end{History}
%
% \bigskip
%
% When you find a mistake or have a suggestion for an improvement of this package,
% please send an e-mail to the maintainer, thanks! (Please see BUG REPORTS in the README.)
%
% \newpage
%
% \PrintIndex
%
% \Finale
\endinput