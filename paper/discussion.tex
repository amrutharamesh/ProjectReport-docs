\section{Discussion}
\label{sec:discussion}
The attacks done as part of this research are evidence that there are both providers that take security seriously and those who don't. By identifying through real-time attacks about the ones which do not implement the defenses, we can prove that there are both popular and lesser known providers alike in the mix. One might be tempted to argue that the login CSRF attacks described are not very plausible for some of the providers which openly show who is logged in and with what email ID. But imagine an attacker who has the resources to have a mail server at his disposal who can create mail IDs that are generic and do not give away information about the logged in user. This way the victim will be made to believe that it is a guest account and nothing else. For the redirect URI attacks, by leaking code and access token information to a subdirectory of a URI can pose significant risks if the said subdirectory is part of some website like a blog that might hold user content. The defenses for these attacks must be taken care of on the resource provider side. Most of the sites have the protection enabled, but there are still websites which do not have them enabled. The possible solutions could be following an exact redirect URI matching approach, incorporating two-step verification, and including a login CSRF token in the request.


