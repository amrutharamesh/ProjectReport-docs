\begin{abstract}

The Web has evolved drastically in the last few decades and so have the security and privacy issues related to it. We rely on websites to perform various daily tasks - both personal and professional, from paying our bills, communicating with friends and family to managing our bank accounts. This means that a user of the Web has multiple online accounts with its respective passwords. This inconvenience was alleviated by the introduction of Single Sign On (SSO) services integrated into these websites. This integration might use one of the available protocols like OAuth2.0, OpenID, OpenID Connect and so on. However, this integration has to be done with the utmost care by developers to avoid privacy risks. In this project, we focus on the potential risks and vulnerabilities of the OAuth2.0 protocol flow. CSRF is the most common vulnerability of OAuth that has already been studied. So we looked into some lesser known vulnerabilities, namely - Redirect URI and Login CSRF.  Previous work on these risks are studies and do not involve attacks. We conduct these attacks by designing a malicious web application for the victim and find evidence that such attacks are plausible and not insignificant. A number of these SSO providers are vulnerable and our findings highlight that even seemingly simple vulnerabilities can have significant privacy repercussions.

\end{abstract}
