\section{Introduction}
\label{sec:intro}
Single Sign On (SSO) is a way of authentication in which the user has the option to login to a website with one of the identity providers like Google, Facebook and so on. This helps the user sign on to a website without going to the trouble of creating a separate username and password. By using the same set of providers for different websites, the need to remember username/password combinations is obviated and user experience becomes better. This integration of SSO providers follow different procedures specific to each website. Every SSO provider also has different protocol implementations for integrating SSO on their website. Some of these protocols are OAuth1.0, OAuth2.0, OpenID, OpenID Connect and so on. Any website integrating an SSO option chooses one of these protocol implementations. OAuth2.0 protocol flow and implementation is the focus of our project. This protocol is an authorization protocol and involves three parties : \\	  
\textbf{Resource Owner: User}\\
The user is responsible for authorizing an application to access resources at the resource provider and this access is restricted by the "scope" of 			the authorization granted. Eg: Read or Write\\
\textbf{Resource/Authorization server: API}\\
The resource server is responsible for hosting the user's accounts and the authorization server verifies the user's identity. It then issues access to the 	application. \\
\textbf{Client: Application}\\
The client is the application that wants to access the user's account at the resource server. It first requires authorization from the user and then this is 	validated by the API. \\
There are also different flows that can be used to implement OAuth2.0. The implicit flow, the authorization code flow, resource owner password flow and client credentials flow. In our research, we are utilizing the authorization code flow for web server applications. In this flow, the client sends an authorization request to the resource server. The server prompts the user to log in and once the authorization is confirmed the user is redirected to the client with authorization code. This code is then exchanged for an access token by sending a request to the server and the token used for subsequent actions performed. 

Cross Site Request Forgery(CSRF) is the most widely studied OAuth2.0 vulnerability. In this, a user's session is hijacked by the attacker and used to perform actions that change some state on the user's account. The defense to this attack is to append a "state" parameter in the client's requests for authorization. This can be any randomly generated value that cannot be guessed by the attacker. Most of the earlier work has been conducted around this vulnerability, therefore we are not focusing on this for our research. There are other lesser known OAuth2.0 vulnerabilities that also have significant privacy impact. Two of them have been explored in this project : Login CSRF and Redirect URI attacks. In the login CSRF attack, the attacker is able to forge login requests with the attacker's credentials that the victim unknowingly uses. This way the attacker gets to know about the user's activities on his account. There are two redirect URI attacks. In the Redirect URI Mismatch attack, by manipulating the redirect URI provided as a parameter to the authorization request, the code and access token can be sent to a URI of the attacker's choice. This occurs because the resource provider does not check for the redirect URI mismatch between the one used for registering the application and the one sent in the authorization request.  In the Cover Redirect URI attack, the access token is directly sent to the location hash of the URL which exposes it to any third party content on a client. 

The vulnerabilities described have been studied and defenses have been proposed to mitigate them. These defense mechanisms include a CSRF token in the login form presented to the user and checking for redirect URI mismatch on the provider's end before giving the code to the client. Unfortunately, these preventive measures are not being followed by most providers and still remains a significant threat. 

The goal of this research was to investigate the weaknesses of OAuth2.0 implementation in real time.  To this end, we developed a malicious application that exploits these implementation flaws to carry out the different attacks. The contributions of this research are :
\begin{itemize}
\item First large scale study on the various OAuth2.0 providers and analyzing their security loopholes.
\item Performed the login CSRF, redirect URI mismatch and cover redirect attacks to identify the incorrect implementation scenarios.
\item We experimentally evaluate the attacks against 22 of the 33 OAuth2.0 providers and discovered different incorrect implementation patterns. This is described in detail in Section~\ref{sec:evaluation}
\end{itemize} The previous work lacks replication of these vulnerabilities in the wild. 


